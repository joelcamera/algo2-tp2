\section{Módulo Registro}

\subsection{Interfaz}

\textbf{MÓDULO Registro extiende a} \tadNombre{Módulo Diccionario Lineal(Campo, Dato)}.

\textbf{género}: registro


~


\subsubsection{Operaciones básicas de Registro}

\begin{Interfaz}

\InterfazFuncion{Campos}{\In{r}{registro}}{Conj(Campo)}
[true]
{$res \igobs$ Campos($r$)}
[$\Theta(1)$]
[Devuelve el conjunto de campos del registro.]
[Devuelve por copia. Como sabemos que por el contexto de uso que la cantidad de campos es acotada la complejidad es $\Theta(1)$.]

% ~

% \InterfazFuncion{Borrar?}{\In{criterio}{registo}, \In{reg}{registro} }{bool}
% [\#Campos($criterio$) $\equiv 1$]
% {$res \igobs$ Borrar?($criterio$, $reg$)}
% [$\Theta(L)$]
% [Devuelve true si coinciden todos los campos de los registros $criterio$ y $reg$.]
% []

~

\InterfazFuncion{CoincidenTodos}{\In{r1}{registro}, \In{cc}{conj(campo)}, \In{r2}{registro} }{bool}
[$cc \subseteq$ campos($r1$) $\cap$ campos($r2$)]
{$res \igobs$ CoincidenTodos($r1$, $cc$, $r2$)}
[$\Theta(L)$]
[Devuelve true si tienen todos los datos iguales ambos registros.]
[]

% ~

% \InterfazFuncion{CoincideAlguno}{\In{r1}{registro}, \In{cc}{conj(campo)}, \In{r2}{registro} }{bool}
% [$cc \subseteq$ Campos($r1$) $\cap$ Campos($r2$)]
% {$res \igobs$ CoincideAlguno($r1$, $cc$, $r2$)}
% [$\Theta(L)$]
% [Devuelve true si tienen al menos un dato igual ambos registros.]
% []

~

\InterfazFuncion{AgregarCampos}{\Inout{r1}{registro}, \In{r2}{registro}}{}
[$r1 \igobs r1_o$]
{$r1 \igobs$ AgregarCampos($r1_o$, $r2$)}
[$\Theta(L)$]
[Agrega los campos del registro $r2$ a $r1$ que $r1$ no posee.]
[]


\end{Interfaz}

~

\pagebreak

%=============================================

%EXTIENDE AL DICCIONARIO, NO SE NECESITA UNA REPRESENTACION NI REP/ABS, SON LOS DEL DICCIONARIO.


% \subsubsection{Representación de Registro}

% \begin{Estructura}{Registro}[estr]

% 	\begin{Tupla}[estr]
% 		\tupItem{registro}{diccLineal(campo, dato)}
% 	\end{Tupla}

% \end{Estructura}



% \subsubsection{Invariante de Representación}


% \Rep[estr][e]{

% }\mbox{}


% \subsubsection{Función de Abastracción}

% \Abs[estr]{registro}[e]{r}{
% 	($\forall c$: Nat) def?($c$,$r$) $\igobs$ Definido?($c$,$e$.registro) $\yluego$ (def?($c$,$r$) $\impluego$ obtener($c$,$r$) $\igobs$ Significado($c$, $r$.registro))
% }


\subsection{Algoritmos}


\begin{algorithm}[H]{{\textbf{iCampos}(\In{r}{dic})} $\to$ $res$ : $Conj(Campo)$}
	{\\ $\textbf{Pre}$ $\equiv$ True}
    	\begin{algorithmic}[1]

    		\State $itDicc(Campo, Dato): it \gets$ CrearIt$(r)$ \Comment $\Theta(1)$
			\State $Conj(Campo): res \gets$ Vacio$()$ \Comment $\Theta(1)$

			\While{HaySiguiente$(it)$} \Comment $\Theta(1)$

				\State AgregarRapido$(res,$ SiguienteClave$(it) )$ \Comment $\Theta(1)$

				\State Avanzar$(it)$ \Comment $\Theta(1)$
			
			\EndWhile \Comment $\Theta(1)$ Porque la cantidad de campos es acotada.

			\State $res \gets conj$ \Comment $\Theta(1)$ Devuelve por copia.

			\medskip
			\Statex \underline{Complejidad:} $\Theta(1)$
			\Statex \underline{Justificación:} Como la cantidad de campos es acotada en el contexto de uso de registro, iterarlos tiene como complejidad $\Theta(1)$ y devolverlo también tiene como complejidad $\Theta(1)$.
    	\end{algorithmic}
	{$\textbf{Post}$ $\equiv$ $res \igobs$ Campos$(r)$}
\end{algorithm}


% \begin{algorithm}[H]{{\textbf{iBorrar?}(\In{criterio}{dic}, \In{r}{dic})} $\to$ $res$ : $bool$}
% 	{\\ $\textbf{Pre}$ \#Campos($criterio$) $\equiv 1$}
%     	\begin{algorithmic}[1]	

%     		\State $res \gets$ CoincidenTodos($criterio$, Campos($criterio$), $r$) \Comment $\Theta(L)$
		
% 			\medskip
% 			\Statex \underline{Complejidad:} $\Theta(L)$
% 			\Statex \underline{Justificación:} Como se por requiere que \#Campos($criterio$) $\equiv 1$ y la complejidad de Campos y de CoincidenTodos es de $\Theta(L)$, la complejidad de esta función es de $\Theta(L)$.
%     	\end{algorithmic}
% 	{$\textbf{Post}$ $\equiv$ $res \igobs$ Borrar?($criterio$, $r$)}
% \end{algorithm}


\begin{algorithm}[H]{{\textbf{iCoincidenTodos}(\In{r1}{dic}, \In{cc}{Conj(Campo)}, \In{r2}{dic})} $\to$ $res$ : $bool$}
	{\\ $\textbf{Pre}$ $cc \subseteq$ Campos($r1$) $\cap$ Campos($r2$)}
    	\begin{algorithmic}[1]

    		\State $itConj(Campo): itC \gets$ CrearIt($cc$) \Comment $\Theta(1)$
    		\State $res \gets true$ \Comment $\Theta(1)$

    		\While{HaySiguiente($itC$) $\land$ $res$} \Comment $\Theta(1)$
    			
    			\State $res \gets$ Significado($r1$, Siguiente($itC$)) == Significado($r2$, Siguiente($itC$)) \Comment $\Theta(L)$
    			
    			\State $Avanzar(itC)$ \Comment $\Theta(1)$
    		
    		\EndWhile 
    		\Comment $\Theta(\#(cc)*L)$

			\medskip
			\Statex \underline{Complejidad:} $\Theta(\#(cc)*L)$ = $\Theta(L)$
			\Statex \underline{Justificación:} Itera sobre el conjunto de campos $cc$ y compara los significados de ese campo en ambos registros. Como por requiere se que $cc \subseteq$ Campos($r1$) $\cap$ Campos($r2$) y se que por contexto de uso que la cantidad de campos está acotada (por ende, Significado tiene complejidad $\Theta(1)$), entonces iterar sobre el conjunto cuesta $\Theta(1)$. Los significados del registro no están acotados, esto hace que la comparación entre ellos tengan complejidad $\Theta(L)$ donde $L$ es la máxima longitud de un valor STRING de un registro en la tabla pasada por parámetro.
    	\end{algorithmic}
	{$\textbf{Post}$ $\equiv$ $res \igobs$ CoincidenTodos($r1$, $cc$, $r2$)}
\end{algorithm}


% \begin{algorithm}[H]{{\textbf{iCoincideAlguno}(\In{r1}{dic}, \In{cc}{ItConj(Campo)}, \In{r2}{dic})} $\to$ $res$ : $bool$}
% 	{\\ $\textbf{Pre}$ $cc \subseteq$ Campos($r1$) $\cap$ Campos($r2$)}
%     	\begin{algorithmic}[1]

%     		\State $itConj(Campo): itC \gets$ CrearIt($cc$) \Comment $\Theta(1)$
%     		\State $res \gets false$ \Comment $\Theta(1)$

%     		\While{HaySiguiente($itC$) $\land$ $!res$} \Comment $\Theta(1)$

%     			\State $res \gets$  Significado($r1$, Siguiente($itC$)) == Significado($r2$, Siguiente($itC$)) \Comment $\Theta(L)$
%     			\State Avanzar($itC$) \Comment $\Theta(1)$
    		
%     		\EndWhile 
%     		\Comment $\Theta(\#(cc)*L)$

% 			\medskip
% 			\Statex \underline{Complejidad:} $\Theta(\#(cc)*L)$ = $\Theta(L)$
% 			\Statex \underline{Justificación:} Itera sobre el conjunto de campos $cc$ y compara los significados de ese campo en ambos registros. Como por requiere se que $cc \subseteq$ Campos($r1$) $\cap$ Campos($r2$) y se que por contexto de uso que la cantidad de campos está acotada (por ende, Significado tiene complejidad $\Theta(1)$), entonces iterar sobre el conjunto cuesta $\Theta(1)$. Los significados del registro no están acotados, esto hace que la comparación entre ellos tengan complejidad $\Theta(L)$ donde $L$ es la máxima longitud de un valor STRING de un registro en la tabla pasada por parámetro.
%     	\end{algorithmic}
% 	{$\textbf{Post}$ $\equiv$ $res \igobs$ CoincideAlguno($r1$, $cc$, $r2$)}
% \end{algorithm}


\begin{algorithm}[H]{{\textbf{iAgregarCampos}(\Inout{r1}{dic}, \In{r2}{dic})}}
	{\\ $\textbf{Pre}$ true}
    	\begin{algorithmic}[1]

    		\State $Conj(campo): cr1 \gets$ Campos($r1$) \Comment $\Theta(1)$
    		\State $Conj(campo): cr2 \gets$ Campos($r2$) \Comment $\Theta(1)$
    		\\

			\State $\backslash\backslash$ RestaCampos le quita a $cr2$ los campos que estan en $cr1$
			\State $\backslash\backslash$ al final quedaria $cr2 = cr2_o - (cr2_o \cap cr1)$

    		\State RestaCampos($cr2$, $cr1$) \Comment $\Theta(\#(cr1) * \#(cr2)) = \Theta(1)$
    		\\

    		\State $itConj(Campo): itC \gets$ CrearIt($cr2$) \Comment $\Theta(1)$
    	
    		\While{HaySiguiente($itC$)} \Comment $\Theta(1)$

    			\\
    			\State $\backslash\backslash$ Agrego a $r1$ los registros de $r2$ que no están en $r1$.
    			\State $\backslash\backslash$ Para esto utilzo el conjunto $cr2$ que tiene los campos que no tienen en común $r1$ y $r2$.
    			\State $\backslash\backslash$ Ya que quite los registros en común cuando utilizamos la función RestaCampos.
    			\State $\backslash\backslash$ Cumple el pre de DefinirRapido, estoy agregando claves que no están definidas.
    			\State DefinirRapido($r1$, Siguiente($itC$), Significado($r2$, Siguiente($itC$))) \Comment $\Theta(copy(s))$
    			\\
    			
    			\State Avanzar($itC$) \Comment $\Theta(1)$
    			\\
    		\EndWhile 
    		\Comment $\Theta(\#(cr2)*copy(s)) = \Theta(copy(s))$
		
			\medskip
			\Statex \underline{Complejidad:} $\Theta(copy(s))$ = $\Theta(L)$
			\Statex \underline{Justificación:} Se crean dos conjuntos de campos que cuestan $\Theta(1)$ cada uno, dado que la cantidad de campos de cada registro está acotada por el contexto de uso. Luego se utiliza la función RestaCampos que filtra los campos de $r1$ que están en $r2$. La complejidad de esta función es $\Theta(\#(cr1) * \#(cr2))$ pero como los conjuntos de campos son los campos de los registros $r1$ y $r2$ y estos, por el contexto de uso, tienen una cantidad de campos acotada sumado a que los campos son STRINGS acotados, esto hace que la funcion RestaCampos termine teniendo una complejidad de $\Theta(1)$. Despues se itera el conjunto $cr2$ que posee los campos que poseía menos la intersección de este conjunto con $cr1$, y como la cantidad del conjunto está acotada, iterar este conjunto tiene como complejidad $\Theta(1)$. Dentro del ciclo se utiliza la función DefinirRapido que define en un diccionario en $\Theta(copy(s))$ = $\Theta(L)$ ya que los significados no están acotados, y cumple con el requiere de la función ya que los elementos que se están agregando son los que no están definidos en $r1$ de $r2$. Y, también, utiliza la función Significado del registro $r2$ que, como la cantidad de campos es acotada, tiene como complejidad $\Theta(1)$. Por lo tanto, la función termina teniendo como complejidad $\Theta(L)$ que surge de generar la copia del significado al definir en el registro.

    	\end{algorithmic}
	{$\textbf{Post}$ $\equiv$ $res \igobs$ AgregarCampos($r1$, $r2$)}
\end{algorithm}


\begin{algorithm}[H]{{\textbf{RestaCampos}(\Inout{cr2}{Conj(Campo)}, \In{cr1}{Conj(Campo)})}}
	{\\ $\textbf{Pre}$ $cr2 = cr2_o$}
    	\begin{algorithmic}[1]

    		\State $itConj(campo): itCr2 \gets$ CrearIt($cr2$) \Comment $\Theta(1)$

    		\While{HaySiguiente($itCr2$)} \Comment $\Theta(1)$

    			\If {Pertenece?($cr1$, Siguiente($itCr2$))} \Comment $\Theta(\#(cr1))$
    				\State EliminarSiguiente($itCr2$) \Comment $\Theta(1)$
    			\Else
    				\State Avanzar($itCr2$) \Comment $\Theta(1)$
    			\EndIf

    		\EndWhile \Comment $\Theta(\#(cr1) * \#(cr2))$
		
			\medskip
			\Statex \underline{Complejidad:} $\Theta(\#(cr1) * \#(cr2))$
			\Statex \underline{Justificación:} Recorre todos los campos del conjunto $cr2$ revisando si pertenecen al conjunto $cr1$. La funcion Pertence? de conjunto hace una comparacion de todos los elementos revisando si está el pasado como parametro, esto hace que la complejidad sea $\Theta\left(\displaystyle\sum_{a' \in c}equal(a,a')\right)$, pero como los campos tienen una longitud acotada de letras la complejidad de cada comparación es $\Theta(1)$ y la complejidad total del pertenece termina siendo $\Theta(\#cr1)$ porque en peor caso debe recorrer todos los elementos.
    	\end{algorithmic}
	{$\textbf{Post}$ $\equiv$ $cr2 = cr2_o - cr1$)}
\end{algorithm}