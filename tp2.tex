\documentclass[10pt, a4paper, spanish]{article}
\usepackage[paper=a4paper, left=1.5cm, right=1.5cm, bottom=1.5cm, top=3.5cm]{geometry}
\usepackage[spanish]{babel}
\selectlanguage{spanish}
\usepackage[utf8]{inputenc}
\usepackage[T1]{fontenc}
\usepackage{indentfirst}
\usepackage{fancyhdr}
\usepackage{latexsym}
\usepackage{lastpage}
\usepackage{aed2-symb,aed2-itef,aed2-tad,caratula}
\usepackage[colorlinks=true, linkcolor=blue]{hyperref}
\usepackage{calc}
\usepackage{ifthen}
\usepackage{caratula/caratula}

\usepackage{xspace}
\usepackage{xargs}
\usepackage{algorithm}% http://ctan.org/pkg/algorithms
\usepackage{algpseudocode}% http://ctan.org/pkg/algorithmicx
\usepackage{verbatim}
\usepackage{listings}

% Estilo para Algoritmos
\lstdefinestyle{alg}{tabsize=4, frame=single, escapeinside=\'\', framesep=10pt}

\newcommand{\alg}[3]{\hangindent=\parindent#1 (#2) \ifx#3\empty\else$\rightarrow$ res: #3\fi}
\newcommand\ote[1]{\hspace*{\fill}~\mbox{O(#1)}\penalty -9999 }
\newcommand\ofi[1]{\ensuremath{\textbf{Complejidad}: #1}}

% Afanado
\newcommand{\moduloNombre}[1]{\textbf{#1}}

\let\NombreFuncion=\textsc
\let\TipoVariable=\texttt
\let\ModificadorArgumento=\textbf
\newcommand{\res}{$res$\xspace}
\newcommand{\tab}{\hspace*{7mm}}

\newcommandx{\TipoFuncion}[3]{%
  \NombreFuncion{#1}(#2) \ifx#3\empty\else $\to$ \res\,: \TipoVariable{#3}\fi%
}
\newcommand{\In}[2]{\ModificadorArgumento{in} \ensuremath{#1}\,: \TipoVariable{#2}\xspace}
\newcommand{\Out}[2]{\ModificadorArgumento{out} \ensuremath{#1}\,: \TipoVariable{#2}\xspace}
\newcommand{\Inout}[2]{\ModificadorArgumento{in/out} \ensuremath{#1}\,: \TipoVariable{#2}\xspace}
\newcommand{\Aplicar}[2]{\NombreFuncion{#1}(#2)}

\newlength{\IntFuncionLengthA}
\newlength{\IntFuncionLengthB}
\newlength{\IntFuncionLengthC}
%InterfazFuncion(nombre, argumentos, valor retorno, precondicion, postcondicion, complejidad, descripcion, aliasing)
\newcommandx{\InterfazFuncion}[9][4=true,6,7,8,9]{%
  \hangindent=\parindent
  \TipoFuncion{#1}{#2}{#3}\\%
  \textbf{Pre} $\equiv$ \{#4\}\\%
  \textbf{Post} $\equiv$ \{#5\}%
  \ifx#6\empty\else\\\textbf{Complejidad:} #6\fi%
  \ifx#7\empty\else\\\textbf{Descripción:} #7\fi%
  \ifx#8\empty\else\\\textbf{Aliasing:} #8\fi%
  \ifx#9\empty\else\\\textbf{Requiere:} #9\fi%
}

\newenvironment{Interfaz}{%
  \parskip=2ex%
  \noindent\textbf{\Large Interfaz}%
  \par%
}{}

\newenvironment{Representacion}{%
  \vspace*{2ex}%
  \noindent\textbf{\Large Representación}%
  \vspace*{2ex}%
}{}

\newenvironment{Algoritmos}{%
  \vspace*{2ex}%
  \noindent\textbf{\Large Algoritmos}%
  \vspace*{2ex}%
}{}


\newcommand{\Titulo}[1]{
  \vspace*{1ex}\par\noindent\textbf{\large #1}\par
}

\newenvironmentx{Estructura}[2][2={estr}]{%
  \par\vspace*{2ex}%
  \TipoVariable{#1} \textbf{se representa con} \TipoVariable{#2}%
  \par\vspace*{1ex}%
}{%
  \par\vspace*{2ex}%
}%

\newboolean{EstructuraHayItems}
\newlength{\lenTupla}
\newenvironmentx{Tupla}[1][1={estr}]{%
    \settowidth{\lenTupla}{\hspace*{3mm}donde \TipoVariable{#1} es \TipoVariable{tupla}$($}%
    \addtolength{\lenTupla}{\parindent}%
    \hspace*{3mm}donde \TipoVariable{#1} es \TipoVariable{tupla}$($%
    \begin{minipage}[t]{\linewidth-\lenTupla}%
    \setboolean{EstructuraHayItems}{false}%
}{%
    $)$%
    \end{minipage}
}

\newcommandx{\tupItem}[3][1={\ }]{%
    %\hspace*{3mm}%
    \ifthenelse{\boolean{EstructuraHayItems}}{%
        ,#1%
    }{}%
    \emph{#2}: \TipoVariable{#3}%
    \setboolean{EstructuraHayItems}{true}%
}

\newcommandx{\RepFc}[3][1={estr},2={e}]{%
  \tadOperacion{Rep}{#1}{bool}{}%
  \tadAxioma{Rep($#2$)}{#3}%
}%

\newcommandx{\Rep}[3][1={estr},2={e}]{%
  \tadOperacion{Rep}{#1}{bool}{}%
  \tadAxioma{Rep($#2$)}{true \ssi #3}%
}%

\newcommandx{\Abs}[5][1={estr},3={e}]{%
  \tadOperacion{Abs}{#1/#3}{#2}{Rep($#3$)}%
  \settominwidth{\hangindent}{Abs($#3$) \igobs #4: #2 $\mid$ }%
  \addtolength{\hangindent}{\parindent}%
  Abs($#3$) \igobs #4: #2 $\mid$ #5%
}%

\newcommandx{\AbsFc}[4][1={estr},3={e}]{%
  \tadOperacion{Abs}{#1/#3}{#2}{Rep($#3$)}%
  \tadAxioma{Abs($#3$)}{#4}%
}%


%FIN Afanado

\newcommand{\f}[1]{\text{#1}}
\renewcommand{\paratodo}[2]{$\forall~#2$: #1}

\sloppy

\hypersetup{%
 % Para que el PDF se abra a página completa.
 pdfstartview= {FitH \hypercalcbp{\paperheight-\topmargin-1in-\headheight}},
 pdfauthor={Grupo 2 - Algoritmos y Estructuras de Datos II},
 pdfkeywords={Trabajo Practico 2},
 pdfsubject={Diseño}
}

\parskip=5pt % 5pt es el tamaño de fuente

% Pongo en 0 la distancia extra entre ítemes.
\let\olditemize\itemize
\def\itemize{\olditemize\itemsep=0pt}

% Acomodo fancyhdr.
\pagestyle{fancy}
\thispagestyle{fancy}
\addtolength{\headheight}{1pt}
\lhead{Algoritmos y Estructuras de Datos II}
\rhead{$1^{\mathrm{er}}$ cuatrimestre de 2016}
\cfoot{\thepage /\pageref{LastPage}}
\renewcommand{\footrulewidth}{0.4pt}

\author{Grupo 2 - Algoritmos y Estructuras de Datos II}
\date{1er cuatrimestre 2016}
\title{RTP2}

\begin{document}


%caratula
\materia{Algoritmos y Estructuras de Datos II}
\titulo{RTP2}
\grupo{Grupo Número 2}

\integrante {German Ariel Cuacci}{609/14}{germancuacci@gmail.com}
\integrante {Joel Esteban Cámera}{257/14}{joel.e.camera@gmail.com}
\integrante {Martin Jonas}{180/05}{martinjonas@gmail.com}

\maketitle
\pagebreak

%Indice
\tableofcontents

\pagebreak
\section{Módulo Diccionario String($\alpha$)}

Se representa mediante un árbol n-ario con invariante de trie. Las claves son strings y permite acceder a un significado en tiempo de la longitud de la clave y definir un significado en el mismo tiempo más el tiempo de copy(s) ya que los significados se almacenan por copia.

\subsection{Interfaz}

\textbf{parametros formales}

\textbf{géneros}: \TipoVariable{$\alpha$}.

\textbf{funcion}: \InterfazFuncion{Copiar}{\In{s}{$\alpha$}}{$\alpha$}
				  [true]
				  {res $=_{obs}$ s}
				  [$O(copy(s))$]
				  [funcion de copia de $\alpha$.]
				  []

\textbf{se explica con}: \tadNombre{Diccionario(String,$\alpha$)},\tadNombre{IteradorBidireccional(String)}.

\textbf{géneros}: \TipoVariable{diccString($\alpha$)}.



~

\subsubsection{Operaciones básicas de Diccionario String($\alpha$)}

\InterfazFuncion{CrearDiccionario}{}{}
{res $=_{obs}$ vacío()}
[$O(1)$]
[Crea un diccionario vacío.]
[]

~



\InterfazFuncion{Definido?}{\In{d}{diccString$(\alpha)$}, \In{c}{string})}{bool}
[true]
{$res$ $\igobs$ def?($d$, $c$)}
[$O(long(c))$]
[Devuelve true si la clave está definida en el diccionario y false en caso contrario.]
[]

~

\InterfazFuncion{Definir}{\Inout{d}{diccString$(\alpha)$}, \In{c}{string}, \In{s}{$\alpha$}}{}
[$ d \igobs d_0 $]
{$ d \igobs$ definir($c$, $s$, $d_0$)}
[O(L + copy(s)) donde L es el largo de la clave mas larga definida en el diccionario]
[Define la clave $c$ con el significado $s$.]
[Se almacena una copia de $s$.]

~

\InterfazFuncion{Obtener}{\In{d}{diccString$(\alpha)$}, \In{c}{string}}{$\alpha$}
[def?($c$, $d$)]
{alias($res$ $\igobs$ obtener($c$, $d$))}
[$\Theta(long(c))$]
[Devuelve el significado correspondiente a la clave $c$.]
[Se devuelve por rerefencia el significado almacenado en el diccionario.]

~

\InterfazFuncion{Borrar}{\Inout{d}{diccString($\alpha$)}, \In{c}{string}}{}
[$ d \igobs d_0 \land $ def?($d$, $c$)]
{$ d \igobs$ borrar($d_0$, $c$)}
[O(L) donde L es el largo de la clave mas larga definida en el diccionario]
[Borra la clave $c$ del diccionario y su significado.]
[]

~

\InterfazFuncion{CrearItClaves}{\In{d}{diccString($\alpha$)}}{itConj(String)}
[true]
{res $\igobs$ CrearItBi(claves(d)) $\land$ alias(esPermutacion?(SecuSuby(res), claves(d)))}
[$\Theta(1)$]
[Devuelve un iterador a las claves del diccionario]
[El iterador es no modificable y se invalida luego de eliminar o definir una clave en el diccionario]

~

\InterfazFuncion{ClaveMinima}{\In{d}{diccString($\alpha$)}}{string}
[true]
{\IF \#(claves(d))=0 THEN res = <> ELSE \\
 res $\in$ claves(d) $\yluego$  $(\forall s:\text{string})$ s $\in$ claves(d) $\land$ s $\neq$ res $\implies$ res $<$ s  FI}
[$\Theta(1)$]
[Devuelve la clave minima del diccionario, utilizando el orden lexicografico, o el string vacio si no hay claves definidas]
[Se deuvelve una referencia no modificable a la clave]

~

\InterfazFuncion{ClaveMaxima}{\In{d}{diccString($\alpha$)}}{string}
[true]
{\IF \#(claves(d))=0 THEN res = <> ELSE \\
 res $\in$ claves(d) $\yluego$  $(\forall s:\text{string})$ s $\in$ claves(d) $\land$ s $\neq$ res $\implies$ res $>$ s  FI}
[$\Theta(1)$]
[Devuelve la clave maxima del diccionario, utilizando el orden lexicografico, o el string vacio si no hay claves definidas]
[Se deuvelve una referencia no modificable a la clave]

~



\pagebreak

\subsubsection{Representación}

\begin{Estructura}{ diccString($\alpha$)}[estr]
	\begin{Tupla}[estr]
		\tupItem{raiz}{puntero(Nodo)}
		\tupItem{\\claves}{conj(string)}
		\tupItem{\\min}{string}
		\tupItem{\\max}{string}
	\end{Tupla}

	~

	\begin{Tupla}[Nodo]
		\tupItem{hijos}{arreglo(puntero(Nodo))}%
		\tupItem{\\	info}{puntero($\alpha$)}%
		\tupItem{\\ clave}{itConj(string)}
		\tupItem{\\ padre}{puntero(nodo)}
		\tupItem{\\ idx}{nat}
	\end{Tupla}

\end{Estructura}

\subsubsection{Invariante de Representación}


%\renewcommand{\labelenumi}{(\Roman{enumi})}

\begin{enumerate}
	\item Nodo representa un arbol con invariarnte de trie de 256 posiciones
	\item Min es la minima clave de las claves del diccionario si hay alguna, sino es un string vacio.
	\item Max es la maxima clave de las claves del diccionario si hay alguna, sino es un string vacio.
	\item Idx del nodo es un entero menor que 256
	\item Padre apunta al nodo del cual es hijo en la posicion idx, salvo que sea raiz y en ese caso es NULL 
	\item Desreferenciando el padre de un nodo siempre se llega a la raiz y nunca se repite un nodo.
	\item Un nodo puede tener significado, o representar los caracteres de la clave que conducen a un significado.
	\item Si un nodo tiene significado, el puntero info apunta al significado, sino es NULL
	\item Si un nodo tiene significado, clave es un iterador de las claves del diccionario cuyo siguiente es la clave en la que esta definido.
	\item Todos los elementos del arreglo hijos de un nodo, son NULL o son nodos validos

\end{enumerate}



\Rep[estr][e]{
	\\(1) $\lnot$(e.raiz = NULL) $\yluego$ nodoValido(e.raiz)
	\\(2) \IF \#(e.claves)=0 THEN e.min = <> ELSE  e.min $\in$ e.claves $\yluego$  $(\forall s:\text{string})$ s $\in$ e.claves $\land$ s $\neq$ e.min $\implies$ e.min $<$ s  FI
	\\(3) \IF \#(e.claves)=0 THEN e.max = <> ELSE e.max $\in$ e.claves $\yluego$  $(\forall s:\text{string})$ s $\in$ e.claves $\land$ s $\neq$ e.max $\implies$ e.max $>$ s  FI
}\mbox{}


~
\tadOperacion{nodoValido}{puntero(Nodo)/nodo,estr/e}{bool}{}
\tadAxioma{nodoValido($nodo$, $e$)}{
	(1) tam(nodo$\rightarrow$hijos) = 256 $\yluego$ \\
	(4) $nodo \rightarrow idx$ $<$ 256 $\yluego$ \\ 
	(5) $nodo \rightarrow padre$ = NULL $\iff$ e.raiz = $nodo$ $\yluego$ \\
	(7,8,9)$\lnot$($nodo \rightarrow padre$ = NULL) $\implies$ $nodo \rightarrow padre \rightarrow hijos[ nodo \rightarrow idx ] $ = $nodo$ $\yluego$ \\
	reconstruirClave($nodo$) $\in$ e.claves $\iff$  $ \lnot (nodo \rightarrow  info = NULL)$ $\land$ siguiente(*($nodo \rightarrow clave$)) = reconstruirClave($nodo$)
	\\(10)	($\forall i$ Nat) i < 256 $\yluego$ $\lnot$ ($nodo \rightarrow hijos [i] = NULL) \impluego $  nodoValido($nodo \rightarrow hijos [i] $, e)
}


\tadOperacion{reconstruirClave}{puntero(Nodo)/nodo}{string}{}
\tadAxioma{reconstruirClave($nodo$)}{ \IF $nodo \rightarrow padre = NULL $ THEN <> ELSE  reconstruirClave($nodo \rightarrow padre$) $\circulito$ $Ord^{-1}$($nodo \rightarrow idx$) FI }



\subsubsection{Funci\'on de Abstracci\'on}

\Abs[estr]{diccString($\alpha$)}[e]{d}{
($\forall s$: string) ( (def?($s$, $d$) $\iff$ s $\in$ e.claves) $\yluego$ \\ 
	(def?($s$, $d$) $\impluego$ obtener($s$, $d$) $\igobs$ obtenerSignificado(s, e.raiz) ) 
}

\tadOperacion{obtenerSignificado}{String/s,puntero(Nodo)/nodo}{$\alpha$}{}
\tadAxioma{obtenerSignificado(s,nodo)}{ \IF Vacio?(s) THEN *($nodo \rightarrow info$) ELSE 
obtenerSignificado(Fin(s), $nodo \rightarrow hijos$[Ord(prim(s))]) FI }



\subsection{Algoritmos}

%\lstset{style=alg}


\begin{algorithm}[H]{\textbf{iCrearDiccionario}() $\to$ $res$ : estr}
	\begin{algorithmic}[1]
		\State estr $res$ $\gets$ $\langle$ CrearNodo(NULL, 0), Vacio(), Vacia(), Vacia() $\rangle$ \Comment O(1)
		\medskip
		\Statex \underline{Complejidad:} O(1)
		\Statex \underline{Justificación:} CrearNodo es en O(1) y crear una tupla de contenedores vacios es O(1).
    \end{algorithmic}
\end{algorithm}

\begin{algorithm}[H]{\textbf{iCrearNodo}(\In{padre}{puntero(Nodo)},\In{idx}{nat}) $\to$ $res$ : puntero(nodo)}
	{\\ $\textbf{Pre}$ $\equiv$ 0 $\leq$ idx < 256}
	\begin{algorithmic}[1]

		\State nodo n $\gets$ $\langle$ CrearArreglo(256), NULL, CrearIt(Vacio()), padre, idx $\rangle$  \Comment O(256) $\in$ O(1)
		\State nat i $\gets$ 0 					\Comment $O(1)$
		\While{i < 256} 								\Comment O(256) $\in$ O(1)
		 	\State n.hijos[i] $\gets$ NULL 		\Comment $O(1)$
			\State $i \gets i + 1$ \Comment $O(1)$
		\EndWhile
		\State res $\gets$ \&(n)          \Comment O(1)

		\medskip
		\Statex \underline{Complejidad:} O(1)
		\Statex \underline{Justificación:} Sumas de O(1)
    \end{algorithmic}
    {$\textbf{Post}$ $\equiv$ *res $\igobs$ $\langle$ CrearArreglo(256), NULL, CrearIt($\emptyset$), padre, idx $\rangle$ $\yluego$ \\ ($\forall$ i : Nat) i $<$ 256 $\impluego$ res$\rightarrow$ hijos[i] = NULL  }

\end{algorithm}


\begin{algorithm}[H]{\textbf{iBuscarYCrear}(\Inout{n}{puntero(Nodo)},\In{clave}{string},\In{crear?}{bool}) $\to$ $res$ : puntero(nodo)}
	{\\ $\textbf{Pre}$ $\equiv$ $n \igobs n_0$ $\land$ ($n_0 \neq$ NULL)  }
	\begin{algorithmic}[1]
		\State nat i $\gets$ 0  									\Comment O(1)

		\While{i $<$ longitud(clave) $\land$ $\lnot$(n = NULL) } 	\Comment O(long(clave)) 
			\State nat idx $\gets$ ord(clave[i])								\Comment O(1)
			\If{$n \rightarrow hijos$[idx] = NULL $\land$ crear?} 		\Comment O(1)
				\State $n \rightarrow hijos$[idx] $\gets$ CrearNodo(n, idx)		\Comment O(1)
			\EndIf	
					
			\State n $\gets$ $n \rightarrow hijos$[idx]		\Comment O(1)
			\State $i \gets i + 1$ 											\Comment O(1)
		\EndWhile
		\State res $\gets$ n					\Comment O(1)
		\medskip
		\Statex \underline{Complejidad:} O(long(clave))
		\Statex \underline{Justificación:} El ciclo se ejecuta a lo sumo una cantidad de veces igual a la longitud de la clave, y las operaciones que se realizan son O(1).
    \end{algorithmic}
    {$\textbf{Post}$ $\equiv$ \\
     \IF BuscaNodo($n_0$, $clave$) = NULL $\land$ crear? THEN   
     (n se modifica, creando recursivamente nodos nuevos donde n$\rightarrow$hijos[$c_i$] = NULL) $\yluego$
    	res $\igobs$ BuscaNodo($n$, $clave$) 
     ELSE 
     res $\igobs$ BuscaNodo($n_0$, $clave$) $\land$ n $\igobs$ $n_0$ 
     FI }
\end{algorithm}


    
\tadOperacion{BuscaNodo}{puntero(Nodo)/p, String/c}{puntero(Nodo)}{}

\tadAxioma{BuscaNodo(p,c)}{
	\IF p = NULL THEN NULL
	ELSE{
		\IF long(c) = 1 THEN p$\rightarrow$hijos[orden(prim(c))]
		ELSE BuscaNodo(p$\rightarrow$hijos[ord(prim(c))], ult(c))
		FI
	}
	FI
}


\begin{algorithm}[H]{\textbf{iDefinido?}(\In{d}{estr)}, \In{clave}{string}) $\to$ $res$ : $bool$}
	\begin{algorithmic}[1]
		\State puntero(nodo) n $\gets$ BuscarYCrear(d.raiz, clave, false)     \Comment O(long(clave))
		\State res $\gets$ $\lnot$(n = NULL) $\yluego$ $\lnot$(n $\rightarrow$ info = NULL)  \Comment O(1)
		\medskip
		\Statex \underline{Complejidad:} O(long(clave))
		\Statex \underline{Justificación:} La complejidad esta determinada por la operacion BuscarYCrear
    \end{algorithmic}
\end{algorithm}


\begin{algorithm}[H]{\textbf{iDefinir}(\Inout{d}{estr}, \In{clave}{string}, \In{s}{$\alpha$})}
	\begin{algorithmic}[1]

		\State puntero(nodo) n $\gets$ BuscarYCrear(d.raiz, clave, true)     \Comment $\Theta$(long(clave))
		\If{ d.info = NULL} 													\Comment O(1)
			\State d.info $\gets$ \&( copy(s)	)					\Comment $\Theta$(copy(s))
			\State d.clave $\gets$ AgregarRapido(d.claves, clave)			\Comment $\Theta$(copy(clave))
			\State actualizarMin(d)		\Comment O(max(longs(claves(d))))
			\State actualizarMax(d)		\Comment O(max(longs(claves(d))))
		\Else
			\State d.info $\gets$ NULL							\Comment O(1)
			\State d.info $\gets$ \&( copy(s)	)					\Comment $\Theta$(copy(s))
		\EndIf

		\medskip
		\Statex \underline{Complejidad:} O(L + copy(s)) Donde L es la clave mas larga definida en el diccionario.
		\Statex \underline{Justificación:} La complejidad esta dada por suma de otras operaciones, donde 
		$\Theta$(long(clave)) y $\Theta$(copy(clave)) $\in$ O(L), con lo cual la suma es entre las dos determinantes independientes L y copy(s) 
    \end{algorithmic}
\end{algorithm}



\begin{algorithm}[H]{\textbf{iObtener}(\In{d}{estr}, \In{clave}{string}) $\to$ $res$ : $\alpha$}
	\begin{algorithmic}[1]
		\State puntero(nodo) n $\gets$ BuscarYCrear(d.raiz, clave, false)     \Comment $\Theta$(long(clave))
		\State res $\gets$ *(n $\rightarrow$ info)									    \Comment O(1)
		\medskip
		\Statex \underline{Complejidad:} $\Theta$(long(clave))
		\Statex \underline{Justificación:} La complejidad esta determinada por la operacion BuscarYCrear ya que el resultado se devuelve por referencia, como para usar obtener tiene que estar definidio, siempre es $\Theta$(long(clave))
    \end{algorithmic}
\end{algorithm}


\begin{algorithm}[H]{\textbf{iBorrar}(\Inout{d}{estr}, \In{clave}{string})}
	\begin{algorithmic}[1]

		\State puntero(nodo) n $\gets$ BuscarYCrear(d.raiz, clave, false)     \Comment O(long(clave))
		\State $n \rightarrow info$ $\gets$ NULL 					\Comment O(1)
		\State eliminarSiguiente(n.clave)							\Comment se elimina la clave del conjunto de claves O(1)
		\State n.clave $\gets$ CrearIt(Vacio())				\Comment se apunta a null O(1)
		\While{$\lnot$(n = NULL) $\yluego$ $\lnot$(tieneHijos(n))}	\Comment por Rep, O(long(clave)) * O(1)
			\State puntero(nodo) padre $\gets$ $n \rightarrow padre$		\Comment O(1)
			\If{$\lnot$(padre = NULL)} 														\Comment O(1)
				\State padre[$n \rightarrow idx$] $\gets$ NULL			\Comment se elimina el nodo O(1)					
			\EndIf
			\State n $\gets$ padre								\Comment O(1)
		\EndWhile
	
		\State actualizarMin(d)		\Comment O(max(longs(claves(d))))
		\State actualizarMax(d)		\Comment O(max(longs(claves(d))))

		\medskip
		\Statex \underline{Complejidad:} O(L) Donde L es la clave mas larga definida en el diccionario.
		\Statex \underline{Justificación:} El algoritmo recorre desde el nodo hasta la raiz, eliminando si es necesario los nodos intermedios, esto se realiza a lo sumo O(long(clave)). De todas maneras como estamos actualizando los minimos y maximos y esto tiene una complejidad de O(L), y O(long(clave)) $\in$ O(L), esto determina la complejidad.
    \end{algorithmic}
\end{algorithm}



\begin{algorithm}[H]{\textbf{iCrearItClaves}(\In{d}{estr}) $\to$ $res$ : itConj(string)}
	\begin{algorithmic}[1]
		\State res $\gets$ CrearIt(d.claves)									    \Comment O(1)
		\medskip
		\Statex \underline{Complejidad:} O(1)
    \end{algorithmic}
\end{algorithm}


\begin{algorithm}[H]{\textbf{iClaveMinima}(\In{d}{estr}) $\to$ $res$ : string}
	\begin{algorithmic}[1]
		\State res $\gets$ d.min								    \Comment O(1)
		\medskip
		\Statex \underline{Complejidad:} O(1)
		\Statex \underline{Justificación:} Se devuelve por referencia	
    \end{algorithmic}
\end{algorithm}



\begin{algorithm}[H]{\textbf{iClaveMaxima}(\In{d}{estr}) $\to$ $res$ : string}
	\begin{algorithmic}[1]
		\State res $\gets$ d.max								    \Comment O(1)
		\medskip
		\Statex \underline{Complejidad:} O(1)
		\Statex \underline{Justificación:} Se devuelve por referencia		
    \end{algorithmic}
\end{algorithm}






\begin{algorithm}[H]{\textbf{iTieneHijos}(\In{nodo}{puntero(Nodo)}) $\to$ $res$ : $bool$}
	{\\ $\textbf{Pre}$ $\equiv$ nodo!=NULL}
	\begin{algorithmic}[1]
		\State nat i $\gets$ 0 \Comment $O(1)$
		\While{i $<$ 256 $\yluego$ $nodo \rightarrow hijos$[i]=NULL )} \Comment O(256) $\in$ O(1)
			\State $i \gets i + 1$ \Comment $O(1)$
		\EndWhile
		\State res $\gets$ i $<$ 256     \Comment $O(1)$

		\medskip
		\Statex \underline{Complejidad:} $O(1)$
		\Statex \underline{Justificación:} Recorre el arreglo de 256 posiciones en caso de que todas las posiciones del mismo tengan NULL. Como es una constante ya que en el peor caso siempre recorre a lo sumo 256 posiciones entonces es O(1).

    \end{algorithmic}
    {$\textbf{Post}$ $\equiv$ res $=_{obs}$  ($\exists i$) i < 256 $\yluego$ $\lnot$($nodo \rightarrow hijos$[i]=NULL) }
\end{algorithm}




\begin{algorithm}[H]{\textbf{iActualizarMin}(\Inout{d}{estr})}
	{\\ $\textbf{Pre}$ $\equiv$ $\lnot$(d=NULL) $\land$ Rep(d) }
	\begin{algorithmic}[1]
		\State d.min $\gets$ Vacio()					\Comment O(1)
		\If{longitud(d.claves) > 0} 								\Comment O(1)
			\State nodo n $\gets$ d.raiz						\Comment O(1)
			\While{$n \rightarrow info$ = NULL}		\Comment O(max(long(claves(d)))
				\State nat i $\gets$ 0								\Comment O(1)
				\While{i $<$ 256 $\land$ $n \rightarrow hijos$[i] = NULL} 		\Comment O(256) $\in$ O(1)
					\State i $\gets$ i + 1				\Comment O(1)
				\EndWhile
				\State n $\gets$ $n \rightarrow hijos$[i]			\Comment O(1)
			\EndWhile
			\State d.min $\gets$  copy(Siguiente(n.clave)) \Comment O(copy(clave))  
		\EndIf
		\medskip
		\Statex \underline{Complejidad:} O(L) Done L es la clave mas larga definida en el diccionario
		\Statex \underline{Justificación:} El ciclo principal va recorriendo los nodos hasta que encuentra el primer significado, esto se ejecuta a lo sumo L veces. Luego las operaciones del ciclio son O(1), por lo tanto O(L)*O(1) $\in$ O(L). Luego se copia la clave, pero esto tambien es O(L), por lo cual O(L) + O(L) = O(L)
    \end{algorithmic}
    {$\textbf{Post}$ $\equiv$ \IF \#(d.claves)=0 THEN d.min = <> ELSE  d.min $\in$ d.claves $\yluego$  $(\forall s:\text{string})$ s $\in$ d.claves $\land$ s $\neq$ d.min $\implies$ d.min $<$ s  FI  }
\end{algorithm}


\begin{algorithm}[H]{\textbf{iActualizarMax}(\Inout{d}{estr})}
	{\\ $\textbf{Pre}$ $\equiv$ $\lnot$(d=NULL) $\land$ Rep(d) }
	\begin{algorithmic}[1]
		\State d.max $\gets$ Vacio()					\Comment O(1)
		\If{longitud(d.claves) > 0} 								\Comment O(1)
			\State nodo n $\gets$ d.raiz						\Comment O(1)
			\While{tieneHijos(n)}									\Comment O(max(long(claves(d))) * O(1)
				\State nat i $\gets$ 0								\Comment O(1)
				\While{i $<$ 256 $\land$ $n \rightarrow hijos$[256 - i - 1] = NULL} 		\Comment O(256) $\in$ O(1)
					\State i $\gets$ i + 1				\Comment O(1)
				\EndWhile
				\State n $\gets$ $n \rightarrow hijos$[i]			\Comment O(1)
			\EndWhile
			\State d.max $\gets$  copy(Siguiente(n.clave)) \Comment O(copy(clave))  
		\EndIf
		\medskip
		\Statex \underline{Complejidad:} O(L)
		\Statex \underline{Justificación:} El ciclo principal va recorriendo los nodos hasta que encuentra el primer significado, esto se ejecuta a lo sumo L veces. Luego las operaciones del ciclio son O(1), por lo tanto O(L)*O(1) $\in$ O(L). Luego se copia la clave, pero esto tambien es O(L), por lo cual O(L) + O(L) = O(L)
    \end{algorithmic}
    {$\textbf{Post}$ $\equiv$ \IF \#(d.claves)=0 THEN d.max = <> ELSE d.max $\in$ d.claves $\yluego$  $(\forall s:\text{string})$ s $\in$ d.claves $\land$ s $\neq$ d.max $\implies$ d.max $>$ s  FI  }
\end{algorithm}




\pagebreak
\section{Módulo Diccionario Natural($\alpha$)}

\subsection{Interfaz}

\textbf{se explica con}: \tadNombre{Diccionario(Nat, $\alpha$)}.

\textbf{géneros}: \TipoVariable{diccNat($\alpha$)}.


~


\subsubsection{Operaciones básicas de Diccionario Natural($\alpha$)}

\begin{Interfaz}

\InterfazFuncion{CrearDiccionario}{}{diccNat($\alpha$)}
[true]
{$res \igobs$ vacio()}
[O(1)]
[Crea un diccionario vacío.]
[]

~

\InterfazFuncion{Definir}{\In{c}{Nat}, \In{s}{$\alpha$}, \Inout{d}{diccNat($\alpha$)} }{}
[$d \igobs d_o$]
{$d \igobs$ Definir($c$, $s$, $d_o$)}
[Caso prom: $O(log(n) + copy(s))$ | Peor Caso $O(n + copy(s))$]
[Define la clave $c$ con el significado $s$. Si la clave esta definida deja el significado $s$ y borra el anterior.]
[El elemento $s$ se agrega por copia.]

~

\InterfazFuncion{Definido?}{\In{c}{Nat}, \In{d}{diccNat($\alpha$)} }{bool}
[true]
{$res \igobs$ Definido?($c$, $d$)}
[Caso prom: $O(log(n))$ | Peor Caso $O(n)$]
[Devuelve true si la clave está definida en el diccionario y false en caso contrario.]
[]

~

\InterfazFuncion{Obtener}{\In{c}{Nat}, \In{d}{diccNat($\alpha$)} }{$\alpha$}
[Definido?($c$, $d$)]
{$res \igobs$ Obtener($c$,$d$)}
[Caso prom: $O(log(n))$ | Peor Caso $O(n)$]
[Devuelve el significado correspondiente a la clave $c$.]
[Devuelve el significado almacenado en el diccionario, por lo que $res$ es modificable si y sólo si $d$ lo es.]

~

\InterfazFuncion{Borrar}{\In{c}{clave}, \Inout{d}{diccNat($\alpha$)} }{}
[$d \igobs d_o \land$ Definido?($c$, $d$)]
{$d \igobs$ Borrar($c$, $d_o$)}
[Caso prom: $O(log(n))$ | Peor Caso $O(n)$]
[Borra la clave $c$ y su significado del diccionario.]
[]

~

\InterfazFuncion{EsVacio?}{\In{d}{diccNat($\alpha$)}}{bool}
[true]
{$res \igobs$ EsVacio?($d$)}
[O(1)]
[Devuelve true si el diccionario esta vacío.]
[]

~

\InterfazFuncion{\#Claves}{\In{d}{diccNat($\alpha$)} }{nat}
[true]
{$res \igobs$ \#Claves($d$)}
[O(1)]
[Devuelve la cantidad de claves del diccionario.]
[]

~

\InterfazFuncion{ClaveMinima}{\In{d}{diccNat($\alpha$)} }{nat}
[$\neg EsVacio(d)$]
{$res \in$ Claves($d$) $\land$ EsClaveMinima($res$, Claves($d$))}
[O(1)]
[Devuelve la clave mas chica del diccionario.]
[]

~

\InterfazFuncion{ClaveMaxima}{\In{d}{diccNat($\alpha$)} }{nat}
[$\neg EsVacio(d)$]
{$res \in$ Claves($d$) $\land$ EsClaveMaxima($res$, Claves($d$))}
[O(1)]
[Devuelve la clave mas alta del diccionario.]
[]

\tadOperacion{EsClaveMaxima}{n,conj(nat)}{bool}{}
\tadOperacion{EsClaveMinima}{n,conj(nat)}{bool}{}

\tadAxioma{EsClaveMaxima(n,c)}{ \IF $\emptyset$(c) THEN true
	ELSE n $\geq$ DameUno(c) $\land$ EsClaveMaxima(n,SinUno(c))
	FI
}

\tadAxioma{EsClaveMinima(n,c)}{ \IF $\emptyset$(c) THEN true
	ELSE n $\leq$ DameUno(c) $\land$ EsClaveMinima(n,SinUno(c))
	FI
}


\end{Interfaz}




%===============================================================
\subsubsection{Representación de Diccionario Natural($\alpha$)}

\begin{Estructura}{diccNat($\alpha$)}[estr]
	\begin{Tupla}[estr]
		\tupItem{raiz}{puntero(Nodo)}%
		\tupItem{\\ masChico}{puntero(Nodo)}%
		\tupItem{\\ masGrande}{puntero(Nodo)}%
		\tupItem{\\ cantclaves}{nat}%
	\end{Tupla}

	\begin{Tupla}[Nodo]
		\tupItem{clave}{nat}%
		\tupItem{\\ sign}{$\alpha$}%
		\tupItem{\\ padre}{puntero(Nodo)}%
		\tupItem{\\ hizq}{puntero(Nodo)}%
		\tupItem{\\ hder}{puntero(Nodo)}%
	\end{Tupla}	

\end{Estructura}

~


\subsubsection{Invariante de Representación}


\begin{enumerate}
	%1
	\item Si e.raiz apunta a null entonces e.cantclaves es igual a cero y e.masChico y e.masGrande apuntan a null, o e.raiz no apunta a null entonces e.cantclaves es distinto de cero y e.masChico y e.masGrande no apuntan a null.
	%2
	\item e.cantclaves es la cantidad de nodos del arbol.
	%3
	\item Para cada nodo, si tiene hijos, hizq tiene una clave mas chica que el nodo y hder tiene una clave mas grande que el nodo.
	%4
	\item Si e.cantclaves es igual a uno, e.raiz, e.masChico y e.masGrande apuntan al mismo nodo, si es mayor que cero e.masChico es el nodo de mas a la izquierda de e.raiz y e.masGrande es el nodo de mas a la derecha de e.raiz.

\end{enumerate}


\Rep[estr][e]{
	\\\textbf{(1)}
	e.raiz $=$ NULL $\Leftrightarrow$ (e.cantclaves $= 0$ $\land$ e.masChico $=$ NULL $\land$ e.masGrande $=$ NULL)
	\\
	$\land$ 
	\\
	e.raiz $\neq$ NULL $\Leftrightarrow$ (e.cantclaves $\neq$ 0 $\land$ e.masChico $\neq$ NULL $\land$ e.masGrande $\neq$ NULL)
	\\
	$\land$
	\\\textbf{(2)}
	e.cantclaves $=$ tamaño(e.raiz)
	\\
	$\land$
	\\\textbf{(3)}
	invarianteABB(e.raiz)
	\\
	$\land$
	\\\textbf{(4)}
	e.cantclaves $= 1$ $\Leftrightarrow$ e.raiz = e.masChico $\land$ e.raiz = e.masGrande
	\\
	$\land$
	\\
	e.cantclaves $> 1$ $\Leftrightarrow$ e.masChico $=$ nodoMasIzq(e.raiz) $\land$ e.masGrande $=$ nodoMasDer(e.raiz)
}\mbox{}

~

\tadOperacion{tamaño}{puntero(Nodo)}{nat}{}
\tadOperacion{invarianteABB}{puntero(Nodo)}{bool}{}
\tadOperacion{tieneDosHijos}{puntero(Nodo)}{bool}{}
\tadOperacion{tieneHijoIzq}{puntero(Nodo)}{bool}{}
\tadOperacion{nodoMasIzq}{puntero(Nodo)}{puntero(Nodo)/p}{p $\neq$ NULL}
\tadOperacion{nodoMasDer}{puntero(Nodo)}{puntero(Nodo)/p}{p $\neq$ NULL}


~

\tadAxioma{tamaño(p)}{
	\IF p $==$ NULL THEN $0$
	ELSE $1 +$ tamaño(p$\rightarrow$hder) $+$ tamaño(p$\rightarrow$hder)
	FI
}

\tadAxioma{invarianteABB(p)}{
	\IF p $==$ NULL THEN $true$
	ELSE {
		\IF tieneDosHijos(p) THEN (p$\rightarrow$clave $>$ p$\rightarrow$hizq$\rightarrow$clave) $\land$ (p$\rightarrow$clave $<$ p$\rightarrow$hder$\rightarrow$clave) $\land$ invarianteABB(p$\rightarrow$hizq) $\land$ invarianteABB(p$\rightarrow$hder)
		ELSE {
			\IF tieneHijoIzq(p) THEN (p$\rightarrow$clave $>$ p$\rightarrow$hizq$\rightarrow$clave) $\land$ invarianteABB(p$\rightarrow$hizq)
			ELSE (p$\rightarrow$clave $<$ p$\rightarrow$hder$\rightarrow$clave) $\land$ invarianteABB(p$\rightarrow$hder)
			FI
		}
		FI
	}
	FI
}

\tadAxioma{tieneDosHijos(p)}{p$\rightarrow$hder $\neq$ NULL $\land$ p$\rightarrow$hizq $\neq$ NULL}

\tadAxioma{tieneHijoIzq(p)}{p$\rightarrow$hizq $\neq$ NULL}

\tadAxioma{nodoMasIzq(p)}{
	\IF p$\rightarrow$hizq $==$ NULL THEN p
	ELSE nodoMasIzq(p$\rightarrow$hizq)
	FI
}

\tadAxioma{nodoMasDer(p)}{
	\IF p$\rightarrow$hder $==$ NULL THEN p
	ELSE nodoMasDer(p$\rightarrow$hder)
	FI
}

~

\subsubsection{Función de Abstracción}


\Abs[estr]{diccNat($\alpha$)}[e]{d}{
($\forall n$: nat)(def?($n$, $d$) $\igobs$ EstaDef?($n$, $e.raiz$) $\yluego$ def?($n$, $d$) $\impluego$ \\ $\impluego$ (obtener($n$, $d$) $\igobs$ Sign($n$,$e.raiz$)))
}

~

\tadOperacion{EstaDef?}{nat/n,ab((nat,$\alpha$))/a}{bool}{}
\tadOperacion{Sign}{nat/n,ab((nat,$\alpha$))/a}{bool}{EstaDef?(n,a)}

\tadAxioma{EstaDef?(n,a)}{
	\IF (a = NULL) THEN false
	ELSE {
		\IF ($\pi_1$(raiz(a)) $=$ n) THEN true
		ELSE{
			\IF ($\pi_1$(raiz(a)) $>$ n) THEN EstaDef?(n, izq(a))
			ELSE EstaDef?(n, der(a))
			FI
		}
		FI
	}
	FI
}

\tadAxioma{Sign(n,a)}{
	\IF ($\pi_1$(raiz(a)) $=$ n) THEN $\pi_2$(raiz(a))
	ELSE{
		\IF ($\pi_1$(raiz(a)) $>$ n) THEN Sign(n, izq(a))
		ELSE Sign(n, der(a))
		FI
	}
	FI
	
}




%================================================================
\subsection{Algoritmos}

%\subsubsection{Algoritmos de Diccionario Natural($\alpha$)}

\begin{algorithm}[H]{\textbf{iCrearDiccionario}() $\to$ $res$ : $estr$}
	{\\ $\textbf{Pre}$ $\equiv$ true}
	\begin{algorithmic}[1]

		\State $res.raiz \gets NULL$ \Comment $O(1)$
		\State $res.masChico \gets NULL$ \Comment $O(1)$
		\State $res.masGrande \gets NULL$ \Comment $O(1)$
		\State $res.cantnodos \gets 0$ \Comment $O(1)$

		\medskip
		\Statex \underline{Complejidad:} $O(1)$
		\Statex \underline{Justificación:} Inicializa los punteros en NULL y le asigna $0$ a la cantidad de nodos.

    \end{algorithmic}
    {$\textbf{Post}$ $\equiv$ res $\igobs$ vacio()}
\end{algorithm}


\begin{algorithm}[H]{\textbf{iDefinir}(\In{c}{Nat}, \In{s}{$\alpha$}, \Inout{e}{estr})}
	{\\ $\textbf{Pre}$ $\equiv$ $d \igobs d_o$}
	\begin{algorithmic}[1]

		\If{$\neg$Definido(c,e)} \Comment $O(log(n))$ en caso promedio.
			\State $\backslash\backslash$ no esta definido, lo agrego
			\\
			\State $puntero(Nodo): nuevonodo$ \Comment $O(1)$

			\State $(nuevonodo$$\rightarrow$$clave) \gets c$ \Comment $O(1)$
			\State $(nuevonodo$$\rightarrow$$sign) \gets s$ \Comment $\Theta(copy(s))$
			\\

			\If{$e.raiz == NULL$} \Comment $O(1)$
				\State $\backslash\backslash$ el arbol está vacio
				
				\State $e.raiz \gets nuevonodo$ \Comment $O(1)$
				\State $e.masChico \gets nuevonodo$ \Comment $O(1)$ 
				\State $e.masGrande \gets nuevonodo$ \Comment $O(1)$
				\\

			\Else
				\State $\backslash\backslash$ el arbol NO está vacio, busco al padre y lo agrego en su lugar

				\State $puntero(Nodo) npapa \gets $BuscaNodoParaInsertar$(c)$ \Comment $O(log(n))$ en caso promedio.
				\\

				\If {$c < (npapa$$\rightarrow$$clave) \land (npapa$$\rightarrow$$hizq) == NULL$} \Comment $O(1)$

					\State $(npapa$$\rightarrow$$hizq) \gets nuevonodo$ \Comment $O(1)$
					\State $(nuevonodo$$\rightarrow$$padre) \gets npapa$ \Comment $O(1)$
					\\
				\ElsIf{$c > (npapa$$\rightarrow$$clave) \land (npapa$$\rightarrow$$hder) == NULL$} \Comment $O(1)$

					\State $(npapa$$\rightarrow$$hder) \gets nuevonodo$ \Comment $O(1)$
					\State $(nuevonodo$$\rightarrow$$padre) \gets npapa$ \Comment $O(1)$
					\\
				\EndIf
				\\

				\State $\backslash\backslash$ si la clave es mas chica que la clave de e.masChico o la clave es mas grande que e.masGrande cambio el puntero.
				\\

				\If{$c < (e.masChico$$\rightarrow$$clave)$} \Comment $O(1)$

					\State $e.masChico \gets nuevonodo$ \Comment $O(1)$

				\ElsIf{$c > (e.masGrande$$\rightarrow$$clave)$}
					\State $e.masGrande \gets nuevonodo$ \Comment $O(1)$					

				\EndIf
				\\

				\State $e.cantnodos \gets e.cantnodos + 1$ \Comment $O(1)$

			\EndIf


		\Else
			\State $\backslash\backslash$ esta definida la clave, lo piso.
			\\

			\State $puntero(Nodo): nodo \gets$ BuscaNodo$(c)$ \Comment $O(log(n))$ en caso promedio.

			\State $(nodo$$\rightarrow$$sign) \gets s$ \Comment $\Theta(copy(s))$


		\EndIf

		\medskip
		\Statex \underline{Complejidad:} Caso promedio: $O(log(n) + copy(s))$ | Peor Caso $O(n + copy(s))$
		\Statex \underline{Justificación:} Lo que hace el algoritmo es fijarse si esta definida la clave (Caso prom: $O(log(n))$ | Peor Caso $O(n)$), si no lo esta crea un nuevo nodo con la clave y el significado (Caso prom: $O(log(n))$ | Peor Caso $O(n)$) y lo inserta en el lugar correspondiente buscando el padre del nuevo nodo (Caso prom: $O(log(n))$ | Peor Caso $O(n)$). Si esta definida la clave solo busca el nodo de la misma (Caso prom: $O(log(n))$ | Peor Caso $O(n)$) y copia el nuevo significado ($O(copy(s))$).

    \end{algorithmic}
    {$\textbf{Post}$ $\equiv$ $d \igobs$ Definir($c$, $s$, $d_o$)}
\end{algorithm}


\begin{algorithm}[H]{\textbf{iDefinido?}(\In{c}{nat}, \In{e}{estr}) $\to$ $res$ : $bool$}
	{\\ $\textbf{Pre}$ $\equiv$ true}
	\begin{algorithmic}[1]

		\State $res \gets $BuscaNodo$(c,e) \neq Null$ \Comment $O(log(n))$ en caso promedio.

		\medskip
		\Statex \underline{Complejidad:} $O(log(n))$ en caso promedio.
		\Statex \underline{Justificación:} Es la complejidad de BuscaNodo basicamente. BuscaNodo devuelve el Nodo de la clave, si existe, o NULL si no existe.

    \end{algorithmic}
    {$\textbf{Post}$ $\equiv$ $res \igobs$ Definido?($c$, $d$)}
\end{algorithm}


\begin{algorithm}[H]{\textbf{iObtener}(\In{c}{nat}, \In{e}{estr}) $\to$ $res$ : $\alpha$}
	{\\ $\textbf{Pre}$ $\equiv$ Definido?($c$, $d$)}
	\begin{algorithmic}[1]

		\State $res \gets ($BuscaNodo$(c,e)$$\rightarrow$$sign)$ \Comment $O(log(n))$ en caso promedio.

		\medskip
		\Statex \underline{Complejidad:} $O(log(n))$ en caso promedio.
		\Statex \underline{Justificación:} Es la complejidad de BuscaNodo.

    \end{algorithmic}
    {$\textbf{Post}$ $\equiv$ $res \igobs$ Obtener($c$,$d$)}
\end{algorithm}


\begin{algorithm}[H]{\textbf{iBorrar}(\In{c}{nat}, \In{e}{estr}) $\to$ $res$ : $\alpha$}
	{\\ $\textbf{Pre}$ $\equiv$ $d \igobs d_o \land$ Definido?($c$, $d$)}
	\begin{algorithmic}[1]

		\If{Definido?$(c,e)$} \Comment $O(log(n))$ en caso promedio.
			\State $puntero(Nodo): paraborrar \gets$ BuscaNodo$(c,e)$ \Comment $O(log(n))$ en caso promedio.
			\\

			\State $\backslash\backslash$ reviso si borro la clave mas chica.
			\If{$(e.masChico$$\rightarrow$$clave) == c$} \Comment $O(1)$

				\State $\backslash\backslash$ reasigno el mas chico
				\\

				\If{TieneHijoDerecho$(e.masChico)$} \Comment $O(1)$
					\State $e.masChico \gets$ BuscaNodoMasIzq$((e.masChico$$\rightarrow$$hder))$ \Comment $O(log(n))$ en caso promedio.
					\\

				\ElsIf{$(e.masChico$$\rightarrow$$padre) \neq NULL$} \Comment $O(1)$
					\State $e.masChico \gets$ $(e.masChico$$\rightarrow$$padre)$ \Comment $O(1)$
					\\

				\Else
					\State $\backslash\backslash$ no tiene hijos ni padre, es unico nodo.
					\State $e.masChico \gets NULL$ \Comment $O(1)$

				\EndIf

			\EndIf
			\\

			\State $\backslash\backslash$ reviso si borro la clave mas alta.
			\If{$(e.masGrande$$\rightarrow$$clave) == c$} \Comment $O(1)$

				\State $\backslash\backslash$ reasigno el mas alto
				\\

				\If{TieneHijoIzq$(e.masGrande)$} \Comment $O(1)$
					\State $e.masGrande \gets$ BuscoNodoMasDerecha$((e.masGrande$$\rightarrow$$hizq))$ \Comment $O(log(n))$ en caso promedio.
					\\

				\ElsIf{$(e.masGrande$$\rightarrow$$padre) \neq NULL$} \Comment $O(1)$
					\State $e.masGrande \gets$ $(e.masGrande$$\rightarrow$$padre)$ \Comment $O(1)$
					\\

				\Else
					\State $\backslash\backslash$ no tiene hijos ni padre, es unico nodo.
					\State $e.masGrande \gets NULL$ \Comment $O(1)$

				\EndIf

			\EndIf
			\\

			\State $\backslash\backslash$ Borro separando en casos.

			\If{NoTieneHijos$(paraborrar)$} \Comment $O(1)$

				\State BorrarCasoNoTieneHijos$(paraborrar, e)$ \Comment $O(1)$

			\ElsIf{TieneUnHijo$(paraborrar)$} \Comment $O(1)$

				\State BorrarCasoTieneUnHijo$(paraborrar, e)$ \Comment $O(1)$

			\Else

				\State BorrarCasoTieneDosHijos$(paraborrar, e)$ \Comment $O(log(n))$ en caso promedio.

			\EndIf
			\\
			
			\State $e.cantclaves \gets e.cantclaves -1$ \Comment $O(1)$

			\State $paraborrar \gets NULL$ \Comment $O(1)$

		\EndIf

		\medskip
		\Statex \underline{Complejidad:} $O(log(n))$ en caso promedio.
		\Statex \underline{Justificación:} Primero revisa si la clave esta definida (Caso prom: $O(log(n))$ | Peor Caso $O(n)$), si lo esta, busca el nodo para borrar (Caso prom: $O(log(n))$ | Peor Caso $O(n)$). Si es el nodo mas chico, lo modifica (Caso prom: $O(log(n))$ | Peor Caso $O(n)$). Despues reacomoda los nodos, en donde el peor caso es si el nodo a borrar tiene dos hijos (Caso prom: $O(log(n))$ | Peor Caso $O(n)$). El resto de las operaciones son basicas con complejidad $O(1)$. El caso promedio se obtiene si las claves fueron insertadas y borradas de forma uniforme.

    \end{algorithmic}
    {$\textbf{Post}$ $\equiv$ $d \igobs$ Borrar($c$, $d_o$)}
\end{algorithm}


\begin{algorithm}[H]{\textbf{iEsVacio?}(\In{e}{estr}) $\to$ $res$ : $bool$}
	{\\ $\textbf{Pre}$ $\equiv$ true}
	\begin{algorithmic}[1]

		\State $res \gets e.raiz \neq NULL \land e.cantclaves \neq 0$ \Comment $O(1)$

		\medskip
		\Statex \underline{Complejidad:} $O(1)$
		\Statex \underline{Justificación:} Simplemente revisa si la raíz es null y si la cantidad de claves es cero.

    \end{algorithmic}
    {$\textbf{Post}$ $\equiv$ $res \igobs$ EsVacio?($d$)}
\end{algorithm}


\begin{algorithm}[H]{\textbf{i\#Claves}(\In{e}{estr}) $\to$ $res$ : $nat$}
	{\\ $\textbf{Pre}$ $\equiv$ true}
	\begin{algorithmic}[1]

		\State $res \gets e.cantclaves$ \Comment $O(1)$

		\medskip
		\Statex \underline{Complejidad:} $O(1)$
		\Statex \underline{Justificación:} Devuelve la cantidad de claves que es parte de la estructura.

    \end{algorithmic}
    {$\textbf{Post}$ $\equiv$ $res \igobs$ \#Claves($d$)}
\end{algorithm}


\begin{algorithm}[H]{\textbf{iClaveMinima}(\In{e}{estr}) $\to$ $res$ : $nat$}
	{\\ $\textbf{Pre}$ $\equiv$ $\neg EsVacio(d)$}
	\begin{algorithmic}[1]

		\State $res \gets (e.masChico$$\rightarrow$$clave)$ \Comment $O(1)$

		\medskip
		\Statex \underline{Complejidad:} $O(1)$

    \end{algorithmic}
    {$\textbf{Post}$ $\equiv$ $res \igobs$ Minimo($d$)}
\end{algorithm}


\begin{algorithm}[H]{\textbf{iClaveMaxima}(\In{e}{estr}) $\to$ $res$ : $nat$}
	{\\ $\textbf{Pre}$ $\equiv$ $\neg EsVacio(d)$}
	\begin{algorithmic}[1]

		\State $res \gets (e.masGrande$$\rightarrow$$clave)$ \Comment $O(1)$

		\medskip
		\Statex \underline{Complejidad:} $O(1)$

    \end{algorithmic}
    {$\textbf{Post}$ $\equiv$ $res \igobs$ Maximo($d$)}
\end{algorithm}


\begin{algorithm}[H]{\textbf{iBuscaNodoParaInsertar}(\In{c}{nat}, \In{e}{estr}) $\to$ $res$ : $puntero(Nodo)$}
	{\\ $\textbf{Pre}$ $\equiv$ $\neg$EsVacio?(e) $\land$ $\neg$Definido?(c,e)}
	\begin{algorithmic}[1]
		\State $\backslash\backslash$ No hay nodo que tenga como clave c, aquí estoy buscando al padre para poder insertar un nodo con la clave c. 
		\\

		\State $puntero(Nodo): buscanodo \gets e.raiz$ \Comment $O(1)$
		\State $puntero(Nodo): nodores \gets NULL$ \Comment $O(1)$
		\\

		\While{$buscanodo \neq NULL \land nodores == NULL$} \Comment $O(1)$
			\\
			\If{$c < (buscanodo$$\rightarrow$$clave) \land (buscanodo$$\rightarrow$$hizq) \neq NULL$} \Comment $O$
				\State $buscanodo \gets (buscanodo$$\rightarrow$$hizq)$ \Comment $O(1)$

			\ElsIf{$c > (buscanodo$$\rightarrow$$clave) \land (buscanodo$$\rightarrow$$hder) \neq NULL$} \Comment $O$
				\State $buscanodo \gets (buscanodo$$\rightarrow$$hder)$ \Comment $O(1)$

			\ElsIf{$(c < (buscanodo$$\rightarrow$$clave) \land (buscanodo$$\rightarrow$$hizq) == NULL) \lor (c > (buscanodo$$\rightarrow$$clave) \land (buscanodo$$\rightarrow$$hder) == NULL)$} \Comment $O(1)$
				\State $nodores \gets buscanodo$ \Comment $O(1)$
			\EndIf

		\EndWhile \Comment $O(log(n))$ en caso promedio.
		\\

		\State $res \gets nodores$ \Comment $O(1)$

		\medskip
		\Statex \underline{Complejidad:} Caso prom: $O(log(n))$ | Peor Caso $O(n)$
		\Statex \underline{Justificación:} Si cada clave Nat se inserta con distrubución uniforme, y no hay claves repetidas en este diccionario, genera que en caso promedio la busqueda de un elemento sea logaritmica ya que el arbol tiende a estar completo (por lo dicho antes y por su invariante) y en cada ciclo del while descarta aproximadamente la mitad de los elementos del subarbol.

    \end{algorithmic}
    {$\textbf{Post}$ $\equiv$ $res \igobs$ BuscoPadreAInsertar($c$, $e.raiz$)}
\end{algorithm}

\tadOperacion{BuscoPadreAInsertar}{nat/c, puntero(Nodo)/p}{puntero(Nodo)}{}

\tadAxioma{BuscoPadreAInsertar(c,p)}{
	\IF (clave(p) > c $\land$ hijoizq(p) = NULL) $\lor$ (clave(p) < c $\land$ hijoder(p) = NULL) THEN p
	ELSE {
		\IF clave(p) > c THEN BuscoPadreAInsertar(c,hijoizq(p))
		ELSE BuscoPadreAInsertar(c,hijoizq(p))
		FI
	}
	FI
}


~


\begin{algorithm}[H]{\textbf{iBuscaNodo}(\In{c}{nat}, \In{e}{estr}) $\to$ $res$ : $puntero(Nodo)$}
	{\\ $\textbf{Pre}$ $\equiv$ true}
	\begin{algorithmic}[1]

		\State $puntero(Nodo): buscanodo \gets e.raiz$ \Comment $O(1)$

		\While{$buscanodo \neq NULL \yluego (buscanodo$$\rightarrow$$clave) \neq c$} \Comment $O(1)$

			\If{$c < (buscanodo$$\rightarrow$$clave)$} \Comment $O(1)$
				\State $buscanodo \gets (buscanodo$$\rightarrow$$hizq)$ \Comment $O(1)$
			\Else
				\State $buscanodo \gets (buscanodo$$\rightarrow$$hder)$ \Comment $O(1)$
			\EndIf

		\EndWhile \Comment $O(log(n))$ en caso promedio.

		\\
		\State $res \gets buscanodo$ \Comment $O(1)$

		\medskip
		\Statex \underline{Complejidad:} $O(log(n))$ en caso promedio.
		\Statex \underline{Justificación:} Si cada clave Nat se inserta con distrubución uniforme, y no hay claves repetidas en este diccionario, genera que en caso promedio la busqueda de un elemento sea logaritmica ya que el arbol tiende a estar completo (por lo dicho antes y por su invariante) y en cada ciclo del while descarta aproximadamente la mitad de los elementos del subarbol.

    \end{algorithmic}
    {$\textbf{Post}$ $\equiv$ $res \igobs$ BuscoNodoDeClave($c$, $e.raiz$)}
\end{algorithm}

\tadOperacion{BuscoNodoDeClave}{nat/c, puntero(Nodo)/p}{puntero(Nodo)}{}

\tadAxioma{BuscoNodoDeClave(c,p)}{
	\IF p = NULL THEN NULL
	ELSE {
		\IF clave(p) = c THEN p
		ELSE {
			\IF clave(p) > c THEN BuscoNodoDeClave(c,hijoizq(p))
			ELSE BuscoNodoDeClave(c,hijoder(p))
			FI
		}
		FI
	}
	FI
}


~


\begin{algorithm}[H]{\textbf{iTieneHijoDerecho}(\In{n}{puntero(Nodo)}) $\to$ $res$ : $bool$}
	{\\ $\textbf{Pre}$ $\equiv$ $n \neq NULL$}
	\begin{algorithmic}[1]

		\State $res \gets (n$$\rightarrow$$hder) \neq NULL$ \Comment $O(1)$

		\medskip
		\Statex \underline{Complejidad:} $O(1)$

    \end{algorithmic}
    {$\textbf{Post}$ $\equiv$ $res \igobs$ hijoder(n) $\neq NULL$}
\end{algorithm}


\begin{algorithm}[H]{\textbf{iTieneHijoIzq}(\In{n}{puntero(Nodo)}) $\to$ $res$ : $bool$}
	{\\ $\textbf{Pre}$ $\equiv$ $n \neq NULL$}
	\begin{algorithmic}[1]

		\State $res \gets (n$$\rightarrow$$hizq) \neq NULL$ \Comment $O(1)$

		\medskip
		\Statex \underline{Complejidad:} $O(1)$

    \end{algorithmic}
    {$\textbf{Post}$ $\equiv$ $res \igobs$ hijoizq($n$) $\neq NULL$}
\end{algorithm}


\begin{algorithm}[H]{\textbf{iBuscaNodoMasIzq}(\In{der}{puntero(Nodo)}) $\to$ $res$ : $puntero(Nodo)$}
	{\\ $\textbf{Pre}$ $\equiv$ $der \neq$ NULL}
	\begin{algorithmic}[1]

		\State $puntero(Nodo): buscanodo \gets der$ \Comment $O(1)$

		\If{$buscanodo \neq NULL$} \Comment $O(1)$
		\\
			\While{$(buscanodo$$\rightarrow$$hizq) \neq NULL $} \Comment $O(1)$
				\State $buscanodo \gets (buscanodo$$\rightarrow$$hizq)$ \Comment $O(1)$
			\EndWhile \Comment $O(log(n))$ en caso promedio.

		\EndIf

		\State $res \gets buscanodo$ \Comment $O(1)$

		\medskip
		\Statex \underline{Complejidad:} Caso prom: $O(log(n))$ | Peor Caso $O(n)$
		\Statex \underline{Justificación:} Si cada clave Nat se inserta con distrubución uniforme y no hay claves repetidas en este diccionario, esto genera que el arbol tienda a estar completo, por lo tanto la altura del arbol tiende a ser $log(n)$. Como tiene que ir al nodo de más a la izquierda del arbol y la altura es en promedio $log(n)$, llegar a ese nodo, en promedio, es $log(n)$. Si las claves no fueron insertadas de forma uniforme se puede obtener un peor caso lineal $O(n)$.

    \end{algorithmic}
    {$\textbf{Post}$ $\equiv$ $res \igobs$ NodoMasIzq($der$)}
\end{algorithm}

\tadOperacion{NodoMasIzq}{puntero(Nodo)/p}{puntero(Nodo)}{$p \neq$ NULL}

\tadAxioma{NodoMasIzq(p)}{
	\IF hijoizq(p) = NULL THEN p
	ELSE NodoMasIzq(hijoizq(p))
	FI
}

~

\begin{algorithm}[H]{\textbf{iBuscoNodoMasDerecha}(\In{a}{puntero(Nodo)}) $\to$ $res$ : $puntero(Nodo)$}
	{\\ $\textbf{Pre}$ $\equiv$ $a \neq$ NULL}
	\begin{algorithmic}[1]

		\If{$a \neq NULL$} \Comment $O(1)$
		\\
			\While{$(a$$\rightarrow$$hder) \neq NULL $} \Comment $O(1)$
				\State $a \gets (a$$\rightarrow$$hder)$ \Comment $O(1)$
			\EndWhile \Comment $O(log(n))$ en caso promedio.

		\EndIf

		\State $res \gets a$ \Comment $O(1)$

		\medskip
		\Statex \underline{Complejidad:} Caso prom: $O(log(n))$ | Peor Caso $O(n)$
		\Statex \underline{Justificación:} Si cada clave Nat se inserta con distrubución uniforme y no hay claves repetidas en este diccionario, esto genera que el arbol tienda a estar completo, por lo tanto la altura del arbol tiende a ser $log(n)$. Como tiene que ir al nodo de más a la izquierda del arbol y la altura es en promedio $log(n)$, llegar a ese nodo, en promedio, es $log(n)$. Si las claves no fueron insertadas de forma uniforme se puede obtener un peor caso lineal $O(n)$.

    \end{algorithmic}
    {$\textbf{Post}$ $\equiv$ $res \igobs$ NodoMasDer($a$)}
\end{algorithm}

\tadOperacion{NodoMasDer}{puntero(Nodo)/p}{puntero(Nodo)}{$p \neq$ NULL}

\tadAxioma{NodoMasDer(p)}{
	\IF hijoder(p) = NULL THEN p
	ELSE NodoMasDer(hijoder(p))
	FI
}

~


\begin{algorithm}[H]{\textbf{iNoTieneHijos}(\In{pb}{puntero(Nodo)}) $\to$ $res$ : $puntero(Nodo)$}
	{\\ $\textbf{Pre}$ $\equiv$ $pb \neq$ NULL}
	\begin{algorithmic}[1]

		\State $res \gets (pb$$\rightarrow$$hder) == NULL \land (pb$$\rightarrow$$hizq) == NULL$ \Comment $O(1)$

		\medskip
		\Statex \underline{Complejidad:} $O(1)$

    \end{algorithmic}
    {$\textbf{Post}$ $\equiv$ $res \igobs$ hijoder$(pb) = NULL$ $\land$ hijoizq$(pb) = NULL$}
\end{algorithm}


\begin{algorithm}[H]{\textbf{iTieneUnHijo}(\In{pb}{puntero(Nodo)}) $\to$ $res$ : $puntero(Nodo)$}
	{\\ $\textbf{Pre}$ $\equiv$ $pb \neq$ NULL}
	\begin{algorithmic}[1]

		\State $res \gets ((pb$$\rightarrow$$hder) \neq NULL \land (pb$$\rightarrow$$hizq) = NULL) \lor ((pb$$\rightarrow$$hizq) \neq NULL \land (pb$$\rightarrow$$hder) = NULL)$ \Comment $O(1)$

		\medskip
		\Statex \underline{Complejidad:} $O(1)$

    \end{algorithmic}
    {$\textbf{Post}$ $\equiv$ $res \igobs$ (hijoder$(pb) \neq NULL$ $\land$ hijoizq$(pb) = NULL) \lor ($hijoder$(pb) = NULL$ $\land$ hijoder$(pb) \neq NULL)$}
\end{algorithm}


\begin{algorithm}[H]{\textbf{iBorrarCasoNoTieneHijos}(\Inout{pb}{puntero(Nodo)}, \Inout{e}{estr})}
	{\\ $\textbf{Pre}$ $\equiv$ $pb \neq$ NULL $\land$ Definido?($pb$$\rightarrow$$clave$,$e$) $\yluego$ Obtener($pb$$\rightarrow$$clave$,$e$) $=$ $pb$$\rightarrow$$sign$}
	\begin{algorithmic}[1]

		\If {$(pb$$\rightarrow$$padre) == NULL$} \Comment $O(1)$
			\State $e.raiz \gets NULL$ \Comment $O(1)$
		\Else
			\If {EsHijoDerecho$((pb$$\rightarrow$$padre), (pb$$\rightarrow$$clave)$)} \Comment $O(1)$
				\State $(pb$$\rightarrow$$padre)$$\rightarrow$$hder \gets NULL$ \Comment $O(1)$
			\Else
				\State $(pb$$\rightarrow$$padre)$$\rightarrow$$hizq \gets NULL$ \Comment $O(1)$
			\EndIf

		\EndIf

		\medskip
		\Statex \underline{Complejidad:} $O(1)$
		\Statex \underline{Justificación:} Todas las operaciones son sobre los punteros del puntero(Nodo) pb. Tanto las de esta función como las de las funciones llamadas dentro de ella.

    \end{algorithmic}
    {$\textbf{Post}$ $\equiv$ BuscoNodoDeClave(clave($pb$),$e$) = NULL}
\end{algorithm}


\begin{algorithm}[H]{\textbf{iBorrarCasoTieneUnHijo}(\Inout{pb}{puntero(Nodo)}, \Inout{e}{estr})}
	{\\ $\textbf{Pre}$ $\equiv$ $pb \neq$ NULL $\land$ Definido?($pb$$\rightarrow$$clave$,$e$) $\yluego$ Obtener($pb$$\rightarrow$$clave$,$e$) $=$ $pb$$\rightarrow$$sign$}
	\begin{algorithmic}[1]

		\If {$(pb$$\rightarrow$$padre) == NULL$} \Comment $O(1)$
			\State $\backslash\backslash$ no tiene padre

			\If{TieneHijoDerecho$(pb)$} \Comment $O(1)$
				\State $e.raiz \gets (pb$$\rightarrow$$hder)$ \Comment $O(1)$
				\State $(pb$$\rightarrow$$hder)$$\rightarrow$$padre \gets NULL$ \Comment $O(1)$

			\Else
				\State $e.raiz \gets (pb$$\rightarrow$$hizq)$ \Comment $O(1)$
				\State $(pb$$\rightarrow$$hizq)$$\rightarrow$$padre \gets NULL$ \Comment $O(1)$

			\EndIf

		\Else
			\State $\backslash\backslash$ tiene padre

			\If {TieneHijoDerecho$(pb)$} \Comment $O(1)$
				
				\If{EsHijoDerecho$((pb$$\rightarrow$$padre), (pb$$\rightarrow$$clave))$} \Comment $O(1)$
					\State $(pb$$\rightarrow$$padre)$$\rightarrow$$hder \gets (pb$$\rightarrow$$hder)$ \Comment $O(1)$
				\Else
					\State $(pb$$\rightarrow$$padre)$$\rightarrow$$hizq \gets (pb$$\rightarrow$$hder)$ \Comment $O(1)$
				\EndIf
				\State $(pb$$\rightarrow$$hder)$$\rightarrow$$padre \gets (pb$$\rightarrow$$padre)$ \Comment $O(1)$
				\\
			\Else

				\If{EsHijoDerecho$((pb$$\rightarrow$$padre), (pb$$\rightarrow$$clave))$} \Comment $O(1)$
					\State $(pb$$\rightarrow$$padre)$$\rightarrow$$hder \gets (pb$$\rightarrow$$hizq)$ \Comment $O(1)$
				\Else
					\State $(pb$$\rightarrow$$padre)$$\rightarrow$$hizq \gets (pb$$\rightarrow$$hizq)$ \Comment $O(1)$
				\EndIf
				\State $(pb$$\rightarrow$$hizq)$$\rightarrow$$padre \gets (pb$$\rightarrow$$padre)$ \Comment $O(1)$
				\\

			\EndIf
			
		\EndIf

		\medskip
		\Statex \underline{Complejidad:} $O(1)$
		\Statex \underline{Justificación:} Todas las operaciones son sobre los punteros del puntero(Nodo) pb. Tanto las de esta función como las de las funciones llamadas dentro de ella.

    \end{algorithmic}
    {$\textbf{Post}$ $\equiv$ BuscoNodoDeClave(clave($pb$),$e$) = NULL}
\end{algorithm}


\begin{algorithm}[H]{\textbf{iBorrarCasoTieneDosHijos}(\Inout{pb}{puntero(Nodo)}, \Inout{e}{estr})}
	{\\ $\textbf{Pre}$ $\equiv$ $pb \neq$ NULL $\land$ Definido?($pb$$\rightarrow$$clave$,$e$) $\yluego$ Obtener($pb$$\rightarrow$$clave$,$e$) $=$ $pb$$\rightarrow$$sign$}
	\begin{algorithmic}[1]

		\State $puntero(Nodo): masizq \gets $BuscaNodoMasIzq$(pb$$\rightarrow$$hder)$ \Comment $O(log(n))$ en caso promedio.
		\\

		\State $\backslash\backslash$ acomodo hijos
		\State $\backslash\backslash$ si el nodo mas izq es hijo de pb
		\If{$(pb$$\rightarrow$$hder)$$\rightarrow$$clave == (masizq$$\rightarrow$$clave)$} \Comment $O(1)$
			\State $(masizq$$\rightarrow$$hizq) \gets (pb$$\rightarrow$$hizq)$ \Comment $O(1)$
			\State $(pb$$\rightarrow$$hizq)$$\rightarrow$$padre \gets masizq$ \Comment $O(1)$

		\Else

			\State AcomodoNodoMasIzq$(pb, masizq)$ \Comment $O(1)$

		\EndIf
		\\

		\State $\backslash\backslash$ acomodo el padre de pb y mas izq
		\If {$(pb$$\rightarrow$$padre) == NULL$} \Comment $O(1)$
			\State $(masizq$$\rightarrow$$padre) \gets NULL$ \Comment $O(1)$
			\State $e.raiz \gets masizq$ \Comment $O(1)$
		
		\Else
			\State $(masizq$$\rightarrow$$padre) (pb$$\rightarrow$$padre)$ \Comment $O(1)$
			\If{EsHijoDerecho$((pb$$\rightarrow$$padre), (pb$$\rightarrow$$clave))$} \Comment $O(1)$
				\State $(pb$$\rightarrow$$padre)$$\rightarrow$$hder \gets masizq$ \Comment $O(1)$

			\Else
				\State $(pb$$\rightarrow$$padre)$$\rightarrow$$hizq \gets masizq$ \Comment $O(1)$

			\EndIf

		\EndIf


		\medskip
		\Statex \underline{Complejidad:} Caso prom: $O(log(n))$ | Peor Caso $O(n)$
		\Statex \underline{Justificación:} Como el nodo a borrar posée dos hijos, se busca el nodo mas a la izquierda del subarbol derecho (el elemento con clave mas chica del subarbol) y se lo reemplaza por el que va a ser borrado. La mayor parte de las operaciones de este algoritmo tienen complejidad $O(1)$ salvo la función BuscaNodoMasIzq que tiene como complejidad Caso prom: $O(log(n))$ | Peor Caso $O(n)$.

    \end{algorithmic}
    {$\textbf{Post}$ $\equiv$ BuscoNodoDeClave(clave($pb$),$e$) = NULL}
\end{algorithm}


\begin{algorithm}[H]{\textbf{iEsHijoDerecho}(\In{pbpadre}{puntero(Nodo)}, \In{pbclave}{nat}) $\to$ $res$ : $bool$}
	{\\ $\textbf{Pre}$ $\equiv$ $pbpadre \neq$ NULL}
	\begin{algorithmic}[1]

		\State $res \gets (pbpadre$$\rightarrow$$hder) \neq NULL \land (pbpadre$$\rightarrow$$hder)$$\rightarrow$$clave == pbclave$ \Comment $O(1)$

		\medskip
		\Statex \underline{Complejidad:} $O(1)$

    \end{algorithmic}
    {$\textbf{Post}$ $\equiv$ $res \igobs$ padre$(pbpadre) \neq NULL) \yluego pbclave =$ clave(hijoder$(pbpadre))$}
\end{algorithm}


\begin{algorithm}[H]{\textbf{iEsHijoIzquierdo}(\In{pa}{puntero(Nodo)}, \In{pbclave}{nat}) $\to$ $res$ : $bool$}
	{\\ $\textbf{Pre}$ $\equiv$ $pa \neq$ NULL}
	\begin{algorithmic}[1]

		\State $res \gets (pa$$\rightarrow$$padre) \neq NULL \land (pa$$\rightarrow$$padre)$$\rightarrow$$hizq \neq NULL \land ((pa$$\rightarrow$$padre)$$\rightarrow$$hizq)$$\rightarrow$$clave == pa$$\rightarrow$$clave$ \Comment $O(1)$

		\medskip
		\Statex \underline{Complejidad:} $O(1)$

    \end{algorithmic}
    {$\textbf{Post}$ $\equiv$ $res \igobs$ padre$(pbpadre) \neq NULL) \yluego pbclave =$ clave(hijoder$(pbpadre))$}
\end{algorithm}


\begin{algorithm}[H]{\textbf{iAcomodoNodoMasIzq}(\Inout{pb}{puntero(Nodo)}, \Inout{masizq}{puntero(Nodo)})}
	{\\ $\textbf{Pre}$ $\equiv$ $pb \neq$ NULL $\land$ $masizq \neq$ NULL}
	\begin{algorithmic}[1]

		\State $\backslash\backslash$ el hijo que puede tener es el derecho
		\If {TieneUnHijo$(masizq)$} \Comment $O(1)$

			\State $(masizq$$\rightarrow$$hder)$$\rightarrow$$padre \gets (masizq$$\rightarrow$$padre) $ \Comment $O(1)$
			\State $(masizq$$\rightarrow$$padre)$$\rightarrow$$hizq \gets (masizq$$\rightarrow$$hder) $ \Comment $O(1)$

		\Else
			\State $\backslash\backslash$ no tiene hijo
			\State $(masizq$$\rightarrow$$padre)$$\rightarrow$$hizq \gets NULL$ \Comment $O(1)$
			\State $(masizq$$\rightarrow$$padre) \gets NULL$ \Comment $O(1)$

		\EndIf
		\\

		\State $(masizq$$\rightarrow$$hder) \gets (pb$$\rightarrow$$hder)$ \Comment $O(1)$
		\State $(pb$$\rightarrow$$hder)$$\rightarrow$$padre \gets masizq$ \Comment $O(1)$
		\State $(masizq$$\rightarrow$$hizq) \gets (pb$$\rightarrow$$hizq$ \Comment $O(1)$
		\State $(pb$$\rightarrow$$hizq)$$\rightarrow$$padre \gets masizq$ \Comment $O(1)$

		\medskip
		\Statex \underline{Complejidad:} $O(1)$
		\Statex \underline{Justificación:} Todas las operaciones son sobre los punteros del puntero(Nodo) pb. Tanto las de esta función como las de las funciones llamadas dentro de ella.
uu
    \end{algorithmic}
    {$\textbf{Post}$ $\equiv$ ((tieneunhijo($masizq$) $\Rightarrow$ (hijoizq(padre($masizq$)) = hijoder($masizq$) $\land$ padre(hijoder(masizq)) = padre(masizq))) $\lor$ ($\neg$ tieneunhijo($masizq$) $\Rightarrow$ hijoizq(padre($masizq$)) = NULL $\land$ padre($masizq$) = NULL)) $\land$ hijoder($masizq$) = hijoder($pb$) $\land$ padre(hijoder($pb$)) = $masizq$ $\land$ hijoizq($masizq$) = hijoizq($pb$) $\land$ padre(hijoizq($pb$)) = $masizq$}
\end{algorithm}


\tadOperacion{padre}{puntero(Nodo)}{puntero(Nodo)}{$p \neq$ NULL}
\tadOperacion{hijoder}{puntero(Nodo)}{puntero(Nodo)}{$p \neq$ NULL}
\tadOperacion{hijoizq}{puntero(Nodo)}{puntero(Nodo)}{$p \neq$ NULL}
\tadOperacion{clave}{puntero(Nodo)/p}{nat}{$p \neq$ NULL}
\tadOperacion{tieneunhijo}{puntero(Nodo)/p}{bool}{$p \neq$ NULL}

\tadAxioma{padre(p)}{ p$\rightarrow$padre}

\tadAxioma{hijoder(p)}{ p$\rightarrow$hder}

\tadAxioma{hijoizq(p)}{ p$\rightarrow$hizq}

\tadAxioma{clave(p)}{ p$\rightarrow$clave}

\tadAxioma{tieneunhijo(p)}{ (p$\rightarrow$hder = NULL $\land$ p$\rightarrow$hizq $\neq$ NULL) $\lor$ (p$\rightarrow$hder $\neq$ NULL $\land$ p$\rightarrow$hizq = NULL)}

\pagebreak

\section{Dato}

El módulo Dato provee una forma de encapsular los distintos tipos de datos soportados en los registros. 
El costo temporal de crear un dato de tipo string esta dado por la complejidad de copiar su contenido y dado que es un vector de chars, esta dado por la longitud del string muliplicado por el tiempo de copia de un char.

\subsection{Interfaz}
 
  \textbf{parámetros formales}\hangindent=2\parindent\\
  \parbox{1.7cm}{\textbf{géneros}} string\\
  \parbox[t]{1.7cm}{\textbf{función}}\parbox[t]{\textwidth-2\parindent-1.7cm}{%
    \InterfazFuncion{Copiar}{\In{s}{string}}{string}
    {$res \igobs s$}
    [$\Theta(copiar(s))$]
    [función de copia de strings]
  }



\textbf{se explica con} \tadNombre{Dato} 

\textbf{géneros}: dato


\subsubsection{Operaciones básicas de Dato}


\InterfazFuncion{datoString}{\In{s}{string}}{dato}
[true]
{$res \igobs$ datoString(s)}
[$\Theta(copiar(s))$]
[Crea un dato con tipo string]
[]

~

\InterfazFuncion{datoNat}{\In{n}{nat}}{dato}
[true]
{$res \igobs$ datoNat(s)}
[$\Theta(1)$]
[Crea un dato con tipo nat]
[]

~

\InterfazFuncion{Nat?}{\In{d}{dato}}{bool}
[true]
{$res \igobs$ Nat?(d)}
[$\Theta(1)$]
[Devuelve true solo si el tipo es nat]
[]

~

\InterfazFuncion{valorNat}{\In{d}{dato}}{nat}
[Nat?(d) $\igobs$ true ]
{$res \igobs$ valorNat(d)}
[$\Theta(1)$]
[Devuelve el valor nat del dato]
[]

~

\InterfazFuncion{valorStr}{\In{d}{dato}}{string}
[Nat?(d) $\igobs$ false ]
{alias($res \igobs$ valorStr(d))}
[$\Theta(1)$]
[Devuelve una referencia al valor string del dato]
[res es un alias y no es modificable]

~

\InterfazFuncion{mismoTipo?}{\In{d1}{dato},\In{d2}{dato}}{bool}
[true]
{$res \igobs$ mismoTipo?(d1,d2) }
[$\Theta(1)$]
[Devuelve true si el tipo de dato de d1 y d2 coinciden]
[]

~

\InterfazFuncion{String?}{\In{d}{dato}}{bool}
[true]
{$res \igobs$ $\lnot$ Nat?(d)}
[$\Theta(1)$]
[Devuelve true solo si el tipo es string]
[]

~

% \InterfazFuncion{min}{\In{ds}{conj(dato)}}{bool}
% [$\lnot$ vacio?(ds) $\land$  $(\forall d1,d2:\text{dato}) (d1 \in ds \land d2 \in ds) \impluego mismoTipo?(d1,d2)$]
% {alias($res \igobs$ min(ds)) }
% [$\Theta\left(\displaystyle\sum_{a' \in c} a \leq a'  \right)$]
% [Devuelve una referencia al minimo elemento del conjunto. ]
% []

% ~


% \InterfazFuncion{max}{\In{ds}{conj(dato)}}{bool}
% [$\lnot$ vacio?(ds) $\land$  $(\forall d1,d2:\text{dato}) (d1 \in ds \land d2 \in ds) \impluego mismoTipo?(d1,d2)$ ]
% {alias($res \igobs$ max(ds)) }
% [$\Theta\left(\displaystyle\sum_{a' \in c} a \leq a'  \right)$]
% [Devuelve una referencia al minimo elemento del conjunto. ]
% []

% ~

\InterfazFuncion{$\bullet = \bullet$}{\In{d1}{dato},\In{d2}{dato}}{bool}
[true]
{$res \igobs d1 = d2$}
[$\Theta(1)$ si Nat?(d1) $\land$ Nat?(d2), si no $\Theta(\min\{\text{long}(valorStr(d1)), \text{long}(valorStr(d2))\}$. ]
[Devuelve true si d1 es igual a d2 ]
[]

~ 

\subsubsection{Representación de Dato}

\begin{Estructura}{Dato}[estr]
	\begin{Tupla}[estr]
		\tupItem{esNat?}{bool}
		\tupItem{\\ valorNat}{nat}
		\tupItem{\\ valorStr}{string}
	\end{Tupla}
\end{Estructura}



\subsubsection{Invariante de Representación}

\begin{enumerate}
	%1
	\item Si e.esNat? es verdadero entonces e.valorStr es vacio.
	%2
	\item Si e.esNat? es falso entonces e.valorNat es cero.

\end{enumerate}

\Rep[estr][e]{
	\\\textbf{(1)}
	(e.esNat? $\Rightarrow$ e.valorStr $= <>$)
	\\
	$\lor$
	\\\textbf{(2)}
	($\neg$e.esNat? $\Rightarrow$ e.valorNat $= 0$)
}\mbox{}


\subsubsection{Función de Abastracción}

\Abs[estr]{dato}[e]{d}{
	e.esNat? = tipo?(d) $\yluego$ \\
	\IF tipo?(d) THEN e.valorNat = valorNat(d) ELSE e.valorStr = valorStr(d) FI
}


\subsection{Algoritmos}

  
\begin{algorithm}[H]{\textbf{iDatoString}(\In{s}{string}) $\to$ $res$ : estr}
    	\begin{algorithmic}[1]
			 \State $res \gets \langle false, 0, s \rangle$ \Comment $\Theta(copiar(s))$
			\medskip
			\Statex \underline{Complejidad:} $\Theta(copiar(s))$
    	\end{algorithmic}
\end{algorithm}


  
\begin{algorithm}[H]{\textbf{iDatoNat}(\In{n}{nat}) $\to$ $res$ : estr}
    	\begin{algorithmic}[1]
			 \State $res \gets \langle true, n, Vacia() \rangle$ \Comment $\Theta(1)$
			\medskip
			\Statex \underline{Complejidad:} $\Theta(1)$
    	\end{algorithmic}
\end{algorithm}


\begin{algorithm}[H]{\textbf{iNat?}(\In{d}{estr}) $\to$ $res$ : bool}
    	\begin{algorithmic}[1]
			 \State $res \gets d.esNat? $ \Comment $\Theta(1)$
			\medskip
			\Statex \underline{Complejidad:} $\Theta(1)$
    	\end{algorithmic}
\end{algorithm}

  
\begin{algorithm}[H]{\textbf{iValorNat}(\In{d}{estr}) $\to$ $res$ : nat}
    	\begin{algorithmic}[1]
			 \State $res \gets d.valorNat $ \Comment $\Theta(1)$
			\medskip
			\Statex \underline{Complejidad:} $\Theta(1)$
    	\end{algorithmic}
\end{algorithm}


\begin{algorithm}[H]{\textbf{iValorStr}(\In{d}{estr}) $\to$ $res$ : string}
    	\begin{algorithmic}[1]
			 \State $res \gets d.valorStr $ \Comment $\Theta(1)$
			\medskip
			\Statex \underline{Complejidad:} $\Theta(1)$
			\Statex \underline{Justificación:} Se devuelve una referencia al contenido del valor sin realizar copia.
    	\end{algorithmic}
\end{algorithm}


\begin{algorithm}[H]{\textbf{iString?}(\In{d}{estr}) $\to$ $res$ : bool}
    	\begin{algorithmic}[1]
			 \State $res \gets \lnot d.esNat? $ \Comment $\Theta(1)$
			\medskip
			\Statex \underline{Complejidad:} $\Theta(1)$
    	\end{algorithmic}
\end{algorithm}


\begin{algorithm}[H]{\textbf{iMismoTipo?}(\In{d1}{estr},\In{d1}{estr}) $\to$ $res$ : bool}
    	\begin{algorithmic}[1]
			 \State $res \gets d1.esNat? = d2.esNat?  $ \Comment $\Theta(1)$
			\medskip
			\Statex \underline{Complejidad:} $\Theta(1)$
    	\end{algorithmic}
\end{algorithm}


% \begin{algorithm}[H]{\textbf{iMin}(\In{ds}{conj(estr)}) $\to$ $res$ : estr}
%     	\begin{algorithmic}[1]
% 			 \State itConj(estr) iter $\gets$ crearIt(ds)    \Comment $\Theta(1)$
% 			 \State estr min $\gets$ siguiente(iter)          \Comment $\Theta(1)$

% 			 \While{haySiguiente(iter)}              		\Comment $\Theta(i)$
% 			 	\State $Avanzar(iter)$	                      	\Comment $\Theta(1)$
%    			    \State estr actual $\gets$ siguiente(iter)     \Comment $\Theta(1)$
% 					 \If{actual $\leq$ min}	 							\Comment $\Theta(*)$
% 			   			\State min $\gets$ actual                    \Comment $\Theta(1)$
% 		    		\EndIf
% 			 \EndWhile
% 			 \State res $\gets$ min          \Comment $\Theta(1)$
		

% 			\medskip
% 			\Statex \underline{Complejidad:} $\Theta(*) suma de comparaciones$
%     	\end{algorithmic}
% \end{algorithm}


% \begin{algorithm}[H]{\textbf{iMax}(\In{ds}{conj(estr)}) $\to$ $res$ : estr}
%     	\begin{algorithmic}[1]
% 			 \State itConj(estr) iter $\gets$ crearIt(ds)    \Comment $\Theta(1)$
% 			 \State estr max $\gets$ siguiente(iter)          \Comment $\Theta(1)$

% 			 \While{haySiguiente(iter)}              		\Comment $\Theta(i)$
% 			 	\State $Avanzar(iter)$	                      	\Comment $\Theta(1)$
%    			    \State estr actual $\gets$ siguiente(iter)     \Comment $\Theta(1)$
% 					 \If{max $\leq$ actual}	 							\Comment $\Theta(*)$
% 			   			\State max $\gets$ actual                    \Comment $\Theta(1)$
% 		    		\EndIf
% 			 \EndWhile
% 			 \State res $\gets$ max          \Comment $\Theta(1)$
		

% 			\medskip
% 			\Statex \underline{Complejidad:} $\Theta(*) suma de comparaciones$
%     	\end{algorithmic}
% \end{algorithm}


\begin{algorithm}[H]{\textbf{$\bullet = \bullet$}(\In{d1}{estr}, \In{d2}{estr}) $\to$ $res$ : bool}
    	\begin{algorithmic}[1]

			 \If {Nat?($d1$) $=$ Nat?($d2$)} \Comment $\Theta(1)$
			 	\State $res \gets d1.valorNat = d2.valorNat$ \Comment $\Theta(1)$

			\ElsIf {String?($d1$) $=$ String?($d2$)} \Comment $\Theta(1)$
				\State $res \gets d1.valorStr = d2.valorStr$ \Comment $\Theta(\min\{\text{long}(valorStr(d1)), \text{long}(valorStr(d2))\}$

			\Else
				\State $res \gets false$ \Comment $\Theta(1)$

			\EndIf

			\medskip
			\Statex \underline{Complejidad:} $\Theta(1)$ si Nat?(d1), si no $\Theta(\min\{\text{long}(valorStr(d1)), \text{long}(valorStr(d2))\}$
    	\end{algorithmic}
\end{algorithm}

\pagebreak
\section{Módulo Registro}

\subsection{Interfaz}

\textbf{MÓDULO Registro extiende a} \tadNombre{Módulo Diccionario Lineal(Campo, Dato)}.

\textbf{género}: registro


~


\subsubsection{Operaciones básicas de Registro}

\begin{Interfaz}

\InterfazFuncion{Campos}{\In{r}{registro}}{Conj(Campo)}
[true]
{$res \igobs$ Campos($r$)}
[$\Theta(1)$]
[Devuelve el conjunto de campos del registro.]
[Devuelve por copia. Como sabemos que por el contexto de uso que la cantidad de campos es acotada la complejidad es $\Theta(1)$.]

% ~

% \InterfazFuncion{Borrar?}{\In{criterio}{registo}, \In{reg}{registro} }{bool}
% [\#Campos($criterio$) $\equiv 1$]
% {$res \igobs$ Borrar?($criterio$, $reg$)}
% [$\Theta(L)$]
% [Devuelve true si coinciden todos los campos de los registros $criterio$ y $reg$.]
% []

~

\InterfazFuncion{CoincidenTodos}{\In{r1}{registro}, \In{cc}{conj(campo)}, \In{r2}{registro} }{bool}
[$cc \subseteq$ campos($r1$) $\cap$ campos($r2$)]
{$res \igobs$ CoincidenTodos($r1$, $cc$, $r2$)}
[$\Theta(L)$]
[Devuelve true si tienen todos los datos iguales ambos registros.]
[]

% ~

% \InterfazFuncion{CoincideAlguno}{\In{r1}{registro}, \In{cc}{conj(campo)}, \In{r2}{registro} }{bool}
% [$cc \subseteq$ Campos($r1$) $\cap$ Campos($r2$)]
% {$res \igobs$ CoincideAlguno($r1$, $cc$, $r2$)}
% [$\Theta(L)$]
% [Devuelve true si tienen al menos un dato igual ambos registros.]
% []

~

\InterfazFuncion{AgregarCampos}{\Inout{r1}{registro}, \In{r2}{registro}}{}
[$r1 \igobs r1_o$]
{$r1 \igobs$ AgregarCampos($r1_o$, $r2$)}
[$\Theta(L)$]
[Agrega los campos del registro $r2$ a $r1$ que $r1$ no posee.]
[]


\end{Interfaz}

~

\pagebreak

%=============================================

%EXTIENDE AL DICCIONARIO, NO SE NECESITA UNA REPRESENTACION NI REP/ABS, SON LOS DEL DICCIONARIO.


% \subsubsection{Representación de Registro}

% \begin{Estructura}{Registro}[estr]

% 	\begin{Tupla}[estr]
% 		\tupItem{registro}{diccLineal(campo, dato)}
% 	\end{Tupla}

% \end{Estructura}



% \subsubsection{Invariante de Representación}


% \Rep[estr][e]{

% }\mbox{}


% \subsubsection{Función de Abastracción}

% \Abs[estr]{registro}[e]{r}{
% 	($\forall c$: Nat) def?($c$,$r$) $\igobs$ Definido?($c$,$e$.registro) $\yluego$ (def?($c$,$r$) $\impluego$ obtener($c$,$r$) $\igobs$ Significado($c$, $r$.registro))
% }


\subsection{Algoritmos}


\begin{algorithm}[H]{{\textbf{iCampos}(\In{r}{dic})} $\to$ $res$ : $Conj(Campo)$}
	{\\ $\textbf{Pre}$ $\equiv$ True}
    	\begin{algorithmic}[1]

    		\State $itDicc(Campo, Dato): it \gets$ CrearIt$(r)$ \Comment $\Theta(1)$
			\State $Conj(Campo): res \gets$ Vacio$()$ \Comment $\Theta(1)$

			\While{HaySiguiente$(it)$} \Comment $\Theta(1)$

				\State AgregarRapido$(res,$ SiguienteClave$(it) )$ \Comment $\Theta(1)$

				\State Avanzar$(it)$ \Comment $\Theta(1)$
			
			\EndWhile \Comment $\Theta(1)$ Porque la cantidad de campos es acotada.

			\State $res \gets conj$ \Comment $\Theta(1)$ Devuelve por copia.

			\medskip
			\Statex \underline{Complejidad:} $\Theta(1)$
			\Statex \underline{Justificación:} Como la cantidad de campos es acotada en el contexto de uso de registro, iterarlos tiene como complejidad $\Theta(1)$ y devolverlo también tiene como complejidad $\Theta(1)$.
    	\end{algorithmic}
	{$\textbf{Post}$ $\equiv$ $res \igobs$ Campos$(r)$}
\end{algorithm}


% \begin{algorithm}[H]{{\textbf{iBorrar?}(\In{criterio}{dic}, \In{r}{dic})} $\to$ $res$ : $bool$}
% 	{\\ $\textbf{Pre}$ \#Campos($criterio$) $\equiv 1$}
%     	\begin{algorithmic}[1]	

%     		\State $res \gets$ CoincidenTodos($criterio$, Campos($criterio$), $r$) \Comment $\Theta(L)$
		
% 			\medskip
% 			\Statex \underline{Complejidad:} $\Theta(L)$
% 			\Statex \underline{Justificación:} Como se por requiere que \#Campos($criterio$) $\equiv 1$ y la complejidad de Campos y de CoincidenTodos es de $\Theta(L)$, la complejidad de esta función es de $\Theta(L)$.
%     	\end{algorithmic}
% 	{$\textbf{Post}$ $\equiv$ $res \igobs$ Borrar?($criterio$, $r$)}
% \end{algorithm}


\begin{algorithm}[H]{{\textbf{iCoincidenTodos}(\In{r1}{dic}, \In{cc}{Conj(Campo)}, \In{r2}{dic})} $\to$ $res$ : $bool$}
	{\\ $\textbf{Pre}$ $cc \subseteq$ Campos($r1$) $\cap$ Campos($r2$)}
    	\begin{algorithmic}[1]

    		\State $itConj(Campo): itC \gets$ CrearIt($cc$) \Comment $\Theta(1)$
    		\State $res \gets true$ \Comment $\Theta(1)$

    		\While{HaySiguiente($itC$) $\land$ $res$} \Comment $\Theta(1)$
    			
    			\State $res \gets$ Significado($r1$, Siguiente($itC$)) == Significado($r2$, Siguiente($itC$)) \Comment $\Theta(L)$
    			
    			\State $Avanzar(itC)$ \Comment $\Theta(1)$
    		
    		\EndWhile 
    		\Comment $\Theta(\#(cc)*L)$

			\medskip
			\Statex \underline{Complejidad:} $\Theta(\#(cc)*L)$ = $\Theta(L)$
			\Statex \underline{Justificación:} Itera sobre el conjunto de campos $cc$ y compara los significados de ese campo en ambos registros. Como por requiere se que $cc \subseteq$ Campos($r1$) $\cap$ Campos($r2$) y se que por contexto de uso que la cantidad de campos está acotada (por ende, Significado tiene complejidad $\Theta(1)$), entonces iterar sobre el conjunto cuesta $\Theta(1)$. Los significados del registro no están acotados, esto hace que la comparación entre ellos tengan complejidad $\Theta(L)$ donde $L$ es la máxima longitud de un valor STRING de un registro en la tabla pasada por parámetro.
    	\end{algorithmic}
	{$\textbf{Post}$ $\equiv$ $res \igobs$ CoincidenTodos($r1$, $cc$, $r2$)}
\end{algorithm}


% \begin{algorithm}[H]{{\textbf{iCoincideAlguno}(\In{r1}{dic}, \In{cc}{ItConj(Campo)}, \In{r2}{dic})} $\to$ $res$ : $bool$}
% 	{\\ $\textbf{Pre}$ $cc \subseteq$ Campos($r1$) $\cap$ Campos($r2$)}
%     	\begin{algorithmic}[1]

%     		\State $itConj(Campo): itC \gets$ CrearIt($cc$) \Comment $\Theta(1)$
%     		\State $res \gets false$ \Comment $\Theta(1)$

%     		\While{HaySiguiente($itC$) $\land$ $!res$} \Comment $\Theta(1)$

%     			\State $res \gets$  Significado($r1$, Siguiente($itC$)) == Significado($r2$, Siguiente($itC$)) \Comment $\Theta(L)$
%     			\State Avanzar($itC$) \Comment $\Theta(1)$
    		
%     		\EndWhile 
%     		\Comment $\Theta(\#(cc)*L)$

% 			\medskip
% 			\Statex \underline{Complejidad:} $\Theta(\#(cc)*L)$ = $\Theta(L)$
% 			\Statex \underline{Justificación:} Itera sobre el conjunto de campos $cc$ y compara los significados de ese campo en ambos registros. Como por requiere se que $cc \subseteq$ Campos($r1$) $\cap$ Campos($r2$) y se que por contexto de uso que la cantidad de campos está acotada (por ende, Significado tiene complejidad $\Theta(1)$), entonces iterar sobre el conjunto cuesta $\Theta(1)$. Los significados del registro no están acotados, esto hace que la comparación entre ellos tengan complejidad $\Theta(L)$ donde $L$ es la máxima longitud de un valor STRING de un registro en la tabla pasada por parámetro.
%     	\end{algorithmic}
% 	{$\textbf{Post}$ $\equiv$ $res \igobs$ CoincideAlguno($r1$, $cc$, $r2$)}
% \end{algorithm}


\begin{algorithm}[H]{{\textbf{iAgregarCampos}(\Inout{r1}{dic}, \In{r2}{dic})}}
	{\\ $\textbf{Pre}$ true}
    	\begin{algorithmic}[1]

    		\State $Conj(campo): cr1 \gets$ Campos($r1$) \Comment $\Theta(1)$
    		\State $Conj(campo): cr2 \gets$ Campos($r2$) \Comment $\Theta(1)$
    		\\

			\State $\backslash\backslash$ RestaCampos le quita a $cr2$ los campos que estan en $cr1$
			\State $\backslash\backslash$ al final quedaria $cr2 = cr2_o - (cr2_o \cap cr1)$

    		\State RestaCampos($cr2$, $cr1$) \Comment $\Theta(\#(cr1) * \#(cr2)) = \Theta(1)$
    		\\

    		\State $itConj(Campo): itC \gets$ CrearIt($cr2$) \Comment $\Theta(1)$
    	
    		\While{HaySiguiente($itC$)} \Comment $\Theta(1)$

    			\\
    			\State $\backslash\backslash$ Agrego a $r1$ los registros de $r2$ que no están en $r1$.
    			\State $\backslash\backslash$ Para esto utilzo el conjunto $cr2$ que tiene los campos que no tienen en común $r1$ y $r2$.
    			\State $\backslash\backslash$ Ya que quite los registros en común cuando utilizamos la función RestaCampos.
    			\State $\backslash\backslash$ Cumple el pre de DefinirRapido, estoy agregando claves que no están definidas.
    			\State DefinirRapido($r1$, Siguiente($itC$), Significado($r2$, Siguiente($itC$))) \Comment $\Theta(copy(s))$
    			\\
    			
    			\State Avanzar($itC$) \Comment $\Theta(1)$
    			\\
    		\EndWhile 
    		\Comment $\Theta(\#(cr2)*copy(s)) = \Theta(copy(s))$
		
			\medskip
			\Statex \underline{Complejidad:} $\Theta(copy(s))$ = $\Theta(L)$
			\Statex \underline{Justificación:} Se crean dos conjuntos de campos que cuestan $\Theta(1)$ cada uno, dado que la cantidad de campos de cada registro está acotada por el contexto de uso. Luego se utiliza la función RestaCampos que filtra los campos de $r1$ que están en $r2$. La complejidad de esta función es $\Theta(\#(cr1) * \#(cr2))$ pero como los conjuntos de campos son los campos de los registros $r1$ y $r2$ y estos, por el contexto de uso, tienen una cantidad de campos acotada sumado a que los campos son STRINGS acotados, esto hace que la funcion RestaCampos termine teniendo una complejidad de $\Theta(1)$. Despues se itera el conjunto $cr2$ que posee los campos que poseía menos la intersección de este conjunto con $cr1$, y como la cantidad del conjunto está acotada, iterar este conjunto tiene como complejidad $\Theta(1)$. Dentro del ciclo se utiliza la función DefinirRapido que define en un diccionario en $\Theta(copy(s))$ = $\Theta(L)$ ya que los significados no están acotados, y cumple con el requiere de la función ya que los elementos que se están agregando son los que no están definidos en $r1$ de $r2$. Y, también, utiliza la función Significado del registro $r2$ que, como la cantidad de campos es acotada, tiene como complejidad $\Theta(1)$. Por lo tanto, la función termina teniendo como complejidad $\Theta(L)$ que surge de generar la copia del significado al definir en el registro.

    	\end{algorithmic}
	{$\textbf{Post}$ $\equiv$ $res \igobs$ AgregarCampos($r1$, $r2$)}
\end{algorithm}


\begin{algorithm}[H]{{\textbf{RestaCampos}(\Inout{cr2}{Conj(Campo)}, \In{cr1}{Conj(Campo)})}}
	{\\ $\textbf{Pre}$ $cr2 = cr2_o$}
    	\begin{algorithmic}[1]

    		\State $itConj(campo): itCr2 \gets$ CrearIt($cr2$) \Comment $\Theta(1)$

    		\While{HaySiguiente($itCr2$)} \Comment $\Theta(1)$

    			\If {Pertenece?($cr1$, Siguiente($itCr2$))} \Comment $\Theta(\#(cr1))$
    				\State EliminarSiguiente($itCr2$) \Comment $\Theta(1)$
    			\Else
    				\State Avanzar($itCr2$) \Comment $\Theta(1)$
    			\EndIf

    		\EndWhile \Comment $\Theta(\#(cr1) * \#(cr2))$
		
			\medskip
			\Statex \underline{Complejidad:} $\Theta(\#(cr1) * \#(cr2))$
			\Statex \underline{Justificación:} Recorre todos los campos del conjunto $cr2$ revisando si pertenecen al conjunto $cr1$. La funcion Pertence? de conjunto hace una comparacion de todos los elementos revisando si está el pasado como parametro, esto hace que la complejidad sea $\Theta\left(\displaystyle\sum_{a' \in c}equal(a,a')\right)$, pero como los campos tienen una longitud acotada de letras la complejidad de cada comparación es $\Theta(1)$ y la complejidad total del pertenece termina siendo $\Theta(\#cr1)$ porque en peor caso debe recorrer todos los elementos.
    	\end{algorithmic}
	{$\textbf{Post}$ $\equiv$ $cr2 = cr2_o - cr1$)}
\end{algorithm}

\pagebreak
\section{M\'odulo Tabla}

\subsection{Interfaz}

\textbf{se explica con}: \tadNombre{Tabla}.

\textbf{géneros}: tabla


~


\subsubsection{Operaciones b\'asicas de Tabla}

\begin{Interfaz}

\InterfazFuncion{nuevaTabla}{\In{nombre}{string}, \In{claves}{conj(campo)}, \In{columnas}{registro} }{tabla}
[$claves \neq \emptyset \land claves \subseteq$ campos($columnas$)]
{$res \igobs$ nuevaTabla($nombre$, $claves$, $columnas$)}
[$\Theta(1)$]
[Crea una nueva tabla tomando como parametros un $nombre$, sus $claves$ y sus $columnas$.]
[Los parametros nombre, claves y columnas toman por copia.]

~

\InterfazFuncion{agregarRegistro}{\Inout{t}{tabla,\In{r}{registro}}}{itConj(registro)}
[$t \igobs t_o \land$ campos($r$) $\igobs$ campos(columnas($t$)) $\land$ puedoInsertar?($r$, $t$)]
{$t \igobs agregarRegistro(r, t_0) \land$ haySiguiente(res) $\yluego$ Siguiente(res) $=$ r \\ $\land$ alias(esPermutacion?(SecuSuby(res), t.registros)}
[$\Theta(L) + O(L + log (\#t.registros))$]
[Agrega el registro $r$ a la tabla $t$.]
[El registro r se agrega por copia. El iterador es no modificable y se invalida al insertar o borrar registros de la tabla]

~

\InterfazFuncion{borrarRegistro}{\Inout{t}{tabla},\In{crit}{registro}}{}
[$t \igobs t_o \land$ \#campos($crit$)$ = 1 \yluego$ dameUno(campos($crit$)) $\in$ claves($t$)]
{$t \igobs$ borrarRegistro($crit$, $t_o$)}
[$O(L + log \#registros + L * \#t.registros)$\\ 
Si el criterio no es indice, entonces $O(L * \#t.registros)$ si es string o $\Theta(L + \#t.registros)$ si es Nat. \\
Si el criterio es indice Nat $O(L + log(\#t.registros))$ \\
Si el criterio es indice String $O(L + log(\#t.registros))$ 
]
[Borra los registros de la tabla $t$ que coincidan con el registro $crit$.]
[]

~

\InterfazFuncion{indexar}{\Inout{t}{tabla} , \In{c}{campo}}{}
[$t \igobs t_o \land$ puedeIndexar($c$, $t$)]
{$t \igobs$ indexar($c$, $t_0$)}
[Si el indice es Nat $O(\#registros * log(\#registros))$
Si el indice es String $O(\#registros * L)$]
[Agrega un indice a la tabla $t$.]
[]

~

\InterfazFuncion{nombre}{\In{t}{tabla}}{string}
[true]
{$res \igobs$ nombre($t$)}
[$\Theta(1)$]
[Devuelve el nombre de la tabla.]
[El string no es modificable.]

~

\InterfazFuncion{claves}{\In{t}{tabla} }{conj(campo)}
[true]
{$res \igobs$ claves($t$)}
[$\Theta(1)$]
[Devuelve el conjunto de claves de la tabla.]
[El conjunto se devuelve por copia]

~

\InterfazFuncion{indices}{\In{t}{tabla} }{conj(campo)}
[true]
{$res \igobs$ indices($t$)}
[$\Theta(1)$]
[Devuelve el conjunto de indices de la tabla.]
[El conjunto se devuelve por copia]

~

\InterfazFuncion{campos}{\In{t}{tabla}}{conj(campo)}
[true]
{$res \igobs$ columnas($t$)}
[$\Theta(1)$]
[Devuelve el conjunto de columnas de la tabla.]
[El conjunto se devuelve por copia]

~

\InterfazFuncion{tipoCampo}{\In{t}{tabla},\In{c}{campo}}{tipo}
[true]
{$res \igobs$ tipo(obtener(c,t.columnas)))}
[$\Theta(1)$]
[Devuelve el tipo del campo pedido]
[]

~

\InterfazFuncion{registros}{\In{t}{tabla} }{conj(registro)}
[true]
{$res \igobs$ registros($t$))}
[$\Theta(1)$]
[Devuelve el conjunto de registros de la tabla]
[El conjunto se devuelve por referencia no modificable]

~

\InterfazFuncion{cantidadDeAccesos}{\In{t}{tabla}}{nat}
[true]
{$res \igobs$ cantidadDeAccesos($t$)}
[$\Theta(1)$]
[Devuelve un natural con la cantidad de modificaciones de la tabla.]
[]

% ~

% % \InterfazFuncion{puedoInsertar?}{\In{r}{registro}, \In{t}{tabla} }{bool}
% % [true]
% % {$res \igobs$ puedoInsertar?($r$, $t$)}
% % [$\Theta(1)$]
% % [Devuelve true si se puede insertar el registro $r$ en la tabla.]
% % []

~

\InterfazFuncion{minimo}{\In{c}{campo}, \In{t}{tabla} }{dato}
[$\neg \emptyset$?(registros($t$)) $\land$ $c \in$ indices($t$)]
{$res \igobs$ minimo($c$, $t$)}
[$\Theta(1)$]
[Devuelve el dato mínimo de un campo de la tabla.]
[]

~

\InterfazFuncion{maximo}{\In{c}{campo}, \In{t}{tabla} }{dato}
[$\neg \emptyset$?(registros($t$)) $\land$ $c \in$ indices($t$)]
{$res \igobs$ maximo($c$, $t$)}
[$\Theta(1)$]
[Devuelve el dato máximo de un campo de la tabla.]
[]

~

\InterfazFuncion{puedeIndexar}{\In{c}{campo}, \In{t}{tabla} }{bool}
[true]
{$res \igobs$ puedeIndexar($c$, $t$)}
[$\Theta(1)$]
[Devuelve true si puede usar a $c$ como indice.]
[]

~

\InterfazFuncion{combinarRegistros}{\In{c}{campo},\In{t}{tabla},\In{cr}{conj(registro)}}{conj(registro)}
[$enTodos(c,registros(t)) \land enTodos(c,cr)$]
{$res \igobs$ combinarRegistros(c,registros(t),cr)}
[$\Theta$($\#$cr) * (O(log $\#$registros(t)) + $\Theta$(L)) con indice nat. \\
$\Theta(\#cr$ * $L)$	con indice string. \\
$\Theta(\#registros(t)$ * $\#cr$ * $L)$ sin indices.]
[Devuelve un conjunto de registros resultado de tomar una tabla y un conjuntos de registros y dejar solo los que tienen el mismo valor en el campo c]
[La tabla t y el conjunto cr se pasan por referencia. Se devuelve un conjunto de registros por copia.]

~

\InterfazFuncion{coincidencias}{\In{crit}{registro},\In{t}{tabla}}{conj(registro)}
[$campos(crit) \subseteq campos(t)$]
{$res \igobs$ coincidencias(r,registros(t))}
[$O(log ($\#$registros(t)) + L * \#($reg mismo indice$))$ con indice nat. \\
$O(L * \#($reg mismo indice$))$	con indice string. \\
$O(\#registros(t) * L)$	sin indices.]
[Devuelve un conjunto de registros resultado de tomar una tabla y un registro de criterio y dejar solo los que coinciden en campo y valor.]
[La tabla t y el registro r se pasan por referencia. Se devuelve un conjunto de registros por copia.]

\end{Interfaz}

~

\pagebreak
\subsubsection{Representaci\'on de Tabla}

\begin{Estructura}{Tabla}[estr]

	\begin{Tupla}[estr]
		\tupItem{nombre}{String}%
		\tupItem{\\ indices}{tupla<iNat:campo,iStr:campo>}
		\tupItem{\\ claves}{conj(campo)}
		\tupItem{\\ columnas}{registro}
		\tupItem{\\ registros}{conj(registro)}
		\tupItem{\\ indiceNat}{diccNat(Nat, conj(itConj(registro)))}
		\tupItem{\\ indiceStr}{diccString(String, conj(itConj(registro))}
		\tupItem{\\ \#accesos}{Nat}
	\end{Tupla}

\end{Estructura}

Todos los tipo conj se refieren al modulo conjunto lineal de la catedra. \\
Registro se refiere al modulo adjunto en la documentacion. \\
Campo es string.
DiccNat se refiere a un diccionario de con busqueda, borrado e inserci\'on O(log n) adjunto en la documentaci\'on.
DiccString se refiere a un diccionario de con busqueda, borrado e inserci\'on O(L(largo de la clave mas larga)) adjunto en la documentaci\'on.

Los significados de los indices contienen un conjunto de interadores de registros que apuntan a estr.registros. Se tiene un conjunto para dar soporte a indices con valores repetidos (que no son clave).


\subsubsection{Invariante de Representaci\'on}

\begin{enumerate}
	\item e.claves esta incluido en campos(e.columnas)
	\item Si no hay indice nat en la tabla e.indices.iNat == <> $\land$ e.indiceNat == Vacio() 
	\item Si no hay indice str en la tabla e.indices.iStr == <> $\land$ e.indiceStr == Vacio() 
	\item Si hay indice nat (e.indice.iNat pertenece a campos(e.columnas)) $\land$ para toda claves(e.indiceNat) existe valorNat perteneciente a dameColumna(t,e.indices.iNat) y el conjunto de iteradores de los significados apunta a los registros cuyo campo indice.iNat es igual a la clave del indiceNat.
	\item Si hay indice string (e.indice.iStr pertenece a campos(e.columnas)) $\land$ para toda claves(e.indiceStr) existe valorStr perteneciente a dameColumna(t,e.indices.iStr) y el conjunto de iteradores de los significados apunta a los registros cuyo campo indice.iStr es igual a la clave del indiceStr.
	\item para todo registro del conjunto e.registros , las columnas son las mismas que e.columnas \\
			y los tipo de datos tambien.
	\item la columnas claves no tiene repetidos
	\item e.\#accesos $\geq$ $|$e.registros$|$
\end{enumerate}

~

\Rep[estr][e]{\\ e.claves $\subseteq$ campos(e.columnas) $\yluego$ \Comment (1)\\
	(vacio?(e.indices.iNat) $\Rightarrow$ vacio?(e.indiceNat)) $\yluego$ \Comment (2)\\
	(vacio?(e.indices.iStr) $\Rightarrow$ vacio?(e.indiceStr)) $\yluego$ \Comment (3)\\
	($\neg$ vacio?(e.indices.iNat) $\Rightarrow_L$ \\ 
	(e.indices.iNat $\in$ campos(e.columnas) $\land$  clavesIndiceNatValidas(e) $\land$ itRegValidosNat(e) $\yluego$ \Comment (4)\\
	($\neg$ vacio?(e.indices.iStr) $\Rightarrow_L$ \\
	(e.indices.iStr $\in$ campos(e.columnas) $\land$ clavesIndiceStrValidas(e)  $\land$ itRegValidosStr(e) $\yluego$ \Comment (5)\\
	(camposValidos(e) $\land$ (tiposDeDatoValidos(e))) $\yluego$ \Comment (6)\\
	(($\forall$ c:campo)(c $\in$ e.claves) $\Rightarrow$ (sinRepetidos(dameColumna(c,e)))) $\yluego$ \Comment (7)\\
	(e.\#accesos $\geq$ $\#(e.registros)$) \Comment (8)
}\mbox{}
\\

clavesIndiceNatValidas(e) $\equiv$ ($\forall$ c:Nat)((c $\in$ claves(e.indiceNat)) $\Rightarrow$ (datoNat(c) $\in$ dameColumna(e.indice.iNat,e)))
clavesIndiceStrValidas(e) $\equiv$ ($\forall$ c:String)((c $\in$ claves(e.indiceStr)) $\Rightarrow$ (datoStr(c) $\in$ dameColumna(e.indice.iStr,e)))
camposValidos(e) $\equiv$ ($\forall$ r:registro)((r $\in$ e.registros) $\Rightarrow$ (campos(r) $\igobs$ campos(e.columnas))) \\
tiposDeDatoValidos(e) $\equiv$ ($\forall$ r:registro)((r $\in$ e.registros) $\Rightarrow$ (mismosTipos(r,e.columnas))) \\
mismoTipos(r1,r2) $\equiv$ ($\forall$ c:campo)((c $\in$ campos(r1)) $\Rightarrow$ (tipo(obtener(c,r1)) $==$ tipo(obtener(c,r2)))) \\
camposClave(r1,r2) $\equiv$ ($\forall$ c:campo)((c $\in$ campos(r1)) $\Rightarrow$ (tipo(obtener(c,r1)) $==$ tipo(obtener(c,r2)))) \\
itRegValidosNat(e) $\equiv$ ($\forall$ c:Nat)((c $\in$ claves(e.indiceNat)) $\Rightarrow$ \\ 
registrosValidos(obtener(c,e.indiceNat),e.registros,e.indices.iNat,datoNat(c))) \\
itRegValidosStr(e) $\equiv$ ($\forall$ c:String)((c $\in$ claves(e.indiceString)) $\Rightarrow$ \\
registrosValidos(obtener(c,e.indiceStr),e.registros,e.indices.iStr,datoStr(c))) \\
registrosValidos(citr,cr,c,d) $\equiv$ ($\forall$ it:itReg)((it $\in$ citr) $\Rightarrow$ (haySiguiente(it) $\yluego$ siguiente(it) $\in$ cr $\yluego$ obtener(c,siguiente(it))) = d) \\


\subsubsection{Funci\'on de Abstracci\'on}


\Abs[estr]{tabla}[e]{t}{e.nombre == nombre(t) \yluego \\
						e.claves == claves(t) \yluego \\
						campos(e.columnas) == campos(t) \yluego \\
						tipoCamposValidos(e,t) \yluego \\
						e.registros = registros(t)  \yluego \\
						e.\#accesos = cantidadAccesos(t);

}					

\tadAxioma{tipoCamposValidos(e,t)}{
	($\forall$ c:Campo $\land$ c $\in$ campos(e.columnas) $\Rightarrow$ tipoCampo(c,t) == tipoCampo(e,c))
}

\subsection{Algoritmos}


\lstset{style=alg}


\begin{algorithm}[H]{\textbf{iNuevaTabla}(\In{nombre}{String},\In{claves}{conj(Campos)},\In{columnas}{registro}) $\to$ $res$ : tabla}
	\begin{algorithmic}[1]
		\State $res.nombre \gets copiar(nombre)$ \Comment $\Theta(|nombre|)$
		\State $res.columnas \gets copiar(columnas)$ \Comment costo de copiar el registro $\Theta(\sum_{r \in R}{(copiar(campo)+copiar(dato))})$
		\State $res.claves \gets copiar(claves)$ \Comment costo de copiar el conjunto $\Theta(\sum_{c \in C}{(copiar(campo))})$
		\State $res.indices \gets \langle<>,<>\rangle$ crea una tupla \Comment $\Theta(1)$
		\State $res.registros \gets Vacio()$ \Comment $\Theta(1)$
		\State $res.\#accesos \gets 0$ \Comment $\Theta(1)$
		\State $res.indiceNat \gets crearDiccionario()$ \Comment $\Theta(1)$
		\State $res.indiceStr \gets crearDiccionario()$ \Comment $\Theta(1)$
		\medskip
		\Statex \underline{Complejidad:} $\Theta(1)$
		\Statex \underline{Justificación:} Como el nombre de la tabla es acotado nombre se copia en O(1); la cantidad de campos es actoda y sus nombres tambien entonces la copia del registro con los datos vacios es O(1); por lo mismo la copia de las claves tambien es O(1), y el resto son operaciones elementales quedando entonces la complejidad O(1)  
    \end{algorithmic}
\end{algorithm}

\begin{algorithm}[H]{\textbf{iAgregarRegistro}(\Inout{t}{tabla},\In{r}{registro}) $\to$ $res$ : itConj(registro)}
	\begin{algorithmic}[1]
		\State $itConj(registro): itreg \gets agregarRapido(t.registros,r)$ \Comment el elemento no esta por \textbf{Pre} $\Theta(copiar(r))$
		\State $t.\#accesos \gets t.\#accesos + 1$ \Comment $\Theta(1)$
		\State $agregarIndices(t,itreg)$	\Comment actualizacion de los indices $O(L + log(\#t.registros)$
		\State $res \gets itreg$
		\medskip
		\Statex \underline{Complejidad:} $\Theta(L) + O(L + log (\#t.registros))$
		\Statex \underline{Justificación:} Copiar a r es igual a L; la actualizacion de los indices, que sino esta indexado es $\Theta(1)$, si solo tiene un indice Nat es $O(log \#(t.registros))$, si solo tiene un indice String es $O(L)$, y si tiene ambas es $O(L + log(\#t.registros))$

	\end{algorithmic}
\end{algorithm}


\begin{algorithm}[H]{\textbf{iBorrarRegistro}(\Inout{t}{tabla},\In{crit}{registro})}
	\begin{algorithmic}[1]
		\State $campo: cCriterio \gets siguiente(CrearIt(campos(crit)))$ \Comment $\Theta(1)$
		\State $dato: dCriterio \gets obtener(crit,cCriterio))$	\Comment $\Theta(\#campos)$
		\If{$cCriterio = t.indices.iNat$}	\Comment $\Theta(1)$
			\State $borrarPorIndiceNat(t,crit)$ \Comment $O(log(\#t.registros))$
		\Else
			\If{$cCriterio = t.indices.iStr$}	\Comment String acotado $\Theta(1)$
				\State $borrarPorIndiceStr(t,crit)$ \Comment $O(L)$
			\Else
				\State $borrarSinIndice(t,crit)$ \Comment $\Theta(\#t.registros * L)$
			\EndIf
		\EndIf
		\State $borrarIndices(t,crit)$ \Comment $O(L + log(\#t.registros))$
		\medskip
		\Statex \underline{Complejidad:}
		\begin{itemize}
		\item Si el criterio no es indice, entonces $O(L * \#t.registros)$ si es string o $\Theta(\#t.registros)$ si es Nat. 
		\item Si el criterio es indice Nat $O(log(\#t.registros))$
		\item Si el criterio es indice String $O(L)$ 
		\end{itemize}
		\\
		Ademas se suma el borrado de los indices que en peor caso es $O(L+log(\#t.registros))$
		\Statex \underline{Justificación:} Segun la rama del algoritmo que se tome se aplican las complejidades de las funciones auxiliares invocadas donde esta debidamente justificada su complejidad. 
	\end{algorithmic}
\end{algorithm}

\begin{algorithm}[H]{\textbf{iBorrarPorIndiceNat}(\Inout{t}{tabla},\In{crit}{registro})}
	{\\ $\textbf{Pre} \equiv \{t \igobs t_0 \land \#campos(crit) = 1 \yluego dameUno(campos(crit)) \in claves(t)\}$}
	\begin{algorithmic}[1]
		\State $campo: cCriterio \gets siguiente(CrearIt(campos(crit)))$ \Comment $\Theta(1)$
		\State $dato: dCriterio \gets obtener(crit,cCriterio))$	\Comment $\Theta(\#campos)$
		\State $conj(itConj(registro)): regMismoIndice \gets obtener(t.indiceNat,valorNat(dCriterio)$ \Comment $O(log \#(t.registros))$
		\State $itConj(itConj(registro)): itreg \gets CrearIt(regMismoIndice)$ \Comment $\Theta(1)$
		%\While{$haySiguiente?(itreg)$} \Comment $\Theta(\#regMismoIndice)$
		\State $eliminarSiguiente(siguiente(itreg))$	\Comment solo borro el siguiente porque es un campos clave $\Theta(1)$
		%\State $avanzar(itreg)$	\Comment $\Theta(1)$
		\State $t.\#accesos \gets t.\#accesos + 1$ \Comment $\Theta(1)$
		%\EndWhile
		\medskip
		\Statex \underline{Complejidad:} $O(log \#t.registros)$
		\Statex \underline{Justificación:} Es el tiempo de busqueda en el diccionario, el resto es O(1) 
	\end{algorithmic}
	{$\textbf{Post} \equiv \{t \igobs borrarRegistro(crit,t_0)\}$}
\end{algorithm}

\begin{algorithm}[H]{\textbf{iBorrarPorIndiceString}(\Inout{t}{tabla},\In{crit}{registro})}
	{\\ $\textbf{Pre} \equiv \{t \igobs t_0 \land \#campos(crit) = 1 \yluego dameUno(campos(crit)) \in claves(t)\}$}
	\begin{algorithmic}[1]
		\State $campo: cCriterio \gets siguiente(CrearIt(campos(crit)))$ \Comment $\Theta(1)$
		\State $dato: dCriterio \gets obtener(crit,cCriterio))$	\Comment $\Theta(\#campos)$
		\State $conj(itConj(registro)): regMismoIndice \gets obtener(t.indiceString,valorString(dCriterio)$ \Comment $O(L)$
		\State $itConj(itConj(registro)): itreg \gets CrearIt(regMismoIndice)$ \Comment $\Theta(1)$
		%\While{$haySiguiente?(itreg)$} \Comment $\Theta(\#regMismoIndice)$
		\State $eliminarSiguiente(siguiente(itreg))$	\Comment solo borro el siguiente porque es un campos clave $\Theta(1)$
		%	\State $avanzar(itreg)$	\Comment $\Theta(1)$
		\State $t.\#accesos \gets t.\#accesos + 1$ \Comment $\Theta(1)$
		%\EndWhile
		\medskip
		\Statex \underline{Complejidad:} $O(L)$
		\Statex \underline{Justificación:} Es el tiempo de busqueda en el diccionario, el resto es O(1)
	\end{algorithmic}
	{$\textbf{Post} \equiv \{t \igobs borrarRegistro(crit,t_0)\}$}
\end{algorithm}

\begin{algorithm}[H]{\textbf{iBorrarSinIndice}(\Inout{t}{tabla},\In{crit}{registro})}
	{\\ $\textbf{Pre} \equiv \{t \igobs t_0 \land \#campos(crit) = 1 \yluego dameUno(campos(crit)) \in claves(t)\}$}
	\begin{algorithmic}[1]
		\State $campo: cCriterio \gets siguiente(CrearIt(campos(crit)))$ \Comment $\Theta(1)$
		\State $dato: dCriterio \gets obtener(crit,cCriterio))$	\Comment $\Theta(\#campos)$
		\State $itConj(registros):itreg \gets CreatIt(t.registros)$ \Comment $\Theta(1)$
		\While{$haySiguiente?(itreg)$} \Comment $\Theta(\#(t.registros))$
			\State $dato: dBorrar = obtener(siguiente(itreg),cCriterio)$
			\If{$dBorrar = dCriterio$}	\Comment O(min(valosStr(dCriterio),valorStr(dBorrar)))
				\State $eliminarSiguiente(itreg)$	\Comment $\Theta(1)$
				\State $t.\#accesos \gets t.\#accesos + 1$ \Comment $\Theta(1)$
			\EndIf
			\State $avanzar(itreg)$	\Comment $\Theta(1)$
		\EndWhile
		\medskip
		\Statex \underline{Complejidad:} $\Theta(\#t.registros * O(L))$ si el campo es string $\Theta(\#t.registro)$ si es nat
		\Statex \underline{Justificación:} Compara el campo criterio con todos los regitros de la tabla, que en el caso de un campo Nat es O(1) y en caso de un string O(longitud de string) donde L es el string mas largo y el peor caso.
	\end{algorithmic}
	{$\textbf{Post} \equiv \{t \igobs borrarRegistro(crit,t_0)\}$}
\end{algorithm}

\begin{algorithm}[H]{\textbf{iIndexar}(\Inout{t}{tabla},\In{c}{campo}) $\to$ res: bool}
	\begin{algorithmic}[1]
		\State $res \gets iPuedeIndexar?(t,c)$	\Comment $\Theta(1)$
		\If{$res$}
			\If{$tipoCampo(t,c)$}	\Comment $\Theta(1)$
				\State $indexarNat(t,c)$	\Comment $O(\#registros * log(\#registros))$
			\Else
				\State $indexarStr(t,c)$	\Comment $O(\#registros * L)$
			\EndIf
		\EndIf
		\medskip
		\Statex \underline{Complejidad:} 
		\begin{itemize}
		\item Si el indice es Nat $O(\#registros * log(\#registros))$
		\item Si el indice es String $O(\#registros * L)$
		\end{itemize}
		\Statex \underline{Justificación:} Segun la rama que se tome se pagan las complejidades de los auxiliares que estan descriptos como cota superior, peor caso.
	\end{algorithmic}
\end{algorithm}

\begin{algorithm}[H]{\textbf{iIndexarNat}(\Inout{t}{tabla},\In{c}{campo}) $\to$ res: bool}
	{\\ $\textbf{Pre} \equiv \{ t \igobs t_0 \land Nat?(obtener(c,t.columnas)) \land puedoIndexar?(c,t)\}$}
	\begin{algorithmic}[1]
		\State $itConj(registros):itreg \gets CreatIt(t.registros)$ \Comment $\Theta(1)$
		\While{$haySiguiente?(itreg)$} \Comment $\Theta(\#t.registros)$
			\State $agregarIndiceNat(t,itreg)$ \Comment $O(log(\#t.indiceNat))$
			\State $avanzar(itreg)$	\Comment $\Theta(1)$
		\EndWhile
		\State $t.indices.iNat \gets c$	\Comment $\Theta(1)$
		\medskip
		\Statex \underline{Complejidad:} $O\left(\#t.registros * \displaystyle\sum_{n = 0}^{\ell}{log(n)}\right)$ donde $\ell = \#registros$
		\Statex \underline{Justificación:} Itera sobre todos los registros de la tabla y por cada uno paga el log de la cantidad de indices agregados hasta el momento.
	\end{algorithmic}
	{$\textbf{Post} \equiv t_0 \igobs indexar(t)$}
\end{algorithm}

\begin{algorithm}[H]{\textbf{iIndexarStr}(\Inout{t}{tabla},\In{c}{campo},\In{itreg}{itConj(registros)})}
	{\\ $\textbf{Pre} \equiv \{t \igobs t_0 \land String?(obtener(c,t.columnas)) \land puedoIndexar?(c,t)\}$}
	\begin{algorithmic}[1]
		\State $itConj(registros):itreg \gets CreatIt(t.registros)$ \Comment $\Theta(1)$
		\While{$haySiguiente?(itreg)$} \Comment $\Theta(\#t.registros)$
			\State $agregarIndiceString(t,itreg)$ \Comment $O(L)$
			\State $avanzar(itreg)$	\Comment $\Theta(1)$
		\EndWhile
		\State $indices.iString \gets copiar(c)$	\Comment string acotado entonces $\Theta(1)$
		\medskip
		\Statex \underline{Complejidad:} $O(\#t.registros * L)$
		\Statex \underline{Justificación:} Itera sobre todos los registros de la tabla y por cada uno paga la longitud del string insertado mas largo. Se considera todas las iteraciones el string mas largo como cota superior. 
	\end{algorithmic}
	{$\textbf{Post} \equiv t_0 \igobs indexar(t)$}
\end{algorithm}

\begin{algorithm}[H]{\textbf{iPuedeIndexar}(\In{t}{tabla},\In{c}{campo}) $\to$ res: bool}
	\begin{algorithmic}[1]
		\State $res \gets pertenece?(t.columnas,c)$	\Comment por campos acotados $\Theta(1)$
		\If{$tipoCampo(t,c)$}
			\State $res \gets res \land vacia?(indices.iNat))$ \Comment $\Theta(1)$
		\Else
			\State $res \gets res \land vacia?(indices.iString))$ \Comment $\Theta(1)$
		\EndIf
		\medskip
		\Statex \underline{Complejidad:} $\Theta(1)$
		\Statex \underline{Justificación:} Por campos acotados y operaciones elementales
	\end{algorithmic}
\end{algorithm}

\begin{algorithm}[H]{\textbf{iNombre}(\In{t}{tabla}) $\to$ res: string}
	\begin{algorithmic}[1]
		\State $res \gets t.nombre$	\Comment $\Theta(1)$
		\medskip
		\Statex \underline{Complejidad:} $\Theta(1)$
		\Statex \underline{Justificación:} Devuelve por referencia
	\end{algorithmic}
\end{algorithm}

\begin{algorithm}[H]{\textbf{iClaves}(\In{t}{tabla}) $\to$ res: conj(campo)}
	\begin{algorithmic}[1]
		\State $res \gets t.claves$	\Comment $\Theta(1)$
		\medskip
		\Statex \underline{Complejidad:} $\Theta(1)$
		\Statex \underline{Justificación:} Devuelve por referencia
	\end{algorithmic}
\end{algorithm}

\begin{algorithm}[H]{\textbf{iIndices}(\In{t}{tabla}) $\to$ res: conj(campo)}
	\begin{algorithmic}[1]
		\State $res \gets \emptyset$ \Comment $\Theta(1)$
		\If{$\neg Vacio?(t.indices.iNat$}	\Comment $\Theta(1)$
			\State $agregarRapido(res,t.indices.iNat)$ \Comment $\Theta(1)$
		\EndIf 
		\If{$\neg Vacio?(t.indices.iStr$}	\Comment $\Theta(1)$
			\State $agregarRapido(res,t.indices.iStr)$ \Comment $\Theta(1)$
		\EndIf 
		\medskip
		\Statex \underline{Complejidad:} $\Theta(1)$
		\Statex \underline{Justificación:} Genera un conjunto con los campos indices, que son acotados; entonces $\Theta(1)$ 
	\end{algorithmic}
\end{algorithm}

\begin{algorithm}[H]{\textbf{iCampos}(\In{t}{tabla}) $\to$ res: conj(campo)}
	\begin{algorithmic}[1]
		\State $res \gets campos(t.columnas)$	\Comment $\Theta(1)$
		\medskip
		\Statex \underline{Complejidad:} $\Theta(1)$
		\Statex \underline{Justificación:} Devuelve por referencia
	\end{algorithmic}
\end{algorithm}

\begin{algorithm}[H]{\textbf{iTipoCampo}(\In{t}{tabla},\In{c}{campo}) $\to$ res: tipo}
	\begin{algorithmic}[1]
		\State $res \gets esNat?(obtener(t.columnas,c))$	\Comment $\Theta(1)$
		\medskip
		\Statex \underline{Complejidad:} $\Theta(1)$
		\Statex \underline{Justificación:} Operacion elemental
	\end{algorithmic}
\end{algorithm}

\begin{algorithm}[H]{\textbf{iRegistros}(\In{t}{tabla}) $\to$ res: conj(registro)}
	\begin{algorithmic}[1]
		\State $res \gets t.registros$	\Comment $\Theta(1)$
		\medskip
		\Statex \underline{Complejidad:} $\Theta(1)$
		\Statex \underline{Justificación:} Devuelve por referencia
	\end{algorithmic}
\end{algorithm}


\begin{algorithm}[H]{\textbf{iCantidadDeAccesos}(\In{t}{tabla}) $\to$ res: Nat}
	\begin{algorithmic}[1]
		\State $res \gets t.\#accesos$	\Comment $\Theta(1)$
		\medskip
		\Statex \underline{Complejidad:} $\Theta(1)$
		\Statex \underline{Justificación:} Es una operacion elemental
	\end{algorithmic}
\end{algorithm}

\begin{algorithm}[H]{\textbf{iMinimo}(\In{t}{tabla},\In{c}{campo}) $\to$ res:dato}
	\begin{algorithmic}[1]
		\If{$t.indices.iNat = c$} \Comment  por string acotado $\Theta(1)$
			\State $res \gets datoNat(claveMinima(t.indices.iNat))$ \Comment $\Theta(1)$
		\Else
			\State $res \gets datoString(claveMinima(t.indices.iStr))$ \Comment $\Theta(1)$
		\EndIf
		\medskip
		\Statex \underline{Complejidad:} $\Theta(1)$
		\Statex \underline{Justificación:} Por que los diccionarios devuelven el dato en $\Theta(1)$ y este se devuelve por referencia.
	\end{algorithmic}
\end{algorithm}

\begin{algorithm}[H]{\textbf{iMaximo}(\In{t}{tabla},\In{c}{campo}) $\to$ res:dato}
	\begin{algorithmic}[1]
		\If{$t.indices.iNat = c$} \Comment por string acotado $\Theta(1)$
			\State $res \gets datoNat(claveMaxima(t.indices.iNat))$ \Comment $\Theta(1)$
		\Else
			\State $res \gets datoString(claveMaxima(t.indices.iStr))$ \Comment $\Theta(1)$
		\EndIf
		\medskip
		\Statex \underline{Complejidad:} $\Theta(1)$
		\Statex \underline{Justificación:}  Por que los diccionarios devuelven el dato en $\Theta(1)$ y este se devuelve por referencia.
	\end{algorithmic}
\end{algorithm}

\begin{algorithm}[H]{\textbf{iAgregarIndices}(\Inout{t}{tabla},\In{itreg}{itConj(registro)})}
	{\\ $\textbf{Pre} \equiv \{t \igobs t_0\}$}
	\begin{algorithmic}[1]
		\If{$\neg EsVacio?(t.indices.iNat)$} \Comment $\Theta(1)$
			\State $agregarIndiceNat(t,itreg)$ \Comment $O(log (\#registros))$
		\EndIf

		\If{$\neg EsVacio?(t.indices.iString)$} \Comment $\Theta(1)$
			\State $agregarIndiceString(t,itreg)$ \Comment $O(L)$
		\EndIf
		\medskip
		\Statex \underline{Complejidad:} $O(L + log (\#registros))$
		\Statex \underline{Justificación:} Suma de las complejidades de los auxiliares usados para los 2 tipos de indice. O(L) si solo indice String, O(log(\#registros)) si solo indice Nat, y O(L + log (\#registros)) si tiene ambos indices; O(1) si no hay ninguno.
	\end{algorithmic}
	{$\textbf{Post} \equiv \{(\neg vacio?(t.indices.iStr) \Rightarrow_L t.indiceStr \igobs agregarIndiceNat(t_0,itreg) \land \\
							 (\neg vacio?(t.indices.iNat) \Rightarrow_L t.indiceNat \igobs agregarIndiceStr(t_0,itreg) \}$}
	\\ \\
	{$agregarIndiceNat(t',itr) \equiv$ \IF $def?(clave,t'.indicesNat)$ THEN $definir(Ag(itr,obtener(t'.indiceNat,claveNat(t'))),t'.indicesNat)$ ELSE $definir(\{itr\},t.indicesNat)$ FI $\}$}\\
	{$claveNat(t',itr) \equiv valorNat(obtener(siguiente(itr),t'.indices.iNat))$}\\ \\

	{$agregarIndiceStr(t',itr) \equiv$ \IF $def?(clave,t'.indicesStr)$ THEN $definir(Ag(it,obtener(t'.indiceStr,claveStr(t',itr))),t'.indicesStr)$ ELSE $definir(\{itr\},t.indicesStr	)$ FI $\}$}\\
	{$claveStr(t',itr) \equiv valorStr(obtener(siguiente(itr),t'.indices.iStr))$}\\
		
	
\end{algorithm}

\begin{algorithm}[H]{\textbf{iAgregarIndiceNat}(\Inout{t}{tabla},\In{itreg}{itConj(registro)})}
	{\\ $\textbf{Pre} \equiv \{t \igobs t_0 \land \neg vacio?(t.indices.iNat)\}$}
	\begin{algorithmic}[1]
		\State $Nat: clave \gets valorNat(obtener(siguiente(itreg),t.indices.iNat))$	\Comment por campos acotados $\Theta(1)$
		\If{$definido?(t.indiceNat,clave))$}	\Comment $O(log (\#registros))$
			\State $definir(t.indiceNat,clave,AgregarRapido(obtener(t.indiceNat,clave),itreg))$ \Comment $O(log (\#registros))$
		\Else
			\State $definir(t.indiceNat,clave,\{itreg\})$ \Comment $O(log (\#registros))$
		\EndIf
		\medskip
		\Statex \underline{Complejidad:} $O(log (\#registros))$
		\Statex \underline{Justificación:} Busca en el diccionario del indice y lo redefine; ambas operaciones el diccionario las realiza en $O(log (\#registros))$; entonces la complejidad temporal $O(log (\#registros) + log (\#registros))$, quedando $O(log (\#registros))$ por algebra de ordenes. \\ Se usa el agregarRapido del conjunto lineal para agregar el iterador en $O(1)$.
	\end{algorithmic}
	{$\textbf{Post} \equiv \{t.indiceNat \igobs$  \IF $def?(clave,t_0.indicesNat)$ THEN $definir(Ag(reg,regAnteriores),t_0.indicesNat)$ ELSE $definir(\{reg\},t_0.indicesNat)$ FI $\}$}\\
	\\ \\
	{$clave \equiv valorNat(obtener(siguiente(itreg),t_0.indice.iNat))$}\\
	{$reg \equiv itreg$} \\
	{$regAnteriores \equiv obtener(t_0.indicesNat,clave))$}
\end{algorithm}

\begin{algorithm}[H]{\textbf{iAgregarIndiceString}(\Inout{t}{tabla},\In{itreg}{itConj(registro)})}
	{\\ $\textbf{Pre} \equiv \{t \igobs t_0 \land \neg vacio?(t.indices.iStr)\}$}
	\begin{algorithmic}[1]
		\State $String: clave \gets valorStr(obtener(siguiente(itreg),t.indices.iString))$	\Comment por campos acotados $\Theta(1)$
		\If{$definido?(t.indiceStr,clave)$}	\Comment O(L)
			\State $definir(t.indiceStr,clave,AgregarRapido(obtener(t.indiceStr,clave),itreg))$ \Comment O(L)
		\Else
			\State $definir(t.indiceStr,clave,\{itreg\})$ \Comment O(L)
		\EndIf
		\medskip
		\Statex \underline{Complejidad:} O(L)
		\Statex \underline{Justificación:} Busca en el diccionario del indice y lo redefine; ambas operaciones el diccionario las realiza en O(L); entonces la complejidad temporal O(L + L), quedando O(L) por algebra de ordenes. Se usa el agregarRapido del conjunto lineal para agregar el iterador en $O(1)$.
	\end{algorithmic}
	{$\textbf{Post} \equiv \{t.indiceStr \igobs$  \IF $def?(clave,t_0.indicesStr)$ THEN $definir(Ag(reg,regAnteriores),t_0.indicesStr)$ ELSE $definir(\{reg\},t_0.indicesStr)$ FI $\}$}\\
	\\ \\
	{$clave \equiv valorStr(obtener(siguiente(itreg),t_0.indice.iStr))$}\\
	{$reg \equiv itreg$} \\
	{$regAnteriores \equiv obtener(t_0.indicesStr,clave))$}
\end{algorithm}

\begin{algorithm}[H]{\textbf{iBorrarIndices}(\Inout{t}{tabla},\In{crit}{registro})}
	{\\ $\textbf{Pre} \equiv \{t \igobs t_0\}$}
	\begin{algorithmic}[1]

		\If{$\neg esVacio(t.indice.iNat)$}	\Comment O(1)
			\State $eliminar(t.indiceNat,valorNat(obtener(crit,t.indices.iNat)))$ \Comment $O(log (\#registros))$
		\EndIf

		\If{$\neg esVacio(t.indice.iStr)$}	\Comment O(1)
			\State $eliminar(t.indiceStr,valorStr(obtener(crit,t.indices.iStr)))$ \Comment O(L)
		\EndIf
		\medskip
		\Statex \underline{Complejidad:} $O(L + log (\#registros))$ peor caso
		\Statex \underline{Justificación:} Suma de las complejidades para eliminar dadas por los diccionarios usados en los indices. O(L) si solo indice String, O(log(\#registros)) si solo indice Nat, y O(L + log (\#registros)) si tiene ambos indices; O(1) si no hay ninguno.
	\end{algorithmic}
	{$\textbf{Post} \equiv \{(\neg vacio?(t.indices.iStr) \Rightarrow_L t.indiceStr \igobs eliminar(t_0.indiceStr,claveStr)) \land \\
							 (\neg vacio?(t.indices.iNat) \Rightarrow_L t.indiceNat \igobs eliminar(t_0.indiceNat,claveNat)) \}$}
	\\ \\
	{$claveStr \equiv valorStr(obtener(t_0.indices.iStr,crit))$}\\
	{$claveNat \equiv valorStr(obtener(t_0.indices.iNat,crit))$}

\end{algorithm}

\begin{algorithm}[H]{\textbf{iCombinarRegistros}(\In{c}{campo},\In{t}{tabla},\In{cr}{conj(registro)}) $\to$ res:conj(registro)}
	\begin{algorithmic}[1]
		\If{c = t.indices.iNat} \Comment $\Theta(1)$
			\State res $\gets$ combinarRegistrosIndiceNat(c,t,cr) \Comment $\Theta$($\#$cr) * (O(log $\#$registros(t)) + $\Theta$(L))
		\Else
			\If{c = t.indices.iStr} \Comment $\Theta(1)$
				\State res $\gets$ combinarRegistrosIndiceStr(c,t,cr) \Comment $\Theta(\#cr * L)$	
			\Else
				\State res $\gets$ combinarRegistrosSinIndice(c,t,cr)  \Comment $\Theta(\#registros(t)$ * $\#cr * L)$
			\EndIf
		\EndIf

		\medskip
		\Statex \underline{Complejidad:} 
		\begin{itemize}
		\item $\Theta$($\#$cr) * (O(log $\#$registros(t)) + $\Theta$(L)) con indice nat.
		\item $\Theta(\#reg2$ * $L)$	con indice string.
		\item $\Theta(\#registros(t)$ * $\#reg2$ * $L)$	sin indices.
		\end{itemize}
		\Statex \underline{Justificación:} La complejidad esta justificada en las funciones auxiliares.
	\end{algorithmic}
\end{algorithm}

\begin{algorithm}[H]{\textbf{iCombinarRegistrosSinIndice}(\In{c}{campo},\In{t}{tabla},\In{reg2}{conj(registro)}) $\to$ res:conj(registro)}
	{\\ $\textbf{Pre} \equiv \{enTodos(c,registros(t)) \land enTodos(c,cr)\}$}
	\begin{algorithmic}[1]
		\State $res \gets \emptyset$ \Comment $\Theta(1)$
		\State $itConj(registro): itreg1 \gets CrearIt(registros(t1))$ \Comment	$\Theta(1)$
		\While{$haySiguiente?(itreg1)$} \Comment $\Theta(\#reg1)$
			\State $itConj(registro): itreg2 \gets CrearIt(registros(t2))$ \Comment	$\Theta(1)$
			\While{$haySiguiente?(itreg2)$} \Comment $\Theta(\#reg2)$
				\If{$obtener(siguiente(itreg1),c) = obtener(siguiente(itreg2),c)$} \Comment $\Theta(L)$
					\State $registro: reg \gets copiar(siguiente(itreg1))$ \Comment $\Theta(L)$
					\State $agregarCampos(reg,siguiente(itreg2))$	\Comment $\Theta(L)$
					\State $agregarRapido(res,reg)$	\Comment $\Theta(L)$
				\EndIf
				\State $avanzar(itreg2)$ \Comment $\Theta(1)$
			\EndWhile
			\State $avanzar(itreg1)$	\Comment $\Theta(1)$
		\EndWhile
		\medskip
		\Statex \underline{Complejidad:} $\Theta(\#registros(t) *$  $\#reg2$ $* L)$
		\Statex \underline{Justificación:} A cada registro de reg1 lo compara con todos los de reg2 por medio del campo join el cual es un string no acotado que tiene como costo de comparacion la longitud de el string, para los calculos tomamos L como el mas largo de los string comparados. A esto se le suma la copia de el registro $O(L)$ y agregar campos y la copia al conjunto. Esto en total da $O(\#reg) * O(\#reg1) * (O(L) + O(L) + O(L) + O(L)))$ que por algebra de ordenes $O(\#registros(t) *$ $\#reg2$ $* L)$
	\end{algorithmic}
	{$\textbf{Post} \equiv \{res \igobs combinarRegistros(c,registros(t),cr)\}$}
\end{algorithm}

\begin{algorithm}[H]{\textbf{iCombinarRegistrosIndiceNat}(\In{c}{campo},\In{t}{tabla},\In{cr}{conj(registro)}) $\to$ res:conj(registro)}
	{\\ $\textbf{Pre} \equiv \{enTodos(c,registros(t)) \land enTodos(c,cr)\}$}
	\begin{algorithmic}[1]
		\State $res \gets \emptyset$ \Comment $\Theta(1)$
		\State $itConj(registro): itreg1 \gets CrearIt(cr)$ \Comment	$\Theta(1)$
		\While{$haySiguiente?(itreg1)$} \Comment $\Theta(\#registros(t))$
			\State $Nat: filtro \gets valorNat(obtener(siguiente(itreg1),c))$ \Comment $\Theta(L)$
			\If{$definido?(filtro,t.indicesNat)$} \Comment $O(log \#registros(t))$
				\State $registro: reg \gets copiar(siguiente(itreg1))$ \Comment $\Theta(L)$
				\State $itconj(registro): itreg2 \gets siguiente(CrearIt(obtener(t.indiceNat,filtro)))$ \Comment el conjunto de it reg tiene un solo elemento porque el campo es clave $\Theta(log \#registros(t))$
				\State $agregarCampos(reg,siguiente(itreg2))$	\Comment $\Theta(L)$
				\State $agregarRapido(res,reg)$	\Comment $\Theta(L)$
			\EndIf
			\State $avanzar(itreg1)$	\Comment $\Theta(1)$
		\EndWhile
		\medskip
		\Statex \underline{Complejidad:} $\Theta$($\#$cr) * (O(log $\#$registros(t)) + $\Theta$(L))
		\Statex \underline{Justificación:} A cada registro de cr lo buscar t por medio del campo join que en este caso es O(log $\#registros(t)$). A esto se le suma la copia de el registro $O(L)$ y agregar campos y la copia al conjunto. Esto en total da $O(\#reg)$ * (O(log $\#$registros(t) + O(L*4)) que por algebra de ordenes $O$($\#$cr * (O(log $\#$registros(t) + $O(L))$)
	\end{algorithmic}
	{$\textbf{Post} \equiv \{res \igobs combinarRegistros(c,registros(t),cr)\}$}
\end{algorithm}

\begin{algorithm}[H]{\textbf{iCombinarRegistrosIndiceStr}(\In{c}{campo},\In{t}{tabla},\In{cr}{conj(registro)}) $\to$ res:conj(registro)}
	{\\ $\textbf{Pre} \equiv \{enTodos(c,registros(t)) \land enTodos(c,cr)\}$}
	\begin{algorithmic}[1]
		\State $res \gets \emptyset$ \Comment $\Theta(1)$
		\State $itConj(registro): itreg1 \gets CrearIt(cr)$ \Comment	$\Theta(1)$
		\While{$haySiguiente?(itreg1)$} \Comment $\Theta(\#registros(t))$
			\State $String: filtro \gets valorString(obtener(siguiente(itreg1),c))$ \Comment $O(L)$
			\If{$definido?(filtro,t.indicesString)$} \Comment $\Theta(L)$
				\State $registro: reg \gets copiar(siguiente(itreg1))$ \Comment $\Theta(L)$
				\State $itconj(registro): itreg2 \gets siguiente(CrearIt(obtener(t.indiceString,filtro)))$ \Comment el conjunto de it reg tiene un solo elemento porque el campo es clave $\Theta(L)$
				\State $agregarCampos(reg,siguiente(itreg2))$	\Comment $\Theta(L)$
				\State $agregarRapido(res,reg)$	\Comment $\Theta(L)$
			\EndIf
			\State $avanzar(itreg1)$	\Comment $\Theta(1)$
		\EndWhile
		\medskip
		\Statex \underline{Complejidad:} $\Theta$($\#$cr * L))
		\Statex \underline{Justificación:} A cada registro de cr lo buscar t por medio del campo join que en este caso es O(L). A esto se le suma la copia de el registro $O(L)$ y agregar campos y la copia al conjunto. Esto en total da $O(\#cr) * O(L*5))$ que por algebra de ordenes $O(\#cr * O(L))$
	\end{algorithmic}
	{$\textbf{Post} \equiv \{res \igobs combinarRegistros(c,registros(t),cr)\}$}
\end{algorithm}


\begin{algorithm}[H]{\textbf{iCoincidencias}(\In{crit}{registro},\In{t}{tabla}) $\to$ res:conj(registro)}
	\begin{algorithmic}[1]
		\If{definido?(crit,t.indices.iNat)} \Comment $\Theta(1)$
			\State res $\gets$ coincidenciasIndiceNat(crit,t) \Comment $O(log ($\#$registros(t)) + L * \#($reg mismo indice$))$
		\Else
			\If{definido?(crit,t.indices.iStr)} \Comment $\Theta(1)$
				\State res $\gets$ coincidenciasIndiceStr(crit,t) \Comment $O(L * \#($reg mismo indice$))$	
			\Else
				\State res $\gets$ coincidenciasSinIndice(crit,t)  \Comment $O(\#registros(t) * L)$
			\EndIf
		\EndIf

		\medskip
		\Statex \underline{Complejidad:} 
		\begin{itemize}
		\item $O(log ($\#$registros(t)) + L * \#($reg mismo indice$))$ con indice nat.
		\item $O(L * \#($reg mismo indice$))$	con indice string.
		\item $O(\#registros(t) * L)$	sin indices.
		\end{itemize}
		\Statex \underline{Justificación:} La complejidad esta justificada en las funciones auxiliares.
	\end{algorithmic}
\end{algorithm}

\begin{algorithm}[H]{\textbf{iCoincidenciasSinIndice}(\In{crit}{registro},\In{t}{tabla}) $\to$ res:conj(registro)}
	{\\ $\textbf{Pre} \equiv \{campos(crit) \subseteq campos(t)\}$}
	\begin{algorithmic}[1]
		\State $res \gets \emptyset$ \Comment $\Theta(1)$
		\State $itConj(registro): itreg \gets CrearIt(registros(t))$ \Comment	$\Theta(1)$
		\While{$haySiguiente?(itreg)$} \Comment $\Theta(\#registros(t))$
			\If{$coincidenTodos(crit,campos(crit),siguiente(itreg))$}	\Comment $\Theta(L)$
				\State $agregarRapido(res,siguiente(itreg))$	\Comment $\Theta(L)$
			\EndIf
			\State $avanzar(itreg)$	\Comment $\Theta(1)$
		\EndWhile
		\medskip
		\Statex \underline{Complejidad:} $O(\#registros(t) * L)$
		\Statex \underline{Justificación:} A cada registro de reg1 con los valores de los campos de el criterio el cual es un string no acotado que tiene como costo de comparacion la longitud de el string, para los calculos tomamos L como el mas largo de los string comparados. A esto se le suma la copia de el registro $O(L)$ al conjunto. Esto en total da $O(\#registros(t) * O(L)$ que por algebra de ordenes $O(\#registros(t) * L)$
	\end{algorithmic}
	{$\textbf{Post} \equiv \{res \igobs coincidencias(crit,registros(t))\}$}
\end{algorithm}

\begin{algorithm}[H]{\textbf{iCoincidenciaIndiceNat}(\In{crit}{registro},\In{t}{tabla}) $\to$ res:conj(registro)}
	{\\ $\textbf{Pre} \equiv \{campos(crit) \subseteq campos(t)\}$}
	\begin{algorithmic}[1]
		\State $res \gets \emptyset$ \Comment $\Theta(1)$
		\State $Nat: filtro \gets valorNat(obtener(crit,t.indices.iNat))$ \Comment $\Theta(1)$
		\If{$definido?(filtro,t.indicesNat)$} \Comment $O(log \#registros(t))$
			\State $itConj(itConj(registro)): itreg \gets CrearIt(obtener(filtro,t.indicesNat))$ \Comment	$\Theta(1)$
			\While{$haySiguiente?(itreg)$} \Comment $O(\#($reg mismo indice$))$
				\If{$coincidenTodos(crit,campos(crit),siguiente(siguiente(itreg)))$}	\Comment $\Theta(L)$
					\State $agregarRapido(res,siguiente(siguiente(itreg)))$	\Comment $\Theta(L)$
				\EndIf
				\State $avanzar(itreg)$	\Comment $\Theta(1)$
			\EndWhile
		\EndIf
		\medskip
		\Statex \underline{Complejidad:} $O(log ($\#$registros(t)) + L * \#($reg mismo indice$))$
		\Statex \underline{Justificación:} Busca en criterio si esta definido el valor indice que es O(log $\#registros(t)$). A esto se le suma la copia de el registro $O(L)$. Si el campo no es clave podria haber mas de un registros con el mismo indice. Esto en total da $O(log $\#$registros(t))$ + $O(L*\#(reg mismo indice))$) que por algebra de ordenes $O(log ($\#$registros(t)) + L * \#(reg mismo indice))$
	\end{algorithmic}
	{$\textbf{Post} \equiv \{res \igobs coincidencias(crit,registros(t))\}$}
\end{algorithm}

\begin{algorithm}[H]{\textbf{iCoincidenciaIndiceStr}(\In{crit}{registro},\In{t}{tabla}) $\to$ res:conj(registro)}
	{\\ $\textbf{Pre} \equiv \{campos(crit) \subseteq campos(t)\}$}
	\begin{algorithmic}[1]
		\State $res \gets \emptyset$ \Comment $\Theta(1)$
		\State $String: filtro \gets valorString(obtener(crit,t.indices.iString))$ \Comment $\Theta(L)$
		\If{$definido?(filtro,t.indicesString)$} \Comment $\Theta(L)$
			\State $itConj(itConj(registro)): itreg \gets CrearIt(obtener(filtro,t.indicesString))$ \Comment	$\Theta(1)$
			\While{$haySiguiente?(itreg)$} \Comment $O(\#($registros con mismo valor de indice$))$
				\If{$coincidenTodos(crit,campos(crit),siguiente(siguiente(itreg)))$}	\Comment $\Theta(L)$
					\State $agregarRapido(res,siguiente(siguiente(itreg)))$	\Comment $\Theta(L)$
				\EndIf
				\State $avanzar(itreg)$	\Comment $\Theta(1)$
			\EndWhile
		\EndIf
		\medskip
		\Statex \underline{Complejidad:} $O(L * \#($reg mismo indice$))$
		\Statex \underline{Justificación:} Busca en criterio si esta definido el valor indice que es O(L). A esto se le suma la copia de el registro $O(L)$. Si el campo no es clave podria haber mas de un registros con el mismo indice. Esto en total da $O(L)$ + $O(L*\#(reg mismo indice))$) que por algebra de ordenes $O(L * \#($reg mismo indice$))$
	\end{algorithmic}
	{$\textbf{Post} \equiv \{res \igobs coincidencias(crit,registros(t))\}$}
\end{algorithm}

\pagebreak
\section{M\'odulo Base De Datos}

\subsection{Interfaz}

\textbf{se explica con}: \tadNombre{BaseDeDatos}, \tadNombre{IteradorUnidireccional($\alpha$)}.

\textbf{géneros}: base


~


\subsubsection{Operaciones b\'asicas de BaseDeDatos}

\begin{Interfaz}

\InterfazFuncion{nuevaDB}{}{base}
[true]
{$res \igobs$ nuevaDB()}
[O(1)]
[Crea una nueva base vacia.]
[]

~

\InterfazFuncion{agregarTabla}{\In{ta}{tabla}, \Inout{db}{base} }{bool}
[$db \igobs db_0$ $\land$ $\emptyset$?(registros($ta$))]
{res $\igobs$ nombre($ta$) $\notin$ tablas($db_0$) $\yluego$ \IF res THEN $db \igobs $ agregarTabla($ta$, $db_o$) ELSE $db \igobs db_0$  FI    }
[O(1)]
[Agrega una nueva tabla a la base si el nombre de la tabla no existe ya en la base, si ya existe devuelve false.]
[La tabla se copia dentro de la estructura de la base.]

~

\InterfazFuncion{insertarEntrada}{\In{reg}{registro}, \In{t}{string}, \Inout{db}{base} }{}
[$db \igobs db_o$ $\land$ $t \in$ tablas($db$) $\yluego$ puedoInsertar?($reg$, $t$)]
{$db \igobs $ insertarEntrada($reg$, $t$, $db_o$)}
[ O(T * L + in) donde T es la cantidad de tablas, L el maximo largo de un string en un registro y in = O(log(n)) en promedio si tiene indice en campo nat y O(1) sino, donde n es la cantidad de registros de la tabla. ]
[Agrega un registro a la tabla $t$.]
[El registro se copia en la estructura de la base.]

~

\InterfazFuncion{borrar}{\In{cr}{registro}, \In{t}{string}, \Inout{bd}{base} }{}
[$bd \igobs bd_o$ $\land$ \#campos($cr$)$=1$ $\land$ $t \in$ tablas($db$)]
{$bd \igobs $borrar($cr$, $t$, $bd_o$)}
[O(T * L + in)  donde T es la cantidad de tablas, L el maximo largo de un string en un registro y in = O(log(n)) en promedio si el criterio tiene indice Nat, O(L) si el criterio tiene indice string y O(L * n) si el criterio no es indice.]
[Borra un registro a la tabla $t$.]
[]

~

\InterfazFuncion{generarVistaJoin}{\In{t1}{string}, \In{t2}{string}, \In{c}{campo}, \Inout{db}{base} }{itConj(registro)}
[$bd \igobs bd_o$ $\land$ $t1 \neq t2 \land \{t1,t2\} \subseteq$ tablas($db$) $\yluego$ $c \in$ claves(dameTabla($t1$,$db$)) $\land$ \\ $\land$ $c \in $ claves(dameTabla($t2$, $db$)) $\land$ $\neg$hayJoin?($t1$,$t2$,$db$)]
{$bd = $ generarVistaJoin($t1$, $t2$, $c$, $bd_o$) $\land$ $res$ = CrearItUni(vistaJoin($bd$)) $\land$ alias(esPermutacion?(SecuSuby($res$), vistaJoin($bd$))}
[Si c no es indice en T1, O(n*m*L), sino O(n+m) * O(L + in) donde in es O(log(n+m)) si c es indice en nat y O(1) es es indice en str. (donde n es \#(registros(obtener(t1,db))), m es \#(registros(obtener(t2,db))) y L el string mas largo en alguno de los registros.)]
[Genera un join de la tabla $t1$ con la tabla $t2$ uniendo sus registros donde coincide el campo clave $c$ y devuelve un iterador a los registros.]
[El iterador es no modificable y se invalida al insertar o borrar registros de alguna de las tablas del join.]

~

\InterfazFuncion{tablas}{\In{bd}{base}}{itConj(string)}
[true]
{$res$ = CrearItUni(tablas($bd$)) $\land$ alias(esPermutacion?(SecuSuby($res$), tablas($bd$))}
[O(1)]
[Devuelve un iterador del conjunto de nombres de las tablas de la base de datos.]
[El iterador es no modificable y se invalida al agregar una nueva tabla.]

~

\InterfazFuncion{dameTabla}{\In{t}{string}, \In{db}{base} }{tabla}
[$t \in$ tablas($db$)]
{$res \igobs$ dameTabla($t$, $db$)}
[O(1)]
[Devuelve la tabla del registro que tiene el nombre $t$ pasado como parametro.]
[La tabla se devuelve por referencia no modificable.]

~

\InterfazFuncion{hayJoin?}{\In{t1}{string}, \In{t2}{string}, \In{db}{base} }{bool}
[$t1 \neq t2 \land \{t1, t2\} \subseteq$ tablas($db$)]
{$res \igobs$ hayJoin?($t1$, $t2$, $db$)}
[O(1)]
[Devuelve true si la tabla $t1$ posee un join con la tabla $t2$.]
[]

~

\InterfazFuncion{campoJoin}{\In{t1}{string}, \In{t2}{string}, \In{db}{base} }{campo}
[hayJoin?($t1$, $t2$, $db$)]
{$res \igobs$ campoJoin($t1$, $t2$, $db$)}
[O(1)]
[Devuelve el campo del join entre las tablas $t1$ y $t2$.]
[Se devuelve una referencia no modificable.]

~

\InterfazFuncion{registros}{\In{t}{string}, \In{db}{base} }{registros}
[$t \in$ tablas($db$)]
{$res$ = CrearItUni(registros($t$,$bd$)) $\land$ alias(esPermutacion?(SecuSuby($res$), registros($t$,$bd$))}
[O(1)]
[Devuelve el conjunto de registros de la tabla t.]
[Devuelve una referencia no modificable.]

~

\InterfazFuncion{vistaJoin}{\In{t1}{string}, \In{t2}{string}, \In{db}{base} }{itConj(registros)}
[hayJoin?($t1$, $t2$, $db$)]
{$res$ = CrearItUni(vistaJoin($t1$, $t2$, $db$)) $\land$ alias(esPermutacion?(SecuSuby($res$), vistaJoin($t1$, $t2$, $db$))}
[$\Theta$(R) * O(L + log(n+m)) si c es indice en t1 y t2, $\Theta$(R) * O(L * n * m) sino, si R=0 es O(1), donde R es la cantidad de cambios desde la ultima vista o generacion del join, n y m la cantidad de registros de las tablas t1 y t2 y L el string mas largo en los registros de estas tablas]
[Devuelve un iterador a los registros del join ya generado entre $t1$ y $t2$.]
[El iterador es no modificable y se invalida al insertar o borrar registros de alguna de las tablas del join.]

~

\InterfazFuncion{cantidadDeAccesos}{\In{t}{string}, \In{db}{base} }{nat}
[$t \in$ tablas($db$)]
{$res \igobs$ cantidadDeAccesos($t$,$db$)}
[O(1)]
[Devuelve la cantidad de modificaciones que tuvo la tabla con nombre $t$.]
[]

~

\InterfazFuncion{tablaMaxima}{\In{db}{base}}{string}
[tablas($db$) $\neq \emptyset$]
{$res \igobs$ tablaMaxima($db$)}
[O(1)]
[Devuelve el nombre de la tabla, o el de una de las tablas, que tuvo mayor cantidad de modificaciones en la base.]
[Devuelve una referencia no modificable]

~

\InterfazFuncion{buscar}{\In{criterio}{registro}, \In{t}{string}, \In{db}{base} }{conj(registro)}
[$t \in$ tablas($db$)]
{$res$ = CrearItUni(buscar($criterio$, $t$, $db$)) $\land$ alias(esPermutacion?(SecuSuby($res$), buscar($criterio$, $t$, $db$))}
[O($L+log(n)$) en promedio si alguno de los criterios de búsqueda $r$ es un campo clave con índice en la tabla $t$.
 O(L) en promedio si hay indice string y $O(\#registros(t) * L)$ si no hay indices.]
[Devuelve el conjunto de los registros que tiene la tabla $t$ que coinciden con el criterio de busqueda.]
[Se devuelve un conjunto de registros por copia]

~

\InterfazFuncion{borrarJoin}{\In{t1}{string}, \In{t2}{string}, \Inout{db}{base} }{}
[$bd \igobs bd_o$ $\land$ hayJoin?($t1$, $t2$, $db$)]
{$bd = $borrarJoin($t1$, $t2$, $c$, $bd_o$)}
[O(1)]
[Borra el join que tenia la tabla $t1$ con la tabla $t2$.]
[]

\end{Interfaz}

~

\pagebreak
\subsubsection{Representaci\'on de BaseDeDatos}

La estructura de base esta pensada para mantener organizada y accesible toda la informacion de la base de datos, esto se realiza utilizando al modulo tabla para organizar los registros, y estructuras auxiliares para mantener los joins entre tablas, y los cambios realizados en las tablas luego de que un join se haya creado. Por otro lado tambien se mantiene informacion para acceder a la tabla mas modificada.


\begin{Estructura}{BaseDeDatos}[estr]
	\begin{Tupla}[estr]
		\tupItem{tablas}{diccString(infoTabla)}
		\tupItem{\\ tablaMaxima}{string}
		\tupItem{\\ maxModificaciones}{nat}
	\end{Tupla}

	~

	\begin{Tupla}[infoTabla]
		\tupItem{tabla}{tabla}%
		\tupItem{\\ joinsCon}{diccString(infoJoin)}%
		\tupItem{\\ esJoinDe}{diccString(puntero(infoJoin))}
	\end{Tupla}

	~

	\begin{Tupla}[infoJoin]
		\tupItem{campo}{string}%
		\tupItem{\\ join}{tabla}%
		\tupItem{\\ cambiosT1}{lista(cambio)}
		\tupItem{\\ cambiosT2}{lista(cambio)}
	\end{Tupla}

~

	\begin{Tupla}[cambio]
		\tupItem{borrado?}{bool}%
		\tupItem{reg}{registro}%
	\end{Tupla}



\end{Estructura}

~

\subsubsection{Invariante de Representaci\'on}


\begin{enumerate}
	\item TablaMaxima es un string vacio si no hay tablas, sino es el nombre de la tabla con mas accessos y maxModificaciones es 0 si no hay tablas, sino es el numero de modificaciones de la tablaMaxima
	
	\item Las claves del diccionario joinsCon y las claves de esJoinDe en la tupla de los significados del diccionario tablas, son un subconjunto de los nombre de tabla, y donde no puede estar el nombre de tabla por el que se accedio al significado.
	
	\item El nombre de la tabla en infoTabla es igual a la clave por la cual se accedio a infoTabla en el diccionario tablas.
	
	\item En esJoinDe hay un puntero al infoJoin que se accede cambiado el orden de t1 y t2 al acceder a los joins
	
	\item En el significado infoJoin, campo, pertenece a las claves de la tabla join y a las claves de la tabla con la que se hizo el join, y en las dos tablas tienen el mismo tipo
	
	\item Todos los registros de la lista cambiosT1 en todos los significados de joinsCon en infoTabla tienen los mismos tipos que las columnas de tabla.

	\item Todos los registros de la lista cambiosT2 en todos los significados de joinsCon en infoTabla tienen los mismos tipos que las columnas de la tabla con la que se hizo el join.
		
		\item  join es una tabla en la cual las columnas son la union de todas las columnas de tabla y las columnas de la tabla con la que se hizo el join que no coinciden con las de tabla.
		
	\item En la lista de cambiosT1 y cambiosT2 estan las modificaciones que ocurrieron en las tablas del join desde que se genero. Por lo tanto (en el ultimo estado para el campo clave en la lista) si el cambio es un borrado se encuentra en la tabla join (ya unidos segun corresponde con los registros de la otra tabla y siempre y cuando existiese coincidencia en el campo clave) y no en la tabla original, y si es una insercion se encuentra en la tabla original y no en el join.
		
\end{enumerate}





\pagebreak


\Rep[estr][e]{
	(1) \IF vacio?(e.tablas) 
	THEN 
	  largo(e.tablaMaxima) = 0 
	  e.maxModificaciones = 0 
	ELSE 
	$(\forall t1:\text{string})$ ( def?(t1, e.tablas) $\land$ ($\forall$ t2:string) def?(t2, e.tablas) $\yluego$ 
	cantidadDeAccesos(obtener(t1, e.tablas)) $\geq$ cantidadDeAccesos(obtener(t2, e.tablas)) ) 
   $\impluego$ \\ e.tablaMaxima = t1  $\land$ e.maxModificaciones = cantidadDeAccesos(obtener(t1, e.tablas))
	FI $\land$ 
	
	~
	
	(2) $(\forall t:\text{string}) def?(t, e.tablas) \impluego $  \\
	( $  claves(obtener(t, e.tablas).joinsCon) \subseteq claves(e.tablas) - \lbrace t \rbrace $ $\land$ \\ 
      $  claves(obtener(t, e.tablas).esJoinDe) \subseteq claves(e.tablas) - \lbrace t \rbrace $ $\land$ \\ 
	 (3) $  nombre(obtener(t, e.tablas).tabla) = t $ ) $\yluego$ \\ 
	~
	
	(4) $(\forall t1,t2:\text{string}) def?(t1, e.tablas) \yluego def?(t2, obtener(t1, e.tablas).joinsCon) \impluego $ \\
	def?(t1, obtener(t2, e.tablas).esJoinDe) $\land$ \&(obtener(t2, obtener(t1, e.tablas).joinsCon) = obtener(t1, obtener(t2, e.tablas).esJoinDe) $\yluego$ \\ 
	~

	$(\forall t1,t2:\text{string}) def?(t1, e.tablas) \yluego def?(t2, obtener(t1, e.tablas).joinsCon) \impluego $ \\
	JoinValido(e, t1, t2, obtener(t1, e.tablas), obtener(t2, e.tablas), obtener(t2, obtener(t1, e.tablas).joinsCon))


}\mbox{}

~

  \tadOperacion{JoinValido}{estr/e,string/n1,string/n2,tabla/t1,tabla/t2,infoJoin/j}{bool}{}
  \tadAxioma{JoinValido(e,n1,n2,t1,t2,j)}{
  
	(5) (j.campo $\in$ claves(t1) $\cap$ claves(t2) $\yluego$  \\
   tipoCampo(j.campo, t1) = tipoCampo(j.campo,t2)) $\land$ \\
~   
   
	(6) $(\forall tr:\text{tupla(bool,reg:registro)}) tr \in j.cambiosT1 \impluego  campos(tr.reg) = campos(t1) \land mismosTipos(tr.reg, t1) \land$ \\
~   
 
	(7) $(\forall tr:\text{tupla(bool,reg:registro)}) tr \in j.cambiosT2 \impluego  campos(tr.reg) = campos(t2) \land mismosTipos(tr.reg, t2) \land$ \\   
~   
 
	(8) (campos(t1) $\cup$ (campos(t2) - campos(t1))) = campos(j.join) $\yluego$ \\
	    $(\forall c:\text{campo})$ c $\in$ campos(j.join) $\impluego$  \\
	    (c $\in$ campos(t1) $\yluego$ tipoCampo(c, j.join) = tipoCampo(c, t1))  $\oluego$ 
	    tipoCampo(c, j.join) = tipoCampo(c, t2) $\land$ \\
  ~ 
  
	(9) $(\forall c:\text{cambio})$ c $\in$ UltimosCambios(j.cambiosT1) $\impluego$ (c.borrado? $\yluego$ c.reg $\in$ registros(j.join)) $\land$  ($\lnot$c.borrado? $\yluego$ c.reg $\in$ registros(t1)) $\yluego$ \\
	$(\forall c:\text{cambio})$ c $\in$ UltimosCambios(j.cambiosT2) $\impluego$ (c.borrado? $\yluego$ c.reg $\in$ registros(j.join)) $\land$  ($\lnot$c.borrado? $\yluego$ c.reg $\in$ registros(t2))
	(para cambiosT2, esto es siempre y cuando cambiosT1 no haya modificado ya los campos)
  }

  ~


\tadOperacion{UltimosCambios}{string/clave,secu(cambio)/cambios}{secu(cambio)}{}
\tadAxioma{UltimosCambios(clave,cambios)}{ \IF  vacia?(cambios) THEN <> ELSE {
	\IF HayOtroCambio(clave,prim(cambios), fin(cambios)) THEN fin(cambios) ELSE prim(cambios) $\bullet$ UltimosCambios(fin(cambios))	FI
	} FI
}

\tadOperacion{HayOtroCambio}{string/clave,cambio/c,secu(cambio)/cambios}{bool}{}
\tadAxioma{HayOtroCambio(clave, c, cambios)}{ \IF  vacia?(cambios) THEN false ELSE 
	obtener(clave, c.reg) = obtener(clave, prim(cambios)) $\lor$ HayOtroCambio(clave, c, fin(cambios)) 
  FI
}


\pagebreak

\subsubsection{Funci\'on de Abastracci\'on}


\Abs[estr]{base}[e]{db}{
	e.tablaMaxima $\igobs$ tablaMaxima(db) $\land$ \\

	claves(e.tablas) = tablas(db) $\yluego$ \\

	$(\forall t1:\text{string})$ def?(t1, e.tablas) $\impluego$ ( \\

	  \hspace*{3mm} obtener(t1, e.tablas).tabla $\igobs$ dameTabla(t1,db) $\yluego$ \\

	  \hspace*{3mm} registros(obtener(t1, e.tablas).tabla) $\igobs$ registros(t1, db) $\land$ \\

	  \hspace*{3mm} cantidadDeAccesos(obtener(t1, e.tablas).tabla) $\igobs$ cantidadDeAccesos(t1, db) $\land$ \\
	  
	  \hspace*{3mm} $(\forall r:\text{registro})$ 
	  coincidencias(r, registros(obtener(t1, e.tablas).tabla)) $\igobs$ buscar(r, t1, db) $\land$ \\
	  

 	 \hspace*{3mm} $(\forall t2:\text{string})$ ( \\
	  
	 \hspace*{6mm}  (def?(t2, obtener(t1, e.tablas).joinsCon)) $\iff$ hayJoin?(t1, t2, db)) $\yluego$ \\
	   
	 \hspace*{6mm}  def?(t2, obtener(t1, e.tablas).joinsCon) $\impluego$ \\
	   
	  \hspace*{9mm}   (  obtener(t2, obtener(t1, e.tablas).joinsCon).campo $\igobs$ campoJoin(t1, t2, db) $\land$   \\
	     
	 \hspace*{9mm}    vistaJoin(t1, t2, estr) $\igobs$ vistaJoin(t1, t2, db)
	    )
	    
	   ) 
	 )
	  
}

\pagebreak


\subsection{Algoritmos}

  
\begin{algorithm}[H]{\textbf{iNuevaDB}() $\to$ $res$ : estr}
    	\begin{algorithmic}[1]
			 \State $res \gets \langle Vacio(), <> \rangle$ \Comment $\Theta(1)$
			\medskip
			\Statex \underline{Complejidad:} $\Theta(1)$
    	\end{algorithmic}
\end{algorithm}


\begin{algorithm}[H]{\textbf{iAgregarTabla}(\In{t}{tabla},\Inout{db}{estr}) $\to$ $res$ : bool}
    	\begin{algorithmic}[1]
    		\State $res \gets \lnot def?(nombre(t), db.tablas) $  	\Comment $\Theta(1)$
			 \If{$\lnot def?(nombre(t), db.tablas)$}							\Comment $\Theta(1)$
	   			\State definir(nombre(t), t, db.tablas)           			\Comment $\Theta(1)$
	   		\EndIf
			\medskip
			\Statex \underline{Complejidad:} $\Theta(1)$
			\Statex \underline{Justificacion:} Como el largo del nombre de las tablas es acotado y se utiliza un diccionario que tiene def? y definir, definido en orden del maximo largo de la clave, estos accesos son en O(1), ademas la copia de la tabla, como no tiene registros y tiene un numero acotado de columnas, es O(1)
    	\end{algorithmic}
\end{algorithm}



\begin{algorithm}[H]{\textbf{iInsertarEntrada}(\In{r}{registro},\In{t}{string},\Inout{db}{estr})}
    	\begin{algorithmic}[1]
    		\State tabla t1 $\gets$ obtener(t, db.tablas).tabla         \Comment $\Theta(1)$
    		\State agregarRegistro(r, t1)					\Comment O(L + in) donde in es: O(log(n)) si tiene indice Nat
    		\Statex											\Comment O(L) si tiene indice string
    		\Statex											\Comment O(log(n)+L) si tiene ambos u O(1) sino 
			\State MantenerCambios(obtener(t, db.tablas), r, false)    \Comment O(T * L)
			
			 \If{cantidadDeAccesos(t1) $>$  db.maxModificaciones}	  \Comment $\Theta(1)$
	   			\State db.maxModificaciones $\gets$ cantidadDeAccesos(t1)  \Comment $\Theta(1)$
	   			\State db.tablaMaxima $\gets$ nombre(t1)  \Comment $\Theta(1)$
	   		\EndIf
						
			\medskip
			\Statex \underline{Complejidad:} O(T * L + in) donde T es la cantidad de tablas, L el maximo largo de un string en un registro y in = O(log(n)) en promedio si tiene indice en campo nat y O(1) sino, donde n es la cantidad de registros de la tabla. 
			\Statex \underline{Justificacion:} O(T * L) + O(L + in) = O(T*L + L + in) = O(T*L + in). Las complejidades estan justificadas en la operaciones llamadas. Como in es en peor caso O(L+log(n)), pero O(L) ya esta conciderado fuera de los indices, tomamos que in es O(log(n)) si tiene indice en nat y O(1) sino.
    	\end{algorithmic}
\end{algorithm}



\begin{algorithm}[H]{\textbf{iBorrar}(\In{cr}{registro},\In{t}{string},\Inout{db}{estr})}
    	\begin{algorithmic}[1]
    		\State tabla t1 $\gets$ obtener(t, db.tablas).tabla         \Comment $\Theta(1)$
    		\State borrarRegistro(r, t1)				\Comment in
			\State MantenerCambios(obtener(t, db.tablas), cr, true)   \Comment O(T * L)
			
			 \If{cantidadDeAccesos(t1) $>$ db.maxModificaciones}	  \Comment $\Theta(1)$
	   			\State db.maxModificaciones $\gets$ cantidadDeAccesos(t1)  \Comment $\Theta(1)$
	   			\State db.tablaMaxima $\gets$ nombre(t1)  \Comment $\Theta(1)$
	   		\EndIf
			
			\medskip
			\Statex \underline{Complejidad:} O(T * L + in)  donde T es la cantidad de tablas, L el maximo largo de un string en un registro y in = O(log(n)) en promedio si el criterio tiene indice Nat, O(L) si el criterio tiene indice string y O(L * n) si el criterio no es indice.
			\Statex \underline{Justificacion:} Suma de complejidades de otras operaciones.
    	\end{algorithmic}
\end{algorithm}


\begin{algorithm}[H]{\textbf{iMantenerCambios}(\Inout{it}{infoTabla},\In{r}{registro},\In{borrado?}{bool})}
		{\\ $\textbf{Pre}$ $\equiv$  $it_0 = it$ }
    	\begin{algorithmic}[1]
    		\State itDicc(string, infoJoin) iter $\gets$ CrearIt( it.joinsCon )			\Comment O(1)
			 \While{HaySiguiente(iter)}              						\Comment $\Theta(\#(it.joinsCon)) * O(1)$
				\State infoJoin ij $\gets$ SiguienteSignificado(iter)					\Comment $\Theta(1)$
				\State AgregarAtras(ij.cambiosT1, $\langle$ borrado?, r $\rangle$)	 \Comment $\Theta(copy(r))$
			 	\State $Avanzar(iter)$	                     					 	\Comment $\Theta(1)$
			 \EndWhile

    		\State itDicc(string, puntero(infoJoin)) iter2 $\gets$ CrearIt( it.esJoinDe ) 	\Comment O(1)
			 \While{HaySiguiente(iter2)}              			\Comment $\Theta(\#(it.esJoinDe)) * O(1)$
				\State infoJoin ij $\gets$ *SiguienteSignificado(iter2)		\Comment $\Theta(1)$
				\State AgregarAtras(ij.cambiosT2, $\langle$ borrado?, r $\rangle$)		\Comment $\Theta(copy(r))$
			 	\State $Avanzar(iter2)$	                     					 	\Comment $\Theta(1)$
			 \EndWhile

			\medskip
			\Statex \underline{Complejidad:} O(T * L), donde T es la cantidad de tablas y L el maximo largo de un string en un registro.
			\Statex \underline{Justificacion:} Cada uno de los ciclos se ejecuta a los sumo T veces y la copia de un registro es a lo sumo O(T) ya que la cantidad de campos es acotada, y los campos Nat se copian en O(1), el costo queda determinado por el string mas largo dentro de un registro. Las operaciones de iteradores estan dadas en O(1) y el agregarAtras de lista es en O(1) tambien. Luego, la suma de dos ciclos O(T*L) = O(2*T*L) $\in$ O(T*L)
    	\end{algorithmic}
	   {$\textbf{Post}$ $\equiv$ $(\forall s:\text{string})$  def?(s, $it_0$.joinsCon) $\impluego$ 
	  \\ obtener(s, it.joinsCon).cambiosT1 $\igobs$  (obtener(s, $it_0$.joinsCon).cambiosT1 $\bullet \langle borrado?, r \rangle$) 
	   \\ obtener(s, it.esJoinDe)$\rightarrow$ cambiosT2 $\igobs$  (obtener(s, $it_0$.esJoinDe)$\rightarrow$ cambiosT2 $\bullet \langle borrado?, r \rangle$)     }
\end{algorithm}




\begin{algorithm}[H]{\textbf{iGenerarVistaJoin}(\In{n1}{string},\In{n2}{string},\In{c}{campo},\Inout{db}{estr}) $\to$ $\res$ : itConj(registro) }
    	\begin{algorithmic}[1]
    		\State infoTabla inft1 $\gets$ obtener(n1, db.tablas)     \Comment $\Theta(1)$
    		\State infoTabla inft2 $\gets$ obtener(n2, db.tablas)     \Comment $\Theta(1)$
    		\State tabla t1 $\gets$ inft1.tabla         \Comment $\Theta(1)$
    		\State tabla t2 $\gets$ inft2.tabla         \Comment $\Theta(1)$
    		
    		\State conj(registro) rt2 $\gets$ registros(t2)         \Comment (Referencia) $\Theta(1)$

    		\State conj(registro) regsJoin $\gets$ combinarRegistros(c, t1, rt2)   \Comment $comp^{1}$
    		\State itConj(registro) itrj $\gets$ CrearIt(regsJoin)        \Comment $\Theta(1)$

    		\State tabla join $\gets$ NuevaTabla(vacio, $\lbrace$c$\rbrace$, columnasJoin(t1, t2))	\Comment $\Theta$(1)
    		\State definir(n2,  $\langle$ c, join, Vacio, Vacio $\rangle$, inft1.joinsCon)     \Comment O(1)
			\State join $\gets$ obtener(n2, inft1.joinsCon).join				\Comment O(1)
    		\State indexar(c, join)																\Comment como la tabla esta vacia es O(1) 
    		
    		\State definir(n1, \&join, inft2.esJoinDe)  \Comment $\Theta(1)$
    		

			 \While{HaySiguiente(itrj)}              														\Comment $\Theta(n+m)$
			 	\State agregarRegistro(Siguiente(itrj), join)	\Comment O(L + in) donde in es O(log(n+m)) si el campo de join es Nat y O(1) sino
			 	\State $Avanzar(itrj)$	                     											 	\Comment $\Theta(1)$
			 \EndWhile    	    		

			\State regsJoin $\gets$ NULL    		
			\State res $\gets$ CrearIt(registros(join))							\Comment  O(1)
    		
			\medskip
			\Statex \underline{Complejidad:} Si c no es indice en T1, O(n*m*L), sino O(n+m) * O(L + in) donde in es O(log(n+m)) si c es indice en nat y O(1) es es indice en str.
			\Statex \underline{Justificacion:} Sean n=\#(registros(t1)),  m=\#(registros(t2)), L=el largo maximo de algun string en un registro de t1 o t2. Entonces  $comp^{1}$ es O(n*m*L) si no hay indices sobre las tablas, O(m * (log(n) + L)) si c es indice Nat en la tabla t1 y O(m * L) si c es indice String en la tabla t1. 
			\Statex Luego el ciclo de la linea 13, es O(n+m) * O(L + in), donde in es O(log(n+m)) si c es un campo nat, y O(1) sino.
			\Statex Luego por algebra de ordenes: 
			\Statex Si c es indice nat en T1, O(m * (log(n) + L)) + O(n+m) * O(L + O(log(n+m)) = O(n+m) * O(L + log(n+m)) 
			\Statex Si c es indice str en T1, O(m * L) + O(n + m) * O(L) = O(n + m) * O(L)
			\Statex Si c no es indice,  O(m * n * L) + O(n + m) * O(L) = O(n * m) * O(L)
			
    	\end{algorithmic}
\end{algorithm}




\begin{algorithm}[H]{\textbf{iColumnasJoin}(\In{t1}{tabla},\In{t2}{tabla}) $\to$ $res$ : registro }
		{\\ $\textbf{Pre}$ $\equiv$  true }
    	\begin{algorithmic}[1]
			\State registro res $\gets$ Vacio()								\Comment $\Theta(1)$
			\State itConj(campo) itc1 $\gets$ campos(t1)				\Comment $\Theta(1)$
			\State itConj(campo) itc2 $\gets$ campos(t2)				\Comment $\Theta(1)$
			 \While{HaySiguiente(itc1)}              								\Comment $\Theta(\#(campos(t1))) * \Theta(1)$
				\If{tipoCampo(t1, Siguiente(itc1))}								\Comment $\Theta(1)$
					\State definirRapido(res, siguiente(itc1), datoNat(0))					\Comment $\Theta(copy(k)+copy(s))$
				\Else		
					\State definirRapido(res, siguiente(itc1), datoString(Vacia()))	    \Comment $\Theta(copy(k)+copy(s))$
				\EndIf	
				\State $Avanzar(itc2)$	                     						\Comment $\Theta(1)$			
			\EndWhile

			 \While{HaySiguiente(itc2)}              								\Comment $\Theta(\#(campos(t2))) * \Theta(1)$
					\If{$\lnot$ definido?(r, Siguiente(itc2)) }			 	\Comment $\Theta(\#(claves(r)) * equal(k) )$
						\If{tipoCampo(t2, Siguiente(itc2))}						\Comment $\Theta(1)$
							\State definirRapido(res, siguiente(itc2), datoNat(0))			  \Comment $\Theta(copy(k)+copy(s))$
						\Else
							\State definirRapido(res, siguiente(itc2), datoString(Vacia())) \Comment $\Theta(copy(k)+copy(s))$
						\EndIf	
	   				\EndIf			 
			 	\State $Avanzar(itc2)$	                     											 	\Comment $\Theta(1)$
			 \EndWhile 

			\medskip
			\Statex \underline{Complejidad:}  $\Theta$(1)
			\Statex \underline{Justificacion:} Dado que \#(campos(t1)) y \#(campos(t2)) son la cantidad de columnas que puede tener una registro, y estan acotadas por contexto, y que los nombres de los campos estan acotados y que los significados definidos son un Nat o un string vacio, los ciclos, el definidio? y el definirRapido terminan siendo O(1)
    	\end{algorithmic}
	   {$\textbf{Post}$ $\equiv$  campos(res) $\igobs$ campos(t1) $\cup$ (campos(t2) - campos(t1))  $\yluego$
	   $(\forall c:\text{campo})$  def?(c, res) $\impluego$ \\ \IF def?(c, t1) THEN tipo?(obtener(c,res)) = tipo?(obtener(c, t1)) ELSE tipo?(obtener(c,res)) = tipo?(obtener(c,t2)) FI    }
\end{algorithm}




\begin{algorithm}[H]{\textbf{iBorrarJoin}(\In{n1}{string},\In{n2}{string},\In{db}{estr}) $\to$ $res$ : campo }
    	\begin{algorithmic}[1]
   	  		\State infoTabla inft1 $\gets$ obtener(n1, db.tablas)     \Comment $\Theta(1)$
    		\State infoTabla inft2 $\gets$ obtener(n2, db.tablas)     \Comment $\Theta(1)$
			%\State infoJoin ij $\gets$ obtener(n2, inft1.joinsCon)  \Comment $\Theta(1)$%
			%\State ij $\gets$ NULL \Comment O(1)%
			\State borrar(t2, inft1.joinsCon)    \Comment $\Theta(1)$
			\State borrar(t1, inft2.esJoindDe)  \Comment $\Theta(1)$
			\medskip
			\Statex \underline{Complejidad:} $\Theta$(1) 
    	\end{algorithmic}
\end{algorithm}




\begin{algorithm}[H]{\textbf{iTablas}(\In{db}{estr}) $\to$ $res$ : itConj(string) }
    	\begin{algorithmic}[1]
    		\State res $\gets$ CrearIt(db.tablas)    \Comment $\Theta(1)$
			\medskip
			\Statex \underline{Complejidad:} $\Theta$(1) 
    	\end{algorithmic}
\end{algorithm}

\begin{algorithm}[H]{\textbf{iDameTabla}(\In{t}{string},\In{db}{estr}) $\to$ $res$ : tabla }
    	\begin{algorithmic}[1]
    		\State res $\gets$ obtener(t, db.tablas).tabla    \Comment $\Theta(1)$
			\medskip
			\Statex \underline{Complejidad:} $\Theta$(1) 
    	\end{algorithmic}
\end{algorithm}

\begin{algorithm}[H]{\textbf{iHayJoin?}(\In{t1}{string},\In{t2}{string},\In{db}{estr}) $\to$ $res$ : bool }
    	\begin{algorithmic}[1]
    		\State res $\gets$ def?(t2, obtener(t1, db.tablas).joinsJoin)    \Comment $\Theta(1)$
			\medskip
			\Statex \underline{Complejidad:} $\Theta$(1) 
    	\end{algorithmic}
\end{algorithm}


\begin{algorithm}[H]{\textbf{iCampoJoin}(\In{t1}{string},\In{t2}{string},\In{db}{estr}) $\to$ $res$ : campo }
    	\begin{algorithmic}[1]
    		\State res $\gets$ obtener(t2, obtener(t1, db.tablas).joinsJoin).campo    \Comment $\Theta(1)$
			\medskip
			\Statex \underline{Complejidad:} $\Theta$(1) 
    	\end{algorithmic}
\end{algorithm}


\begin{algorithm}[H]{\textbf{iRegistros}(\In{t}{string},\In{db}{estr}) $\to$ $res$ : registro }
    	\begin{algorithmic}[1]
    		\State res $\gets$ registros(obtener(t, db.tablas).tabla)   \Comment $\Theta(1)$
			\medskip
			\Statex \underline{Complejidad:} $\Theta$(1) 
    	\end{algorithmic}
\end{algorithm}

\begin{algorithm}[H]{\textbf{iVistaJoin}(\In{n1}{string},\In{n2}{string},\In{db}{estr}) $\to$ $res$ : itConj(registro) }
    	\begin{algorithmic}[1]
    	   	\State infoTabla inft1 $\gets$ obtener(n1, db.tablas)     \Comment $\Theta(1)$
    	   	\State infoTabla inft2 $\gets$ obtener(n2, db.tablas)     \Comment $\Theta(1)$
    		\State tabla t1 $\gets$ inft1.tabla         \Comment $\Theta(1)$
    		\State tabla t2 $\gets$ inft2.tabla         \Comment $\Theta(1)$

			\State infoJoin ij $\gets$ obtener(n2, inft1.joinsCon)     \Comment $\Theta(1)$

			\State actualizarCambios(ij.cambiosT1, c, t2, ij.join) 	\Comment $Comp^{1}$
			\State actualizarCambios(ij.cambiosT2, c, t1, ij.join) 	\Comment $Comp^{2}$
			
    		\State res $\gets$ CrearIt(registros(ij.join))                   \Comment $\Theta(1)$
			\medskip
			\Statex \underline{Complejidad:} $\Theta$(R) * O(L + log(n+m)) si c es indice en t1 y t2, $\Theta$(R) * O(L * n * m) sino. Si R=0, $\Theta(1)$
			\Statex \underline{Justifiacion:}  Sea $R_{1}$ = long(ij.cambiosT1), $R_{2}$ = long(ij.cambiosT2), R=$R_{1}$+$R_{2}$,  m=\#registros(t1),  n=\#registros(t2), L=el largo maximo de algun string en un registro de t1,t2 y ij.join y rj=\#registros(ij.join) (por como se genera el join y se lo actualiza, rj = (n + m))
		\Statex $Comp^{1}$ es $\Theta$($R_{1}$) * O(L + log(rj)) si c es indice en t2 y $\Theta$($R_{1}$) * O(L * n) sino.
		\Statex $Comp^{2}$ es $\Theta$($R_{2}$) * O(L + log(rj)) si c es indice en t1 y $\Theta$($R_{2}$) * O(L * m) sino.	
		\Statex La complejidad del algoritmo es la suma de $Comp^{1}$ y $Comp^{2}$, y por simplicidad separemos en dos casos, en los que c es indice en t1 y t2, y en los que c no es indice en ninguna de las 2.
		\Statex Luego si c es indice, $\Theta$($R_{1}$) * O(L + log(rj)) + $\Theta$($R_{2}$) * O(L + log(rj)) = $\Theta$(R) * O(L + log(rj))
		\Statex si c no es indice, $\Theta$($R_{1}$) * O(L * n) + $\Theta$($R_{2}$) * O(L * m) = $\Theta$(R) * O(L * n * m)
		\Statex Si R=0, $Comp^{1}$ y $Comp^{2}$ son $\Theta$(1), y todo el algoritmo son sumas de $\Theta(1)$
    	\end{algorithmic}
\end{algorithm}



\tadOperacion{ActualizarCambios}{secu(cambio)/cambios,string/c,tabla/t,tabla/join}{tabla}{$(\forall ca:\text{cambio})$ esta?(ca, cambios) $\impluego$ 
		\\ c $\in$ campos(ca.reg) $\land$ c $\in$ campos(t) $\land$ campos(ca.reg) $\subseteq$ campos(join) $\land$ campos(t) $\subseteq$  campos(join)}
\tadAxioma{ActualizarCambios(cambios,c,t,join)}{ \IF vacio?(cambios) THEN join ELSE  
	{\IF prim(cambios).borrado? THEN
		ActualizarCambios(fin(cambios), c, t, \\ borrarRegistro( dejarSolo(c, prim(cambios).reg), join))
	ELSE
		{ \IF vacio?(combinarTodos(c, prim(cambios).reg, registros(t)) THEN 
			ActualizarCambios(fin(cambios), c, t,  join )
		ELSE
			ActualizarCambios(fin(cambios), c, t, agregarRegistro( \\ dameUno(combinarTodos(c, prim(cambios).reg, registros(t), join)))
		FI }	
	FI }
FI 
}


\begin{algorithm}[H]{\textbf{iActualizarCambios}(\In{cambios}{lista(cambio)},\In{c}{string},\In{t}{tabla},\Inout{join}{tabla}) }
		{\\ $\textbf{Pre}$ $\equiv$  $join = join_0$ }
    	\begin{algorithmic}[1]
			\State itLista(cambio) it $\gets$ 	CrearIt(cambios)		\Comment O(1)

			 \While{HaySiguiente(it)}              \Comment R = long(cambios), $\Theta(R)$
			 	\State cambio cr $\gets$ Siguiente(it) \Comment $\Theta(1)$

					 \If{cr.borrado?}	  \Comment $\Theta(1)$
					 	\State registro critero $\gets$ Vacio() 			\Comment O(1)
					 	\State definirRapido(criterio, c, significado(cr.reg, c))  \Comment O(L)
	   					\State borrarRegistro(criterio, join) \Comment como c es indice en join, O(L + log(\#(regs(join))))
					\Else 
						\State conj(registro) conjDeUno $\gets$ Vacio() 		\Comment O(1)
						\State AgregarRapido(conjDeuno, cr.reg) 			\Comment O(L)
						\State conj(registro) regsJoin $\gets$ combinarRegistros(c, conjDeUno, t)	\Comment $comp^{1}$
						
						\State itConj(registro) itrj $\gets$ CrearIt(regsJoin)        \Comment $\Theta(1)$

						 \If{HaySiguiente(itrj)}          \Comment (a lo sumo hay un registro combinado) $\Theta(1)$
						 	\State agregarRegistro(Siguiente(itrj), join)	 \Comment O(L + in) donde in es O(\#(regs(join)))) si tiene indice Nat y O(1) sino
						 \EndIf    	    		
    		
	   				\EndIf
	   				
				\State EliminarSiguiente(it)			\Comment O(1)
			 	\State Avanzar(it)                \Comment $\Theta(1)$
			 \EndWhile

			\medskip
			\Statex \underline{Complejidad:} $\Theta$(R) * O(L + log(rj)) si c es indice en t y $\Theta$(R) * O(L * m) sino. Si R=0, $\Theta(1)$
			\Statex \underline{Justificacion:} Sean rj=\#(registros(join)),  m=\#(registros(t)), L=el largo maximo de algun string en un registro de t o join y R = long(cambios).
			\Statex Entonces  $comp^{1}$ es O(m*L) si no hay indices sobre las tablas, O((log(m) + L)) si c es indice Nat en la tabla t y O(L) si c es indice String en la tabla t.
			\Statex Luego el ciclo, tiene una complejidad de $\Theta$(R) * O(L + log(rj)) si c es indice en t y $\Theta$(R) * O(L * m) sino.
			\Statex Si R=0, no se ejecuta el ciclo y todo el algoritmo es $\Theta$(1)
    	\end{algorithmic}
	   {$\textbf{Post}$ $\equiv$  join $\igobs$ ActualizarCambios($cambios, c, t, join_0$)  }    	
\end{algorithm}


\begin{algorithm}[H]{\textbf{iCantidadDeAccesos}(\In{t}{string},\In{db}{estr}) $\to$ $res$ : nat }
    	\begin{algorithmic}[1]
    		\State res $\gets$ cantidadDeAccesos(obtener(t, db.tablas).tabla)   \Comment $\Theta(1)$
			\medskip
			\Statex \underline{Complejidad:} $\Theta$(1) 
    	\end{algorithmic}
\end{algorithm}

\begin{algorithm}[H]{\textbf{iTablaMaxima}(\In{db}{estr}) $\to$ $res$ : string }
    	\begin{algorithmic}[1]
    		\State res $\gets$ db.tablaMaxima  \Comment $\Theta(1)$
			\medskip
			\Statex \underline{Complejidad:} $\Theta$(1) 
    	\end{algorithmic}
\end{algorithm}


\begin{algorithm}[H]{\textbf{iBuscar}(\In{r}{registro},\In{t}{string},\In{db}{estr}) $\to$ $res$ : conj(registro) }
    	\begin{algorithmic}[1]
    		\State conj(registro) regs $\gets$ registros(obtener(t, db.tablas).tabla)  \Comment obtiene una referencia $\Theta$(1) 
    		\State res $\gets$ coincidencias(r, regs)   \Comment $O(log ($\#$registros(t)) + L * \#($reg mismo indice$))$ con indice nat. 
    		 \Statex 		\Comment  $O(L * \#($reg mismo indice$))$	con indice string. 
    		 \Statex		\Comment  $O(\#registros(t) * L)$	sin indices.

			\medskip
			\Statex \underline{Complejidad:} 
			\Statex  O(Log(n) + L) en promedio si hay indice Nat.
			\Statex  O(L) en promedio si hay indice string.
			\Statex  $O(\#registros(t) * L)$	sin indices.
			\Statex \underline{Justificacion:} La complejidad esta dada por el llamado a la funcion coincidencias, se desetima para el calculo en promedio la cantidad de registros donde se repite un registro para un mismo indice, ya que por contexto de uso los nats se insertan con distribucion uniforme.
			
    	\end{algorithmic}
\end{algorithm}




%\pagebreak
%\input{modulo}

\end{document}
