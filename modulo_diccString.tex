\section{Módulo Diccionario String($\alpha$)}

Se representa mediante un árbol n-ario con invariante de trie. Las claves son strings y permite acceder a un significado en tiempo de la longitud de la clave y definir un significado en el mismo tiempo más el tiempo de copy(s) ya que los significados se almacenan por copia.

\subsection{Interfaz}

\textbf{parametros formales}

\textbf{géneros}: \TipoVariable{$\alpha$}.

\textbf{funcion}: \InterfazFuncion{Copiar}{\In{s}{$\alpha$}}{$\alpha$}
				  [true]
				  {res $=_{obs}$ s}
				  [$O(copy(s))$]
				  [funcion de copia de $\alpha$.]
				  []

\textbf{se explica con}: \tadNombre{Diccionario(String,$\alpha$)},\tadNombre{IteradorBidireccional(String)}.

\textbf{géneros}: \TipoVariable{diccString($\alpha$)}.



~

\subsubsection{Operaciones básicas de Diccionario String($\alpha$)}

\InterfazFuncion{CrearDiccionario}{}{}
{res $=_{obs}$ vacío()}
[$O(1)$]
[Crea un diccionario vacío.]
[]

~



\InterfazFuncion{Definido?}{\In{d}{diccString$(\alpha)$}, \In{c}{string})}{bool}
[true]
{$res$ $\igobs$ def?($d$, $c$)}
[$O(long(c))$]
[Devuelve true si la clave está definida en el diccionario y false en caso contrario.]
[]

~

\InterfazFuncion{Definir}{\Inout{d}{diccString$(\alpha)$}, \In{c}{string}, \In{s}{$\alpha$}}{}
[$ d \igobs d_0 $]
{$ d \igobs$ definir($c$, $s$, $d_0$)}
[O(L + copy(s)) donde L es el largo de la clave mas larga definida en el diccionario]
[Define la clave $c$ con el significado $s$.]
[Se almacena una copia de $s$.]

~

\InterfazFuncion{Obtener}{\In{d}{diccString$(\alpha)$}, \In{c}{string}}{$\alpha$}
[def?($c$, $d$)]
{alias($res$ $\igobs$ obtener($c$, $d$))}
[$\Theta(long(c))$]
[Devuelve el significado correspondiente a la clave $c$.]
[Se devuelve por rerefencia el significado almacenado en el diccionario.]

~

\InterfazFuncion{Borrar}{\Inout{d}{diccString($\alpha$)}, \In{c}{string}}{}
[$ d \igobs d_0 \land $ def?($d$, $c$)]
{$ d \igobs$ borrar($d_0$, $c$)}
[O(L) donde L es el largo de la clave mas larga definida en el diccionario]
[Borra la clave $c$ del diccionario y su significado.]
[]

~

\InterfazFuncion{CrearItClaves}{\In{d}{diccString($\alpha$)}}{itConj(String)}
[true]
{res $\igobs$ CrearItBi(claves(d)) $\land$ alias(esPermutacion?(SecuSuby(res), claves(d)))}
[$\Theta(1)$]
[Devuelve un iterador a las claves del diccionario]
[El iterador es no modificable y se invalida luego de eliminar o definir una clave en el diccionario]

~

\InterfazFuncion{ClaveMinima}{\In{d}{diccString($\alpha$)}}{string}
[true]
{\IF \#(claves(d))=0 THEN res = <> ELSE \\
 res $\in$ claves(d) $\yluego$  $(\forall s:\text{string})$ s $\in$ claves(d) $\land$ s $\neq$ res $\implies$ res $<$ s  FI}
[$\Theta(1)$]
[Devuelve la clave minima del diccionario, utilizando el orden lexicografico, o el string vacio si no hay claves definidas]
[Se deuvelve una referencia no modificable a la clave]

~

\InterfazFuncion{ClaveMaxima}{\In{d}{diccString($\alpha$)}}{string}
[true]
{\IF \#(claves(d))=0 THEN res = <> ELSE \\
 res $\in$ claves(d) $\yluego$  $(\forall s:\text{string})$ s $\in$ claves(d) $\land$ s $\neq$ res $\implies$ res $>$ s  FI}
[$\Theta(1)$]
[Devuelve la clave maxima del diccionario, utilizando el orden lexicografico, o el string vacio si no hay claves definidas]
[Se deuvelve una referencia no modificable a la clave]

~



\pagebreak

\subsubsection{Representación}

\begin{Estructura}{ diccString($\alpha$)}[estr]
	\begin{Tupla}[estr]
		\tupItem{raiz}{puntero(Nodo)}
		\tupItem{\\claves}{conj(string)}
		\tupItem{\\min}{string}
		\tupItem{\\max}{string}
	\end{Tupla}

	~

	\begin{Tupla}[Nodo]
		\tupItem{hijos}{arreglo(puntero(Nodo))}%
		\tupItem{\\	info}{puntero($\alpha$)}%
		\tupItem{\\ clave}{itConj(string)}
		\tupItem{\\ padre}{puntero(nodo)}
		\tupItem{\\ idx}{nat}
	\end{Tupla}

\end{Estructura}

\subsubsection{Invariante de Representación}


%\renewcommand{\labelenumi}{(\Roman{enumi})}

\begin{enumerate}
	\item Nodo representa un arbol con invariarnte de trie de 256 posiciones
	\item Min es la minima clave de las claves del diccionario si hay alguna, sino es un string vacio.
	\item Max es la maxima clave de las claves del diccionario si hay alguna, sino es un string vacio.
	\item Idx del nodo es un entero menor que 256
	\item Padre apunta al nodo del cual es hijo en la posicion idx, salvo que sea raiz y en ese caso es NULL 
	\item Desreferenciando el padre de un nodo siempre se llega a la raiz y nunca se repite un nodo.
	\item Un nodo puede tener significado, o representar los caracteres de la clave que conducen a un significado.
	\item Si un nodo tiene significado, el puntero info apunta al significado, sino es NULL
	\item Si un nodo tiene significado, clave es un iterador de las claves del diccionario cuyo siguiente es la clave en la que esta definido.
	\item Todos los elementos del arreglo hijos de un nodo, son NULL o son nodos validos

\end{enumerate}



\Rep[estr][e]{
	\\(1) $\lnot$(e.raiz = NULL) $\yluego$ nodoValido(e.raiz)
	\\(2) \IF \#(e.claves)=0 THEN e.min = <> ELSE  e.min $\in$ e.claves $\yluego$  $(\forall s:\text{string})$ s $\in$ e.claves $\land$ s $\neq$ e.min $\implies$ e.min $<$ s  FI
	\\(3) \IF \#(e.claves)=0 THEN e.max = <> ELSE e.max $\in$ e.claves $\yluego$  $(\forall s:\text{string})$ s $\in$ e.claves $\land$ s $\neq$ e.max $\implies$ e.max $>$ s  FI
}\mbox{}


~
\tadOperacion{nodoValido}{puntero(Nodo)/nodo,estr/e}{bool}{}
\tadAxioma{nodoValido($nodo$, $e$)}{
	(1) tam(nodo$\rightarrow$hijos) = 256 $\yluego$ \\
	(4) $nodo \rightarrow idx$ $<$ 256 $\yluego$ \\ 
	(5) $nodo \rightarrow padre$ = NULL $\iff$ e.raiz = $nodo$ $\yluego$ \\
	(7,8,9)$\lnot$($nodo \rightarrow padre$ = NULL) $\implies$ $nodo \rightarrow padre \rightarrow hijos[ nodo \rightarrow idx ] $ = $nodo$ $\yluego$ \\
	reconstruirClave($nodo$) $\in$ e.claves $\iff$  $ \lnot (nodo \rightarrow  info = NULL)$ $\land$ siguiente(*($nodo \rightarrow clave$)) = reconstruirClave($nodo$)
	\\(10)	($\forall i$ Nat) i < 256 $\yluego$ $\lnot$ ($nodo \rightarrow hijos [i] = NULL) \impluego $  nodoValido($nodo \rightarrow hijos [i] $, e)
}


\tadOperacion{reconstruirClave}{puntero(Nodo)/nodo}{string}{}
\tadAxioma{reconstruirClave($nodo$)}{ \IF $nodo \rightarrow padre = NULL $ THEN <> ELSE  reconstruirClave($nodo \rightarrow padre$) $\circulito$ $Ord^{-1}$($nodo \rightarrow idx$) FI }



\subsubsection{Funci\'on de Abstracci\'on}

\Abs[estr]{diccString($\alpha$)}[e]{d}{
($\forall s$: string) ( (def?($s$, $d$) $\iff$ s $\in$ e.claves) $\yluego$ \\ 
	(def?($s$, $d$) $\impluego$ obtener($s$, $d$) $\igobs$ obtenerSignificado(s, e.raiz) ) 
}

\tadOperacion{obtenerSignificado}{String/s,puntero(Nodo)/nodo}{$\alpha$}{}
\tadAxioma{obtenerSignificado(s,nodo)}{ \IF Vacio?(s) THEN *($nodo \rightarrow info$) ELSE 
obtenerSignificado(Fin(s), $nodo \rightarrow hijos$[Ord(prim(s))]) FI }



\subsection{Algoritmos}

%\lstset{style=alg}


\begin{algorithm}[H]{\textbf{iCrearDiccionario}() $\to$ $res$ : estr}
	\begin{algorithmic}[1]
		\State estr $res$ $\gets$ $\langle$ CrearNodo(NULL, 0), Vacio(), Vacia(), Vacia() $\rangle$ \Comment O(1)
		\medskip
		\Statex \underline{Complejidad:} O(1)
		\Statex \underline{Justificación:} CrearNodo es en O(1) y crear una tupla de contenedores vacios es O(1).
    \end{algorithmic}
\end{algorithm}

\begin{algorithm}[H]{\textbf{iCrearNodo}(\In{padre}{puntero(Nodo)},\In{idx}{nat}) $\to$ $res$ : puntero(nodo)}
	{\\ $\textbf{Pre}$ $\equiv$ 0 $\leq$ idx < 256}
	\begin{algorithmic}[1]

		\State nodo n $\gets$ $\langle$ CrearArreglo(256), NULL, CrearIt(Vacio()), padre, idx $\rangle$  \Comment O(256) $\in$ O(1)
		\State nat i $\gets$ 0 					\Comment $O(1)$
		\While{i < 256} 								\Comment O(256) $\in$ O(1)
		 	\State n.hijos[i] $\gets$ NULL 		\Comment $O(1)$
			\State $i \gets i + 1$ \Comment $O(1)$
		\EndWhile
		\State res $\gets$ \&(n)          \Comment O(1)

		\medskip
		\Statex \underline{Complejidad:} O(1)
		\Statex \underline{Justificación:} Sumas de O(1)
    \end{algorithmic}
    {$\textbf{Post}$ $\equiv$ *res $\igobs$ $\langle$ CrearArreglo(256), NULL, CrearIt($\emptyset$), padre, idx $\rangle$ $\yluego$ \\ ($\forall$ i : Nat) i $<$ 256 $\impluego$ res$\rightarrow$ hijos[i] = NULL  }

\end{algorithm}


\begin{algorithm}[H]{\textbf{iBuscarYCrear}(\Inout{n}{puntero(Nodo)},\In{clave}{string},\In{crear?}{bool}) $\to$ $res$ : puntero(nodo)}
	{\\ $\textbf{Pre}$ $\equiv$ $n \igobs n_0$ $\land$ ($n_0 \neq$ NULL)  }
	\begin{algorithmic}[1]
		\State nat i $\gets$ 0  									\Comment O(1)

		\While{i $<$ longitud(clave) $\land$ $\lnot$(n = NULL) } 	\Comment O(long(clave)) 
			\State nat idx $\gets$ ord(clave[i])								\Comment O(1)
			\If{$n \rightarrow hijos$[idx] = NULL $\land$ crear?} 		\Comment O(1)
				\State $n \rightarrow hijos$[idx] $\gets$ CrearNodo(n, idx)		\Comment O(1)
			\EndIf	
					
			\State n $\gets$ $n \rightarrow hijos$[idx]		\Comment O(1)
			\State $i \gets i + 1$ 											\Comment O(1)
		\EndWhile
		\State res $\gets$ n					\Comment O(1)
		\medskip
		\Statex \underline{Complejidad:} O(long(clave))
		\Statex \underline{Justificación:} El ciclo se ejecuta a lo sumo una cantidad de veces igual a la longitud de la clave, y las operaciones que se realizan son O(1).
    \end{algorithmic}
    {$\textbf{Post}$ $\equiv$ \\
     \IF BuscaNodo($n_0$, $clave$) = NULL $\land$ crear? THEN   
     (n se modifica, creando recursivamente nodos nuevos donde n$\rightarrow$hijos[$c_i$] = NULL) $\yluego$
    	res $\igobs$ BuscaNodo($n$, $clave$) 
     ELSE 
     res $\igobs$ BuscaNodo($n_0$, $clave$) $\land$ n $\igobs$ $n_0$ 
     FI }
\end{algorithm}


    
\tadOperacion{BuscaNodo}{puntero(Nodo)/p, String/c}{puntero(Nodo)}{}

\tadAxioma{BuscaNodo(p,c)}{
	\IF p = NULL THEN NULL
	ELSE{
		\IF long(c) = 1 THEN p$\rightarrow$hijos[orden(prim(c))]
		ELSE BuscaNodo(p$\rightarrow$hijos[ord(prim(c))], ult(c))
		FI
	}
	FI
}


\begin{algorithm}[H]{\textbf{iDefinido?}(\In{d}{estr)}, \In{clave}{string}) $\to$ $res$ : $bool$}
	\begin{algorithmic}[1]
		\State puntero(nodo) n $\gets$ BuscarYCrear(d.raiz, clave, false)     \Comment O(long(clave))
		\State res $\gets$ $\lnot$(n = NULL) $\yluego$ $\lnot$(n $\rightarrow$ info = NULL)  \Comment O(1)
		\medskip
		\Statex \underline{Complejidad:} O(long(clave))
		\Statex \underline{Justificación:} La complejidad esta determinada por la operacion BuscarYCrear
    \end{algorithmic}
\end{algorithm}


\begin{algorithm}[H]{\textbf{iDefinir}(\Inout{d}{estr}, \In{clave}{string}, \In{s}{$\alpha$})}
	\begin{algorithmic}[1]

		\State puntero(nodo) n $\gets$ BuscarYCrear(d.raiz, clave, true)     \Comment $\Theta$(long(clave))
		\If{ d.info = NULL} 													\Comment O(1)
			\State d.info $\gets$ \&( copy(s)	)					\Comment $\Theta$(copy(s))
			\State d.clave $\gets$ AgregarRapido(d.claves, clave)			\Comment $\Theta$(copy(clave))
			\State actualizarMin(d)		\Comment O(max(longs(claves(d))))
			\State actualizarMax(d)		\Comment O(max(longs(claves(d))))
		\Else
			\State d.info $\gets$ NULL							\Comment O(1)
			\State d.info $\gets$ \&( copy(s)	)					\Comment $\Theta$(copy(s))
		\EndIf

		\medskip
		\Statex \underline{Complejidad:} O(L + copy(s)) Donde L es la clave mas larga definida en el diccionario.
		\Statex \underline{Justificación:} La complejidad esta dada por suma de otras operaciones, donde 
		$\Theta$(long(clave)) y $\Theta$(copy(clave)) $\in$ O(L), con lo cual la suma es entre las dos determinantes independientes L y copy(s) 
    \end{algorithmic}
\end{algorithm}



\begin{algorithm}[H]{\textbf{iObtener}(\In{d}{estr}, \In{clave}{string}) $\to$ $res$ : $\alpha$}
	\begin{algorithmic}[1]
		\State puntero(nodo) n $\gets$ BuscarYCrear(d.raiz, clave, false)     \Comment $\Theta$(long(clave))
		\State res $\gets$ *(n $\rightarrow$ info)									    \Comment O(1)
		\medskip
		\Statex \underline{Complejidad:} $\Theta$(long(clave))
		\Statex \underline{Justificación:} La complejidad esta determinada por la operacion BuscarYCrear ya que el resultado se devuelve por referencia, como para usar obtener tiene que estar definidio, siempre es $\Theta$(long(clave))
    \end{algorithmic}
\end{algorithm}


\begin{algorithm}[H]{\textbf{iBorrar}(\Inout{d}{estr}, \In{clave}{string})}
	\begin{algorithmic}[1]

		\State puntero(nodo) n $\gets$ BuscarYCrear(d.raiz, clave, false)     \Comment O(long(clave))
		\State $n \rightarrow info$ $\gets$ NULL 					\Comment O(1)
		\State eliminarSiguiente(n.clave)							\Comment se elimina la clave del conjunto de claves O(1)
		\State n.clave $\gets$ CrearIt(Vacio())				\Comment se apunta a null O(1)
		\While{$\lnot$(n = NULL) $\yluego$ $\lnot$(tieneHijos(n))}	\Comment por Rep, O(long(clave)) * O(1)
			\State puntero(nodo) padre $\gets$ $n \rightarrow padre$		\Comment O(1)
			\If{$\lnot$(padre = NULL)} 														\Comment O(1)
				\State padre[$n \rightarrow idx$] $\gets$ NULL			\Comment se elimina el nodo O(1)					
			\EndIf
			\State n $\gets$ padre								\Comment O(1)
		\EndWhile
	
		\State actualizarMin(d)		\Comment O(max(longs(claves(d))))
		\State actualizarMax(d)		\Comment O(max(longs(claves(d))))

		\medskip
		\Statex \underline{Complejidad:} O(L) Donde L es la clave mas larga definida en el diccionario.
		\Statex \underline{Justificación:} El algoritmo recorre desde el nodo hasta la raiz, eliminando si es necesario los nodos intermedios, esto se realiza a lo sumo O(long(clave)). De todas maneras como estamos actualizando los minimos y maximos y esto tiene una complejidad de O(L), y O(long(clave)) $\in$ O(L), esto determina la complejidad.
    \end{algorithmic}
\end{algorithm}



\begin{algorithm}[H]{\textbf{iCrearItClaves}(\In{d}{estr}) $\to$ $res$ : itConj(string)}
	\begin{algorithmic}[1]
		\State res $\gets$ CrearIt(d.claves)									    \Comment O(1)
		\medskip
		\Statex \underline{Complejidad:} O(1)
    \end{algorithmic}
\end{algorithm}


\begin{algorithm}[H]{\textbf{iClaveMinima}(\In{d}{estr}) $\to$ $res$ : string}
	\begin{algorithmic}[1]
		\State res $\gets$ d.min								    \Comment O(1)
		\medskip
		\Statex \underline{Complejidad:} O(1)
		\Statex \underline{Justificación:} Se devuelve por referencia	
    \end{algorithmic}
\end{algorithm}



\begin{algorithm}[H]{\textbf{iClaveMaxima}(\In{d}{estr}) $\to$ $res$ : string}
	\begin{algorithmic}[1]
		\State res $\gets$ d.max								    \Comment O(1)
		\medskip
		\Statex \underline{Complejidad:} O(1)
		\Statex \underline{Justificación:} Se devuelve por referencia		
    \end{algorithmic}
\end{algorithm}






\begin{algorithm}[H]{\textbf{iTieneHijos}(\In{nodo}{puntero(Nodo)}) $\to$ $res$ : $bool$}
	{\\ $\textbf{Pre}$ $\equiv$ nodo!=NULL}
	\begin{algorithmic}[1]
		\State nat i $\gets$ 0 \Comment $O(1)$
		\While{i $<$ 256 $\yluego$ $nodo \rightarrow hijos$[i]=NULL )} \Comment O(256) $\in$ O(1)
			\State $i \gets i + 1$ \Comment $O(1)$
		\EndWhile
		\State res $\gets$ i $<$ 256     \Comment $O(1)$

		\medskip
		\Statex \underline{Complejidad:} $O(1)$
		\Statex \underline{Justificación:} Recorre el arreglo de 256 posiciones en caso de que todas las posiciones del mismo tengan NULL. Como es una constante ya que en el peor caso siempre recorre a lo sumo 256 posiciones entonces es O(1).

    \end{algorithmic}
    {$\textbf{Post}$ $\equiv$ res $=_{obs}$  ($\exists i$) i < 256 $\yluego$ $\lnot$($nodo \rightarrow hijos$[i]=NULL) }
\end{algorithm}




\begin{algorithm}[H]{\textbf{iActualizarMin}(\Inout{d}{estr})}
	{\\ $\textbf{Pre}$ $\equiv$ $\lnot$(d=NULL) $\land$ Rep(d) }
	\begin{algorithmic}[1]
		\State d.min $\gets$ Vacio()					\Comment O(1)
		\If{longitud(d.claves) > 0} 								\Comment O(1)
			\State nodo n $\gets$ d.raiz						\Comment O(1)
			\While{$n \rightarrow info$ = NULL}		\Comment O(max(long(claves(d)))
				\State nat i $\gets$ 0								\Comment O(1)
				\While{i $<$ 256 $\land$ $n \rightarrow hijos$[i] = NULL} 		\Comment O(256) $\in$ O(1)
					\State i $\gets$ i + 1				\Comment O(1)
				\EndWhile
				\State n $\gets$ $n \rightarrow hijos$[i]			\Comment O(1)
			\EndWhile
			\State d.min $\gets$  copy(Siguiente(n.clave)) \Comment O(copy(clave))  
		\EndIf
		\medskip
		\Statex \underline{Complejidad:} O(L) Done L es la clave mas larga definida en el diccionario
		\Statex \underline{Justificación:} El ciclo principal va recorriendo los nodos hasta que encuentra el primer significado, esto se ejecuta a lo sumo L veces. Luego las operaciones del ciclio son O(1), por lo tanto O(L)*O(1) $\in$ O(L). Luego se copia la clave, pero esto tambien es O(L), por lo cual O(L) + O(L) = O(L)
    \end{algorithmic}
    {$\textbf{Post}$ $\equiv$ \IF \#(d.claves)=0 THEN d.min = <> ELSE  d.min $\in$ d.claves $\yluego$  $(\forall s:\text{string})$ s $\in$ d.claves $\land$ s $\neq$ d.min $\implies$ d.min $<$ s  FI  }
\end{algorithm}


\begin{algorithm}[H]{\textbf{iActualizarMax}(\Inout{d}{estr})}
	{\\ $\textbf{Pre}$ $\equiv$ $\lnot$(d=NULL) $\land$ Rep(d) }
	\begin{algorithmic}[1]
		\State d.max $\gets$ Vacio()					\Comment O(1)
		\If{longitud(d.claves) > 0} 								\Comment O(1)
			\State nodo n $\gets$ d.raiz						\Comment O(1)
			\While{tieneHijos(n)}									\Comment O(max(long(claves(d))) * O(1)
				\State nat i $\gets$ 0								\Comment O(1)
				\While{i $<$ 256 $\land$ $n \rightarrow hijos$[256 - i - 1] = NULL} 		\Comment O(256) $\in$ O(1)
					\State i $\gets$ i + 1				\Comment O(1)
				\EndWhile
				\State n $\gets$ $n \rightarrow hijos$[i]			\Comment O(1)
			\EndWhile
			\State d.max $\gets$  copy(Siguiente(n.clave)) \Comment O(copy(clave))  
		\EndIf
		\medskip
		\Statex \underline{Complejidad:} O(L)
		\Statex \underline{Justificación:} El ciclo principal va recorriendo los nodos hasta que encuentra el primer significado, esto se ejecuta a lo sumo L veces. Luego las operaciones del ciclio son O(1), por lo tanto O(L)*O(1) $\in$ O(L). Luego se copia la clave, pero esto tambien es O(L), por lo cual O(L) + O(L) = O(L)
    \end{algorithmic}
    {$\textbf{Post}$ $\equiv$ \IF \#(d.claves)=0 THEN d.max = <> ELSE d.max $\in$ d.claves $\yluego$  $(\forall s:\text{string})$ s $\in$ d.claves $\land$ s $\neq$ d.max $\implies$ d.max $>$ s  FI  }
\end{algorithm}


